%%%%%%%%%%%%%%%%%%%%%%%%%%%%%%%%%%%%%%%%%%%%%%%%%%%%%%%%%%%%%%%%%%%%%%%%%%%%%%%
%
%  Appendix 2
%
%  S. Scott Collis
%
%  Written: 10-22-95
%
%  Revised: 10-22-96
%
%%%%%%%%%%%%%%%%%%%%%%%%%%%%%%%%%%%%%%%%%%%%%%%%%%%%%%%%%%%%%%%%%%%%%%%%%%%%%%%
\chapter{Computation of Mesh Metrics \label{a:metrics}}

For generality, we assume that there are two levels of mapping used to
transform physical space to computational space.  These mapping can be written
abstractly as
%
\begin{equation}
  (x,y) \mapsto (\tilde\xi,\tilde\eta) \mapsto (\xi,\eta)
\end{equation}
%
An example of such a mapping is presented in Chapter \ref{c:pcyl} where a
conformal mapping transforms physical space to parabolic coordinates with an
additional mapping used to transform to a uniform computational space.  The
metric terms required by the flow solver (see \S\ref{s:gencoord}) to convert
derivatives in computational space to derivatives in physical space are
computed using the derivatives of the mapping functions along with the chain
rule.

For example, to compute the first derivative metrics [see equation
(\ref{e:jacobian})], the chain rule is used to obtain
%
\begin{eqnarray}
x_{,\xi}  &=&  x_{,\tilde\xi}  \, \tilde\xi_{,\xi}  + 
               x_{,\tilde\eta} \, \tilde\eta_{,\xi} \, ,  \label{e:map1} \\
y_{,\xi}  &=&  y_{,\tilde\xi}  \, \tilde\xi_{,\xi}  + 
               y_{,\tilde\eta} \, \tilde\eta_{,\xi} \, ,  \label{e:map2} \\ 
x_{,\eta} &=&  x_{,\tilde\xi}  \, \tilde\xi_{,\eta} + 
               x_{,\tilde\eta} \, \tilde\eta_{,\eta} \, , \label{e:map3} \\
y_{,\eta} &=&  y_{,\tilde\xi}  \, \tilde\xi_{,\eta} + 
               y_{,\tilde\eta} \, \tilde\eta_{,\eta} \, . \label{e:map4}
\end{eqnarray}
%
The right-hand-sides of equations (\ref{e:map1})--(\ref{e:map4}) are known
from the definitions of the mapping functions.  Thus, the inverse Jacobian is
given by
%
\begin{equation}
{\bf J}^{-1} = \left [ \matrix{x_{,\xi} & y_{,\xi} \cr
                               \noalign{\smallskip}
                               x_{,\eta} & y_{,\eta} \cr } \right ]
\end{equation}
%
and the determinant of the Jacobian is
%
\begin{equation}
|{\bf J}| = \frac{1}{x_{,\xi} \, y_{,\eta} - x_{,\eta} \, y_{,\xi}}.
\end{equation}
%
Taking the inverse, the Jacobian is obtained
%
\begin{equation}
{\bf J} \equiv \left [ \matrix{\xi_{,x} & \eta_{,x} \cr
                               \noalign{\smallskip}
                               \xi_{,y} & \eta_{,y} \cr } \right ] =
     |{\bf J}| \left [ \matrix{ y_{,\eta} & -y_{,\xi} \cr
                               \noalign{\smallskip}
                                -x_{,\eta} & x_{,\xi} \cr } \right ] .
\end{equation}
%
A similar procedure is used to determine the second derivative metrics [see
equation (\ref{e:hessian})] using the second derivatives of the mapping
functions along with the chain rule.  The results are
%
\newcommand{\xixx}{\xi_{,xx}}
\newcommand{\xixy}{\xi_{,xy}}
\newcommand{\xiyy}{\xi_{,yy}}
\newcommand{\etaxx}{\eta_{,xx}}
\newcommand{\etaxy}{\eta_{,xy}}
\newcommand{\etayy}{\eta_{,yy}}
%
\newcommand{\xix}{\xi_{,x}}
\newcommand{\xiy}{\xi_{,y}}
\newcommand{\etax}{\eta_{,x}}
\newcommand{\etay}{\eta_{,y}}
%
\newcommand{\xxixi}{x_{,\xi\xi}}
\newcommand{\xxieta}{x_{,\xi\eta}}
\newcommand{\xetaeta}{x_{,\eta\eta}}
%
\newcommand{\yxixi}{y_{,\xi\xi}}
\newcommand{\yxieta}{y_{,\xi\eta}}
\newcommand{\yetaeta}{y_{,\eta\eta}}
%
\newcommand{\dxdxi}{x_{,\xi}}
\newcommand{\dydxi}{y_{,\xi}}
\newcommand{\dxdeta}{x_{,\eta}}
\newcommand{\dydeta}{y_{,\eta}}
%
\newcommand{\rjac}{|{\bf J}|}
\newcommand{\drjacdxi}{|{\bf J}|_{,\xi}}
\newcommand{\drjacdeta}{|{\bf J}|_{,\eta}}
%
\newcommand{\dsdxi}{\tilde\xi_{,\xi}}
\newcommand{\dsdeta}{\tilde\xi_{,\eta}}
\newcommand{\drdxi}{\tilde\eta_{,\xi}}
\newcommand{\drdeta}{\tilde\eta_{,\eta}}
%
\newcommand{\dsdxidxi}{ \tilde\xi_{,\xi\xi} }
\newcommand{\dsdxideta}{ \tilde\xi_{,\xi\eta} }
\newcommand{\dsdetadeta}{ \tilde\xi_{,\eta\eta} }
\newcommand{\drdxidxi}{ \tilde\eta_{,\xi\xi} }
\newcommand{\drdxideta}{ \tilde\eta_{,\xi\eta} }
\newcommand{\drdetadeta}{ \tilde\eta_{,\eta\eta} }
%
\newcommand{\dxds}{x_{,\tilde\xi} }
\newcommand{\dxdr}{x_{,\tilde\eta} }
\newcommand{\dyds}{y_{,\tilde\xi} }
\newcommand{\dydr}{y_{,\tilde\eta} }
%
\newcommand{\dxdsds}{x_{,\tilde\xi\tilde\xi} }
\newcommand{\dxdsdr}{x_{,\tilde\xi\tilde\eta} }
\newcommand{\dxdrdr}{x_{,\tilde\eta\tilde\eta} }
\newcommand{\dydsds}{y_{,\tilde\xi\tilde\xi} }
\newcommand{\dydsdr}{y_{,\tilde\xi\tilde\eta} }
\newcommand{\dydrdr}{y_{,\tilde\eta\tilde\eta} }
%
\begin{eqnarray}
\xxixi &=& \dxdsds \dsdxi^2 + \dxds \dsdxidxi + 2 \dxdsdr \drdxi \dsdxi + 
  \dxdrdr \drdxi^2 + \dxdr \drdxidxi \\
\xxieta &=& \dxdsds \dsdeta \dsdxi + \dxds \dsdxideta + \dxdsdr \drdeta \dsdxi
  + \dxdsdr \drdxi \dsdeta + \dxdrdr \drdeta \drdxi + \dxdr \drdxideta \\ 
\xetaeta &=& \dxdsds \dsdeta^2 + \dxds \dsdetadeta + 2 \dxdsdr \drdeta 
  \dsdeta + \dxdrdr \drdeta^2 + \dxdr \drdetadeta \\
\yxixi &=& \dydsds \dsdxi^2 + \dyds \dsdxidxi + 2 \dydsdr \drdxi \dsdxi + 
  \dydrdr \drdxi^2 + \dydr \drdxidxi \\
\yxieta &=& \dydsds \dsdeta \dsdxi + \dyds \dsdxideta + 
  \dydsdr \drdeta \dsdxi + \dydsdr \drdxi \dsdeta + \dydrdr \drdeta \drdxi + 
  \dydr \drdxideta \\
\yetaeta &=& \dydsds \dsdeta^2 + \dyds \dsdetadeta + 
  2 \dydsdr \drdeta \dsdeta + \dydrdr \drdeta^2 + \dydr \drdetadeta
\end{eqnarray}
%
\begin{eqnarray}
\drjacdxi & = & -\left( \xxixi\dydeta + \dxdxi\yxieta - 
                      \xxieta\dydxi - \dxdeta\yxixi \right ) \rjac^2 \\
\drjacdeta & = & -\left( \xxieta\dydeta + \dxdxi\yetaeta - 
                       \xetaeta\dydxi - \dxdeta\yxieta \right) \rjac^2
\end{eqnarray}
%
\begin{eqnarray}
\xixx & = &   \;\; ( \drjacdxi \dydeta + \yxieta \rjac ) \xix + 
                   ( \drjacdeta \dydeta + \yetaeta \rjac )  \etax \\
\xixy & = &   \;\; ( \drjacdxi \dydeta + \yxieta \rjac ) \xiy + 
                   ( \drjacdeta \dydeta + \yetaeta \rjac ) \etay \\
\xiyy & = &   -( \drjacdxi \dxdeta + \xxieta \rjac ) \xiy - 
               ( \drjacdeta \dxdeta + \xetaeta \rjac ) \etay  \\
\etaxx & = &  -( \drjacdxi \dydxi + \yxixi \rjac ) \xix - 
               ( \drjacdeta \dydxi + \yxieta \rjac ) \etax \\
\etaxy & = &  -( \drjacdxi \dydxi + \yxixi \rjac ) \xiy - 
               ( \drjacdeta \dydxi + \yxieta \rjac ) \etay \\
\etayy & = &  \;\; ( \drjacdxi \dxdxi + \xxixi \rjac ) \xiy + 
                   ( \drjacdeta \dxdxi + \xxieta \rjac ) \etay
\end{eqnarray}

