%%%%%%%%%%%%%%%%%%%%%%%%%%%%%%%%%%%%%%%%%%%%%%%%%%%%%%%%%%%%%%%%%%%%%%%%%%%%%%%
%
%  Appendix 3:  Falkner--Skan--Cooke boundary layers
%
%  S. Scott Collis
%
%  Written: 9-5-95
%
%  Revised: 9-18-96
%
%%%%%%%%%%%%%%%%%%%%%%%%%%%%%%%%%%%%%%%%%%%%%%%%%%%%%%%%%%%%%%%%%%%%%%%%%%%%%%%
\chapter{Compressible Falkner--Skan--Cooke Solutions \label{a:FSC}}

Considerable research into the stability and receptivity characteristics of
three-dimen\-sional boundary layers has been conducted for the family of
incompressible yawed-wedge flows, often referred to as Falkner--Skan--Cooke
(FSC) flows \cite{Cooke:50}.  Under suitable assumptions, an analogous family
of self-similar compressible, three-dimensional boundary layers flows are
possible.  These flows were first studied by Reshotko \& Beckwith
\cite{ReBe:58} as a model of the flow near the stagnation-line of a yawed
cylinder.  In this appendix, we summarize the derivation of the compressible
FSC equations and present solutions in a form useful for stability/receptivity
calculations.

Using the Stewartson transformation along with a linear viscosity-temperature
relation and $\Pr=1$, the compressible boundary layer equations (subject to
isothermal, no-slip wall boundary conditions and edge conditions at infinity)
reduce to a system of two, ordinary differential equations.  The details of the
derivation are presented by Reshotko \& Beckwith \cite{ReBe:58} in the context
of the stagnation-line flow and here we only summarize the derivation with
suitable changes to consider more general three-dimensional boundary layers.
In the following, velocities are nondimensionalized by the free-stream
speed-of-sound, $c_\infty^*$, lengths by a convenient length scale $L^*$
(defined explicitly below), and all other quantities are nondimensionalized by
their freestream values.  Consider the following Falkner-Skan distributions of
inviscid slip velocities
%
\begin{equation} \label{e:uedge}
  u_e \equiv \frac{u^*_e}{c_\infty^*} = \M_\infty c_e(\tilde x) \, 
                                   {\tilde x}^m \cos\theta  \comma
\end{equation}
%
\begin{equation} \label{e:wedge}
  w_e \equiv \frac{w^*_e}{c_\infty^*}  = \M_\infty \sin\theta
\end{equation}
%
where $c_e(\tilde x) \equiv c^*_e / c^*_\infty$ is the local speed-of-sound at
the boundary layer edge.  The tilde denotes that $x$ has been transformed to
account for variable viscosity and density effects using the Stewartson
transformation \cite{Stewartson:64} which is defined as
%
\begin{equation}
  \tilde x \equiv \frac{\tilde x^*}{L^*} = 
	\int_0^{x} \frac{\mu_w T_0}{\mu_0 T_w}
	\left(\frac{c_e}{c_0}\right)^\frac{3\gamma-1}{\gamma-1} dx \comma
\end{equation}
%
\begin{equation}
  \tilde y \equiv \frac{\tilde y^*}{L^*} = \frac{c_e}{c_0} 
	\int_0^y \frac{\rho}{\rho_e} dy  \comma
\end{equation}
%
where the subscript ``$0$'' denotes freestream stagnation quantities while
subscript ``$w$'' denotes conditions at the wall.  Note that the quantity
$(\mu_w T_0)/(\mu_0 T_w)$ can be determined from Sutherland's law given
$T_0/T_w$ and for the special case of $T_0/T_w = 1$, $(\mu_w T_0)/(\mu_0 T_w)
= 1$.  Note that Sutherland's law is only used once to determine the ratio
$\mu_w/\mu_0$ given $T_0/T_w$.  The viscosity is then given by the linear
relationship
%
\begin{equation}
  \frac{\mu}{\mu_0} = \left(\frac{\mu_w T_o}{\mu_0 T_w}\right)
  \frac{T}{T_0} \period
\end{equation}

To convert the equations to a set of ordinary differential equations, a
similarity transformation is applied:
%
\begin{equation}
  \xi = \tilde x ,
\end{equation}
%
\begin{equation} \label{e:sim}
  \eta = \tilde y  
  	 \sqrt{ \frac{m+1}{2} \Re_c
	        \left(\frac{c_0}{c_e}\right)^\frac{2\gamma}{\gamma-1}
                \tilde x^{m-1} } \comma
\end{equation}
%
\begin{equation}
  u(\xi,\eta) = f'(\eta) \, u_e(\xi) \comma
\end{equation}
%
\begin{equation}
  w(\eta) = g(\eta) \, w_e  \comma
\end{equation}
%
\begin{equation}
  \frac{H - H_w}{H_e - H_w} = \phi(\eta) \comma
\end{equation}
%
where $H$ is the total enthalpy.  In equation (\ref{e:sim}) 
%
\begin{equation}
  \Re_c = \frac{u^*_e(\tilde x^* = L^*) \, L^*}{\nu^*_e(\tilde x^* = L^*)}
\end{equation}
%
is the chordwise Reynolds number, based on edge quantities at $\tilde x^* =
L^*$.

When $\Pr=1$, the similarity transformation converts the boundary layer
equations into the following system of ordinary differential equations
%
\begin{equation} \label{e:sxmon}
f''' + f f'' = \beta_h \left[ {f'}^2 - 1 - 
               \left( \frac{T_0}{T_{N0}} - 1 \right ) (1 - g^2) - 
               \left( \frac{T_w}{T_0} - 1 \right) 
	       \left(\frac{T_0}{T_{N0}}\right) (1 - \phi) \right] \comma
\end{equation}
%
\begin{equation} \label{e:szmon} 
g'' + f g' = 0  \comma
\end{equation}
%
\begin{equation} \label{e:senergy}
\phi'' + f \phi' = 0  \comma
\end{equation}
%
where primes denote differentiation with respect to $\eta$, the pressure
gradient (\ie\ Hartree) parameter is given by $\beta_h =
\frac{2m}{m+1}$, and the ratio $T_0/T_{N0}$, which combines the effects of
sweep-angle and Mach number (see \cite{ReBe:58}), is defined as
%
\begin{equation}
  \frac{T_0}{T_{N0}} = \frac{1 + \frac{\gamma-1}{2} \M^2_\infty}
                       {1 + \frac{\gamma-1}{2} \M^2_\infty \cos^2\theta} 
                       \period
\end{equation}
%
The above equations are solved subject to isothermal, no-slip wall boundary
conditions with edge conditions enforced at infinity.  After the Stewartson
and similarity transformations, these boundary conditions become
%
\begin{equation}
\begin{array}{llll}
  f(0) = 0 , &
  f'(0) = 0, &
  g(0) = 0 , &
  \phi(0) = 0 , \\[0.1in]
  f(\infty) = 1 , &
  g(\infty) = 1 , &
  \phi(\infty) = 1 \period
\end{array}
\end{equation}
%
We note that when $\Pr=1$ the spanwise momentum equation and the energy
equation (along with their boundary conditions) are identical such that one
equation can be dropped.  However, unlike the incompressible case, the
chordwise momentum and spanwise momentum (or energy) equations are coupled for
compressible flow and must be solved simultaneously.  In order for a
similarity solution to exist, the coefficients in equations (\ref{e:sxmon})
must be independent of $\xi$.  This occurs when $\beta_h$ is constant and the
wall is isothermal (\ie\ $T_w/T_0$ is constant) and the solutions presented
below satisfy both of these requirements.  See \cite{ReBe:58} for a discussion
of similarity when $\Pr \ne 1$.

Returning to the definitions of the local edge-velocities
(\ref{e:uedge},\ref{e:wedge}) we see that the local sweep-angle is given by
%
\begin{equation}
  \theta_e = \tan^{-1}\left( \frac{\tan\theta}
                                   {\tilde x^{-m} c_e} \right)
\end{equation}
%
and, since
%
\begin{equation}
  {c_e} = \frac{ 1 + \frac{\gamma-1}{2} \M^2_\infty \cos^2 \theta }
 	   { 1 + \frac{\gamma-1}{2} \M^2_\infty \tilde x^{2m} \cos^2 \theta } 
\end{equation}
%
the local sweep-angle is equal to the freestream sweep-angle, $\theta$, when
$\tilde x = 1$ .  This, in fact, is what motivates the specific form of the
edge velocities given in equations (\ref{e:uedge}) and (\ref{e:wedge}).  Thus,
the length scale $L^*$ can now be identified as the chordwise location where
the local sweep-angle equals the freestream sweep-angle and, the chordwise
Reynolds number can be explicitly evaluated as
%
\begin{equation}
  \Re_c = \frac{(U^*_\infty \cos\theta) L^*}{\nu^*_\infty}
\end{equation}
%
since $U^*_e = U^*_\infty$ when $\tilde x^* = L^*$.

Once the ODE's have been solved for $f(\eta)$ and $g(\eta)$, the primitive
variables at $\tilde x = 1$, nondimensionalized by edge quantities, are given
by
%
\begin{eqnarray}
  \frac{T}{T_e} = \left( 1 + \frac{\gamma-1}{2} \M^2_\infty \right) 
        \Bigg[ 1 + \left( \frac{T_w}{T_0} - 1 \right) (1 - \phi) - 
	\\ \nonumber
	\left(\frac{\frac{\gamma-1}{2} \M^2_\infty \cos^2\theta}
	     {1 + \frac{\gamma-1}{2} \M^2_\infty}\right) f'^2 -
	\left( 1 - \frac{T_{N0}}{T_0} \right) g^2 \Bigg] ,
\end{eqnarray}
%
\begin{equation}
  \frac{u}{U_e} = f' \cos\theta ,
\end{equation}
%
\begin{equation}
  \frac{w}{U_e} = g \sin\theta ,
\end{equation}
%
\begin{equation}
  \frac{\rho}{\rho_e} = \frac{T_e}{T} \period
\end{equation}

For stability calculations, it is useful to rotate the velocities to the local
streamline coordinates:
%
\begin{equation}
  \frac{u_s}{U_e} =  f' \cos^2\theta + g \sin^2\theta ,
\end{equation}
%
\begin{equation}
  \frac{w_s}{U_e} = (g - f') \cos\theta \sin\theta .
\end{equation}

Given the temperature profile, the physical $y$ coordinate is obtained from
%
\begin{equation}
  y \equiv \frac{y^*}{L^*} = 
  	\frac{\Psi}{\Re_c^{1/2}}\int_0^\eta \frac{T}{T_e} d\eta 
\end{equation}
%
where
%
\begin{equation}
  \Psi = \left( 1 + \frac{\gamma-1}{2} \M_\infty^2 \right)^
         {\frac{1}{2 - 2\gamma}}
         \sqrt{\frac{2}{m+1}} .
\end{equation}
%
The displacement thickness of the streamwise velocity component is computed
from the definition
%
\begin{equation}
  \delta_1 \equiv 
  \int_0^\infty \left( 1 - \frac{\rho u_s}{\rho_e U_e} \right) dy
\end{equation}
%
which, when converted to an $\eta$ integral, yields
%
\begin{equation}
  \delta_1 \equiv \frac{\delta^*_1}{L^*} =
	\frac{\Psi}{\Re_c^{1/2}} \int_0^\infty \left( \frac{T}{T_e} - 
        \frac{u_s}{U_e} \right) d\eta .
\end{equation}
%
The momentum thickness of the streamwise profile is computed in a similar
fashion using the definition
%
\begin{equation}
  \delta_2 \equiv \int_0^\infty \frac{\rho u_s}{\rho_e U_e}
                  \left( 1 - \frac{u_s}{U_e} \right) dy 
\end{equation}
%
leading to
%
\begin{equation}
  \delta_2 \equiv \frac{\delta^*_2}{L^*} = 
  	\frac{\Psi}{\Re_c^{1/2}} \int_0^\infty \frac{u_s}{U_e}
        \left( 1 - \frac{u_s}{U_e} \right) d\eta .
\end{equation}
%
The factors $\delta^*_1/L^*$ and $\delta_2^*/L^*$ are useful in converting
solutions to the thickness length-scales commonly used in
stability/receptivity analysis.  For example, if $\delta^*_1$ is the desired
length-scale, then
%
\begin{equation}
  \frac{y^*}{\delta^*_1} = \frac{y^*}{L^*} \frac{L^*}{\delta^*_1} =
  \frac{\int_0^\eta \frac{T}{T_e} d\eta}
                                {\int_0^\infty \left( \frac{T}{T_e} - 
        			 \frac{u_s}{U_e} \right) d\eta} \period
\end{equation}

In order to obtain solutions for $\tilde x = 1$, the following parameters must
be specified: $\M_e$, $T_w/T_0$, $\theta_e$, and $\beta_h$.  Then equations
(\ref{e:sxmon}) and (\ref{e:szmon}) are solved using fourth-order accurate
Runge-Kutta integration with a fixed step-size.  The integration is started
from the wall using assumed values of $f''(0)$ and $g'(0)$.  Then a Newton
iteration is performed to determine $f''(0)$ and $g'(0)$ such that the edge
boundary conditions are satisfied.  The infinite domain in $\eta$ is truncated
to $\eta=20$ which is over 5 boundary layer thicknesses above the wall.

Our motivation in computing FSC profiles is to provide a convenient analytical
set of boundary-layer profiles that can be used to model the profiles obtained
from full NS calculations.  As an example of matching a FSC solution to a
boundary-layer profile from a NS solution, we consider the flow over a
parabolic cylinder with the conditions $\M = 0.8$, $Re=1 \times 10^5$, $\theta
= 35^\circ$, $T_w/T_0 = 1$ and extract the boundary-layer profile at the point
of maximum crossflow, $s^*/r^*_n=0.804$, where $\delta^*_1/r^*_n = 3.842
\times 10^{-3}$.  At this location, the solution is projected to the
body-normal coordinate-system, $(s,n,z)$, and the local edge-conditions are
determined to be: $\M_e = 0.657$, $\theta_e = 54.4^\circ$, and $\beta_h =
0.766$ with $\delta^*_1/r^*_n = 3.842 \times 10^{-3}$.  Using these
parameters, a FSC profile is computed and converted to the coordinate system
used for the parabolic cylinder.  In figure \ref{f:upro} velocity profiles
from the NS and FSC solutions are compared demonstrating very good agreement.
The maximum crossflow for the NS profile is $-0.107$ while the FSC predicts a
value of $-0.103$ which is within 4\% of the NS solution.  Figure \ref{f:tpro}
compares the density and temperature profiles from the NS and FSC solutions
and again the agreement is good.  Also in good agreement is the shape factor,
$H \equiv \delta_1/\delta_2$, which is equal to $2.67$ for the FSC profile as
compared to the measured value of $2.66$ for the NS profile.

Using the linear stability solver discussed in Appendix \ref{a:LST}, the
chordwise wavenumber and growth-rates for a stationary cross-flow instability
have been computed for both the NS and FSC profile as a function of spanwise
wavenumber.  Figure \ref{f:wavnum} shows that predicted wavenumbers for both
profiles are in good agreement with a slight deviation at higher spanwise
wavenumbers.  The predicted grow-rates are shown in figure \ref{f:FSCgr} where
it is seen that the FSC profile has a lower peak growth-rate (by $\approx
2\%$), and the unstable range is shifted slightly to higher spanwise
wavenumbers.  The reduction in peak grow-rate is not surprising given the
slightly reduced level of crossflow in the FSC profile.  It is, of course,
possible to construct a FSC profile that better matches the NS solution by
tweaking some of the parameters ($\theta_e$ in particular).  The advantage of
the current approach is that it provides an unambiguous definition of the
parameters required to compute FSC profiles, based solely on boundary-layer
edge quantities and the displacement thickness.

In summary, we have presented the derivation of the compressible FSC
similarity equations and have obtained solutions in a form suitable for
stability and receptivity analysis of infinite-span swept-wings.  It has also
been demonstrated that when boundary-layer edge-conditions and displacement
thickness are matched to a local NS solution, FSC solutions are obtained which
possess similar stability characteristics.

%==============================================================================
%
%  Tables and Figures
%
%==============================================================================
%
\begin{figure}[p]
\centering
\sethlabel{$n$} 
\figlab 3.66in 2.4in {$u_s$} 
\figlab 3.66in 0.9in {$w_s$} 
\epsfxsize=5.5in \epsfbox{figures/ap3/upro.eps}
\caption[Comparison of FSC and NS boundary layer profiles for the local
streamwise and crossflow velocities at $s_{max}$] {Comparison of FSC
$\solid$ and NS $\dashed$ boundary layer profiles for the local streamwise
and crossflow velocities at $s_{max}$. Recall that $n = n^*/r^*_n$ is the
wall-normal coordinate nondimensionalized by the nose radius of the parabolic
cylinder. \label{f:upro} }
\end{figure}
%
\begin{figure}[p]
\centering
\sethlabel{$n$} 
\figlab 1.524in 1.0in {$\rho$} 
\figlab 1.524in 2.25in {$T$} 
\epsfxsize=5.5in \epsfbox{figures/ap3/tpro.eps}
\caption[Comparison of FSC and NS boundary layer profiles for the temperature
and density at $s_{max}$] {Comparison of FSC $\solid$ and NS $\dashed$
boundary layer profiles for the temperature and density at
$s_{max}$. \label{f:tpro} }
\end{figure}
%
%.... Stability comparison
%
\begin{figure}[p]
\centering
\setvlabel{$\alpha^*_r r^*_n$}
\sethlabel{$\beta^* r^*_n$} 
\epsfxsize=5.5in \epsfbox{figures/ap3/wavnum.eps}
\caption[Comparison of LST wavenumber predictions] {Comparison of LST
wavenumber predictions for the FSC $\solid$ and NS $\dashed$ boundary
layer profiles. \label{f:wavnum} }
\end{figure}
%
\begin{figure}[p]
\centering
\setvlabel{$\alpha^*_i r^*_n$}
\sethlabel{$\beta^* r^*_n$} 
\epsfxsize=5.5in \epsfbox{figures/ap3/gr.eps}
\caption[Comparison of LST growth-rate predictions] {Comparison of LST
growth-rate predictions for the FSC $\solid$ and NS $\dashed$ boundary
layer profiles. \label{f:FSCgr} }
\end{figure}
