%%%%%%%%%%%%%%%%%%%%%%%%%%%%%%%%%%%%%%%%%%%%%%%%%%%%%%%%%%%%%%%%%%%%%%%%%%%%%%%
%
%  Appendix 6
%
%  S. Scott Collis
%
%  Written: 9-5-95
%
%  Revised: 9-5-96
%
%%%%%%%%%%%%%%%%%%%%%%%%%%%%%%%%%%%%%%%%%%%%%%%%%%%%%%%%%%%%%%%%%%%%%%%%%%%%%%%
\chapter{Linearized Roughness Boundary Condition \label{a:bumpbc}}

Figure~\ref{f:bump} shows a schematic of the geometry under consideration.  In
this appendix, all lengths are nondimensionalized by a reference length-scale
$L^*$ which is the local boundary layer thickness at the roughness location.
This scaling leads to a meaningful perturbation expansion for small roughness
heights.  When using the results of this appendix, care must be taken to
properly convert the scaling for the problem at hand.  All other quantities,
$u_j$ and $T$, have been nondimensionalized by some convenient reference
values.

The smooth wall is denoted by $n(s,z)=0$, where $s$ is the wall tangent
coordinate, $n$ is the wall-normal coordinate, and $z$ is the spanwise
coordinate.  Note that this is a standard body-fitted coordinate system with
$n$ a straight-line coordinate normal to the surface.  The rough wall is
obtained by adding a small, spanwise-periodic perturbation given by
%
\begin{equation} \label{e:bump.w}
  n_w(s,z) = \ew h_w(s) e^{i \bw z}
\end{equation}
%
where $\ew = \ew^* / L^* \ll 1$ is the scaled roughness height, $h_w$ is the
streamwise shape function, and $\bw = \bw^* L^*$ is the spanwise wavenumber of
the bump.  Note that we have limited the study to spanwise periodic roughness
distributions.  Since the analysis is linear, the current results can be
thought of as the response of the boundary layer to a single Fourier mode of a
more complicated spanwise distribution.

The no-slip and isothermal boundary conditions are applied on the perturbed
surface
%
\begin{eqnarray} \label{e:bump.bc}
  u_j(s,n_w,z) = 0 , \\ 
  T(s,n_w,z) = T_w .
\end{eqnarray}
%
Similar to the expansions used in section~\ref{s:linNS}, the flow variables
are written as
%
\begin{eqnarray} \label{e:bump.var}
    u_j(x_i) &=& \bar u_j(x_i) + \ew \tilde u_j(x_i) + 
                 \O(\ew^2), \label{e:bump.u} \\
      T(x_i) &=& \bar T(x_i)   + \ew \tilde T(x_i) +   
                 \O(\ew^2), \label{e:bump.t}
\end{eqnarray}
%
where perturbations are denoted with a tilde.

Linearized boundary conditions, which can be applied at the unperturbed wall,
are derived by expanding equations (\ref{e:bump.bc}) in a Taylor series about
the undisturbed wall location.  This yields
%
\begin{eqnarray} \label{e:bump.taylor}
  u_j(s,n_w,z) = 0 &=& u_j(s,0,z) + \ew h_w(s) e^{i \bw z} 
                   \frac{\partial u_j}{\partial n}(s,0,z) + \O(\ew^2) \\
    T(s,n_w,z) = T_w &=& T(s,0,z) + \ew h_w(s) e^{i \bw z} 
                  \frac{\partial T}{\partial n}(s,0,z) + \O(\ew^2) .
\end{eqnarray}
%
Then using equations (\ref{e:bump.var}) in equation (\ref{e:bump.taylor}) gives
%
\begin{eqnarray}
  \bar u_j(s,0,z)   &=& 0, \\
  \bar T(s,0,z)     &=& T_w, \\
  \tilde u_j(s,0,z) &=& -h_w(s) e^{i \bw z} 
                        \frac{\partial \bar u_j}{\partial n}(s,0,z), \\
  \tilde T(s,0,z)   &=& -h_w(s) e^{i \bw z} 
                        \frac{\partial \bar T}{\partial n}(s,0,z), 
\end{eqnarray}
%
where only terms of $\O(\ew)$ have been retained.  The mean flow solution is
obtained using homogeneous boundary conditions on the unperturbed surface
while the disturbance boundary conditions are inhomogeneous and proportional
to the local, mean-flow gradients normal to the surface.

If instead of an isothermal boundary condition, a constant heat-flux condition
is used then the derivation of the linearized boundary conditions is
substantially more involved.  This is due to the fact that the normal
direction used in determining the wall-normal temperature gradient must be
defined relative to the perturbed surface:
%
\begin{equation}
 \frac{\partial T}{\partial \tilde n}(s,n_w,z) = g_w
\end{equation}
%
where $\tilde n$ is the perturbed normal direction and $g_w$ is the prescribed,
constant heat-flux.  

Since we wish to apply these conditions for an infinite-span wing, the
derivation is limited to a two-dimensional cylinder.  To derive the proper
linearized heat-flux boundary condition we begin by assuming that the
unperturbed surface is parametrically described by the arc-length, $s$.  Thus,
%
\begin{equation}
  (x, y) = ( f(s), g(s) ),
\end{equation}
%
as shown in figure~\ref{f:bump}.  The unit tangent and normal vectors are then
given, respectively, by
%
\begin{eqnarray}
  \hat s = \frac{(f',g')}{\sqrt{f'^2 + g'^2}}  \comma \\
  \hat n = \frac{(-g',f')}{\sqrt{f'^2 + g'^2}} \comma
\end{eqnarray}
%
where primes denote differentiation with-respect-to $s$.  The perturbed
surface is denoted by $(\tilde x, \tilde y)$ where
%
\begin{equation}
  \tilde x(s,z) = f(s) - \ew h_w(s) e^{i \bw z} 
                         \frac{g'}{\sqrt{f'^2 +g'^2}} \comma
\end{equation}
%
\begin{equation}
  \tilde y(s,z) = g(s) + \ew h_w(s) e^{i \bw z} 
                         \frac{f'}{\sqrt{f'^2 +g'^2}} \period
\end{equation}

To construct the normal to the perturbed surface we begin by defining two
vectors which are tangent to the surface.  For this, we need derivatives of
$\tilde x$ and $\tilde y$ with respect to $s$ and $z$.
%
\begin{equation}
  \frac{\partial \tilde x}{\partial s} = f' - \frac{\ew \e^{i\bw z}}
  {\sqrt{f'^2 +g'^2}} \left[ h'_w g' + h_w \left\{ g'' - 
  \frac{g'(f'f'' + g'g'')}{(f'^2 +g'^2)} \right\} \right] \comma
\end{equation}
%
\begin{equation}
  \frac{\partial \tilde y}{\partial s} = g' - \frac{\ew \e^{i\bw z}}
  {\sqrt{f'^2 +g'^2}} \left[ h'_w f' + h_w \left\{ f'' - 
  \frac{f'(f'f'' + g'g'')}{(f'^2 +g'^2)} \right\} \right] \comma
\end{equation}
%
\begin{equation}
  \frac{\partial \tilde x}{\partial z} = -\ew h_w i 
  \bw e^{i \bw z} \frac{g'}{\sqrt{f'^2 +g'^2}} \comma
\end{equation}
%
\begin{equation}
  \frac{\partial \tilde y}{\partial z} = \ew h_w i 
  \bw e^{i \bw z} \frac{f'}{\sqrt{f'^2 +g'^2}} \comma
\end{equation}
%
In the following derivation, these expressions are abbreviated to
%
\begin{eqnarray}
  \tilde x_{,s} &=& f' + \ew A \comma \label{e:shortxs} \\
  \tilde y_{,s} &=& g' + \ew B \comma \label{e:shortys} \\
  \tilde x_{,z} &=& \ew C      \comma \label{e:shortxz} \\
  \tilde y_{,z} &=& \ew D      \comma \label{e:shortyz}
\end{eqnarray}
%
where $A$, $B$, $C$, and $D$ are complex functions of $(s,z)$.

Using these abbreviated expressions, two vectors, each tangent to the
perturbed surface, can be defined as
%
\begin{equation} 
S_{xy} = \left\{
\begin{array}{c}
\tilde x_{,s} \\ \tilde y_{,s} \\ 0
\end{array}
\right\}  \qquad {\rm and} \qquad
Z_{xy} = \left\{
\begin{array}{c}
\tilde x_{,z} \\ \tilde y_{,z} \\ 1
\end{array}
\right\} \period
\end{equation}
%
The perturbed normal vector is then given by the cross product, $\tilde n =
S_{xy} \times Z_{xy}$ which is found to be
%
\begin{equation}
\tilde n = \left\{
\begin{array}{c}
-\tilde y^\dagger_{,s} \\ \tilde x^\dagger_{,s} \\ 
(\tilde x^\dagger_{,z} \tilde y^\dagger_{,s} - 
 \tilde x^\dagger_{,s} \tilde y^\dagger_{,z})
\end{array}
\right\} \period
\end{equation}
%
Rewriting the normal vector using equations (\ref{e:shortxs}-\ref{e:shortyz})
yields
%
\begin{equation}
\tilde n = \left\{
\begin{array}{c}
-(g' + \ew B^\dagger) \\ 
 (f' + \ew A^\dagger) \\ 
\ew \left[ C^\dagger (g' + \ew B^\dagger) - 
             D^\dagger (f' + \ew A^\dagger) \right]
\end{array}
\right\} \period
\end{equation}
%
The normalization of this vector requires the quantity $1/|\tilde n|$, which,
when expanded for small $\ew$, is given by
%
\begin{equation}
  \frac{1}{|\tilde n|} = \overbrace{\frac{1}{\sqrt{f'^2 + g'^2}}}^E \: + \:
                         \ew \overbrace{\frac{-( f' A_r + g' B_r )}
			 {(f'^2 + g'^2)^{3/2}}}^F \: + \: \O(\epsilon^2_w)
\end{equation}
%
where the terms $E$ and $F$ have been defined to simplify the notation.  Using
this expression to normalize $\tilde n$ and expanding the result in powers of
$\ew$ leads to
%
\begin{equation}
\frac{\tilde n}{|\tilde n|} = \left\{
\begin{array}{c}
-g' E - \ew (g' F + B^\dagger E) \\ 
 f' E + \ew (f' F + A^\dagger E) \\ 
\ew ( C^\dagger g' - D^\dagger f' )
\end{array}
\right\} \: + \: \O(\epsilon^2_w) \period
\end{equation}
%
Projecting this vector onto the undisturbed tangent and normal vectors yields
the following expression
%
\begin{eqnarray}
\frac{\tilde n}{|\tilde n|} &=& 
    -\ew \frac{h'_w e^{-i \bw z}}{f'^2 + g'^2} \: \hat s +
    \left\{ 1 - \ew i h_w \sin(\bw z) 
    \frac{g'f''-f'g''}{(f'^2 + g'^2)^{3/2}} \right\} \: \hat n +  \nonumber \\
& & \ew i h_w \bw e^{-i \bw z} \: \hat z \ + \ \O(\ew^2) \period
\end{eqnarray}
%
For convenience, this expression is rewritten using the shorthand notation
%
\begin{equation}
\frac{\tilde n}{|\tilde n|} = 
    \ew \tilde n_s \hat s +
    \left\{ 1 + \ew \tilde n_n \right\} \hat n +
    \ew \tilde n_z \hat z \ + \ \O(\ew^2) \period
\end{equation}

We can now evaluate the temperature gradient in the perturbed normal-direction
$\partial T / \partial \tilde n$ in terms of the gradients normal and tangent
to the original body,
%
\begin{equation}
\frac{\partial T}{\partial \tilde n} = 
  \ew \tilde n^\dagger_s \frac{\partial T}{\partial s} +
  (1 + \ew \tilde n^\dagger_n) \frac{\partial T}{\partial n} +
  \ew \tilde n^\dagger_z \frac{\partial T}{\partial z} \ + \ 
  \O(\ew^2) \period
\end{equation}
%
To evaluate this gradient at $n=n_w$, derivatives are expanded in a Taylor
series about $n=0$.  Using equation (\ref{e:bump.t}) and collecting terms of
equal order in $\ew$ yields
%
\begin{equation}
  \frac{\partial \bar T}{\partial n}(s,0,z) = q_w \comma
\end{equation}
%
\begin{eqnarray}
  \frac{\partial \tilde T}{\partial n}(s,0,z) &=& 
      -\tilde n^\dagger_s \frac{\partial \bar T}{\partial s}(s,0,z) -
       \tilde n^\dagger_n \frac{\partial \bar T}{\partial n}(s,0,z) - 
  \nonumber\\
   & & h_w e^{i \bw z} \frac{\partial^2 \bar T}{\partial n^2}(s,0,z) -
       \tilde n^\dagger_z \frac{\partial \bar T}{\partial z}(s,0,z) \period
\end{eqnarray}
%
Writing this expression out in full results in
%
\begin{eqnarray} \label{e:bump.tn}
  \frac{\partial \tilde T}{\partial n}(s,0,z) &=& 
         \frac{h'_w e^{i \bw z}}{f'^2 + g'^2} 
         \frac{\partial \bar T}{\partial s}(s,0,z)  - \nonumber \\
     & & i h_w \sin(\bw z)\frac{g'f''-f'g''}{(f'^2 + g'^2)^{3/2}}
         \frac{\partial \bar T}{\partial n}(s,0,z) - \nonumber \\
     & & h_w e^{i \bw z} \frac{\partial^2 \bar T}{\partial n^2}(s,0,z) +
         i h_w \bw e^{i \bw z}
         \frac{\partial \bar T}{\partial z}(s,0,z) \period
\end{eqnarray}

Equation (\ref{e:bump.tn}) is valid for any two-dimensional body with a small,
spanwise periodic perturbation.  For all the cases considered here, $\partial
\bar T/\partial z = 0$ so that the last term can be ignored.  From the general
expression, several special cases follow directly:

\medskip
\noindent{\it Two-dimensional disturbance: $\bw = 0$}

\begin{eqnarray} \label{e:2dtn}
  \frac{\partial \tilde T}{\partial n}(s,0,z) &=& 
         \frac{h'_w}{f'^2 + g'^2} 
         \frac{\partial \bar T}{\partial s}(s,0,z) - 
	 h_w \frac{\partial^2 \bar T}{\partial n^2}(s,0,z) \comma
\end{eqnarray}

\medskip
\noindent{\it Flat Plate}

\begin{eqnarray} \label{e:fptn}
  \frac{\partial \tilde T}{\partial n}(s,0,z) &=& 
         h'_w e^{i \bw z} 
         \frac{\partial \bar T}{\partial s}(s,0,z)  -
         h_w e^{i \bw z} \frac{\partial^2 \bar T}{\partial n^2}(s,0,z) \comma
\end{eqnarray}

\medskip
\noindent{\it Flat Plate with parallel flow assumption}

\begin{eqnarray} \label{e:fpptn}
  \frac{\partial \tilde T}{\partial n}(s,0,z) &=& 
          - h_w e^{i \bw z} \frac{\partial^2 \bar T}{\partial n^2}(s,0,z) 
  \period
\end{eqnarray}

%==============================================================================
%
%  Tables and Figures
%
%==============================================================================

\begin{figure}[p]
\centering
\epsfxsize=4.0in \epsfbox{figures/ap6/bump.ai}
\caption [Schematic of surface roughness geometry]{Schematic of surface
roughness geometry. Note that the height of the bump is exaggerated for
clarity. \label{f:bump}}
\end{figure}

