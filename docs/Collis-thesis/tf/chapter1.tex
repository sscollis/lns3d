%%%%%%%%%%%%%%%%%%%%%%%%%%%%%%%%%%%%%%%%%%%%%%%%%%%%%%%%%%%%%%%%%%%%%%%%%%%%%%%
%
%  Chapter 1:  Introduction
%
%  S. Scott Collis
%
%  Written: 9-5-95
%
%  Revised: 7-18-96
%
%%%%%%%%%%%%%%%%%%%%%%%%%%%%%%%%%%%%%%%%%%%%%%%%%%%%%%%%%%%%%%%%%%%%%%%%%%%%%%%
\chapter{Introduction \label{c:intro}}

Receptivity is the process by which environmental disturbances are converted
into instability waves in a boundary or shear layer.  In this manner,
receptivity provides the initial amplitudes of instability waves which
eventually lead to transition from laminar to turbulent flow.  An
understanding of receptivity is fundamental to the development of reliable
transition prediction schemes which, in turn, are necessary for the
development of laminar-flow-wing technology.

The research described in this \thesis\ focuses on the receptivity of the
three-dimensional boundary layer near the leading-edge of a high-speed
swept-wing.  In particular, the influence of surface roughness near the
leading-edge on the formation of crossflow (CF) vortices is examined.  Recent
experiments \cite{ReSaCaCh:96,DeBi:96} have demonstrated that surface
roughness is the primary cause of stationary CF vortices and that stationary
vortices dominate the transition process, near the leading-edge, in
low-disturbance environments---such as the cruise condition for an aircraft in
flight.  Recently, the linear and nonlinear evolution of CF vortices has been
well-predicted by analysis based on the Parabolized-Stability-Equations
\cite{HaRe:96,MaLiCh:94}.  However, the receptivity process for CF vortices
remains an open issue mainly because experimental studies of receptivity due
to surface roughness are hampered by several factors: (1) the complete
disturbance environment is generally unknown, (2) the amplitudes of the
crossflow vortices near the source of receptivity are not measurable, and (3)
at downstream stations where measurements are possible, nonlinear effects may
have already occurred.  Through the numerical simulations presented here, each
of these limitations are overcome and receptivity results are obtained that
can be used to test recent theoretical predictions
\cite{Choudhari:94,Crouch:93}.

This chapter begins with the general motivation for the study of receptivity
along with a description of the laminar-to-turbulent transition process on a
swept-wing and a summary of transition prediction techniques used in wing
design.  Included is a brief review of the voluminous literature on
receptivity including theoretical, experimental, and computational studies.
An attempt is made to focus the review on the receptivity of three-dimensional
boundary layers to surface roughness, and where appropriate, references are
provided to the many review articles in the literature.  The chapter closes
with a summary of the accomplishments of this research and an overview of the
material discussed herein.

\section{Motivation: Laminar Flow Technology \label{s:motive}}

Research on laminar-flow technology has been underway for over 50 years.
However, as fuel resources become more scarce and manufacturing techniques
improve, both the incentive and practicality of incorporating laminar-flow
technology in new aircraft designs has increased.  Possible applications of
laminar-flow wings run the gamut of aircraft types---from general aviation to
super-sonic passenger aircraft \cite{BrMaBaWaCo:90}.  The primary incentive in
applying laminar-flow technology stems from projections that maintaining
laminar flow over a large portion of a wing can provide as much as a 25\% to
30\% fuel savings on long-range flights \cite{ReHoKoQu:86}.

Laminar-flow wings can be broadly classified into two categories:
Natural Laminar Flow (NLF) wings for which the pressure distribution is
carefully tailored to delay transition, and Laminar Flow Control (LFC) wings
which use surface suction to delay transition.  Examples of both types of
laminar-flow wings are shown in figure \ref{f:lfc}. The wing-type used depends
on the particular application with NLF advocated for small, low-speed aircraft
with small wing-sweep and LFC used for large, high-speed, swept-wing aircraft.
For medium-sized aircraft with moderate wing-sweep ($\sim 20^\circ$), recent
designs take a hybrid approach by incorporating LFC near the leading-edge with
NLF in the mid-chord region in an effort to achieve drag reduction more
economically \cite{BrMaBaWaCo:90}.

Regardless of the particular type, the success of a laminar-flow wing depends
critically on the designer's ability to predict the location of
laminar-to-turbulent transition.  The transition process typically begins with
the conversion of environmental disturbances into the instability waves of the
flow by a process called receptivity \cite{Morkovin:69}.  This \thesis\ is
motivated by the need to increase the understanding of receptivity and to use
this knowledge to improve transition prediction.

\section{Transition Mechanisms on a Swept Wing \label{s:transition}}

Transition in many practical situations can be modeled as a three-stage
process: receptivity, linear amplification, and non-linear breakdown.
Receptivity involves the conversion of environmental disturbances into the
instability waves of the flow.  The receptivity process typically occurs in
the ``subcritical'' region, upstream of the location where instabilities begin
to grow.  Because of this, the waves excited in the receptivity phase often
decay before undergoing amplification.  Both the decay and subsequent
amplification are generally linear processes that are well predicted by
quasi-parallel Linear-Stability-Theory (LST).  As the instability waves
continue to amplify downstream, they eventually reach sufficiently large
magnitudes such that nonlinear effects become important.  These nonlinearities
manifest themselves, at least initially, as secondary and tertiary
instabilities.  This final phase, non-linear breakdown, usually occurs
rapidly, over a short streamwise length.  In fact, the length of the linear
amplification region is often an order-of-magnitude greater than the length of
the nonlinear region (at least for two-dimensional instabilities).  This fact
is the rationale behind current transition prediction schemes which are based
solely on the linear amplification stage, see \S\ref{s:prediction}.  Recently,
a new transition prediction technique based on the Parabolized Stability
Equations (PSE) \cite{Herbert:91} has been developed which yields accurate
predictions for both the linear amplification and early nonlinear stages of
transition.  We will return to the topic of PSE methods in
\S\ref{s:prediction}.  Transition phenomena which do not fit into this
three-stage model are called by-pass transition and are characterized by
nonlinearities from the onset.  Examples of by-pass transition include
attachment-line contamination and transition due to large surface
protuberances or high-intensity free-stream turbulence.

The transition process in the three-dimensional boundary layer on a swept-wing
typically involves one or more of the following phenomena:
%
\begin{enumerate}
  \item {Attachment-line instabilities and contamination}
  \item {Crossflow vortices: stationary and non-stationary}
  \item {Streamwise instabilities (\ie\ Tollmien--Schlichting waves)}
  \item {G\"ortler (centrifugal) instabilities }
\end{enumerate}
%
A schematic of possible transition routes for a swept-wing aircraft is given
in figure \ref{f:transition}.

Transition can begin on the attachment-line due to contamination from the
turbulent fuselage boundary layer at the wing/fuselage junction.  This by-pass
mode of transition is due to a subcritical (\ie\ finite amplitude) instability
of the attachment-line flow which can be catastrophic since it results in a
turbulent boundary layer over the entire wing surface.  However, contamination
can be prevented using a leading-edge notch or a Gaster bump \cite{Gaster:65}
and, in the following discussion, we assume that such a device is employed.

In addition to contamination, the attachment-line boundary layer is
susceptible to a viscous, Tollmien--Schlichting (TS) type, instability.  Thus,
on a laminar-flow wing, the attachment-line boundary-layer must be maintained
subcritical by careful selection of the leading-edge radius or by surface
suction.  Assuming this to be the case, transition on the upper (convex)
surface of a swept-wing is typically initiated by crossflow vortices in the
favorable pressure gradient region near the leading-edge.  The crossflow
vortices result from an inviscid instability of the inflectional crossflow
velocity profile (see figure \ref{f:transition}), called the crossflow (CF)
instability.  If transition is delayed near the leading-edge (typically by
surface suction) then transition typically occurs in the mid-chord region due
to the TS instability.  This fact motivates the Hybrid Laminar-Flow-Control
(HLFC) approach, with suction used near the leading-edge to control the CF
vortices and a favorable pressure gradient used in the mid-chord region to
weaken the TS instability.

On the lower surface of the wing, an additional instability mechanism comes
into play due to the concave curvature of the surface and streamlines.  This
centrifugal instability, called the G\"ortler instability, results in the
formation of counter-rotating streamwise vortex pairs (in contrast to the
crossflow instability which results in vortices with the same sense of
rotation).  Recently, there has been a number of investigations dealing with
the combined effects of G\"ortler and CF vortices
\cite{BaHa:91,Bassom:92,ZuMa:95}.  In the current investigation only convex
surfaces are considered thereby avoiding the G\"ortler instability.

\section{Transition Prediction Techniques \label{s:prediction}}

Current transition prediction schemes (\eton\ methods) are based solely on the
linear instability phase of transition, ignoring both the external disturbance
environment and nonlinear effects.  Both Malik \cite{Malik:90a} and Redeker
\etal \cite{ReHoKoQu:86} outline the origins of the \eton\ method (dating back
to 1952) which has since been used extensively in laminar-flow-wing design.

%These methods have been widely used in laminar-flow-wing design dating back to
%1952 and the history and development of the \eton\ method is outlined by Malik
%\cite{Malik:90a} and Redeker \etal \cite{ReHoKoQu:86}.  The origins of the
%\eton\ method date back to 1952 when the method was first used by Smith
%\cite{Smith:52} in studying the G{\"o}rtler instability and later applied to
%the TS instability by Smith \& Gamberoni \cite{SmGa:56} and Van Ingen
%\cite{VanIngen:56}.

The basic premise of the \eton\ method is that the transition location can be
correlated with the amount of linear amplification of disturbances in the
boundary layer.  The $N$ factor is given by the the integral of the spatial
amplification rate, $\sigma$, from the point where disturbances first become
unstable (Branch I, denoted by $x_I$) to the transition location $x_{tr}$:
%
\begin{equation}
  N = \int_{x_I}^{x_{tr}} \sigma(x) \ dx .
\end{equation}
%
Thus, the ratio of the amplitude of disturbances at transition to that at
Branch I is given by \eton.  By comparing experimental transition results in
low disturbance environments with $N$-factors computed using LST, it was found
for two-dimensional flows that transition occurs when $N=10$ \cite{JaOkSm:70}.
Because the $N$-factor represents a ratio of amplitudes, this method is
insensitive to the external disturbance environment, thereby requiring
different values of $N$ for different levels and types of environmental
disturbances.  In addition, different values of $N$ are expected for each type
of instability mechanism: TS, CF, and G\"ortler instabilities.  However, Malik
\cite{Malik:90a}, in providing a comprehensive review of the \eton\ results in
the literature (as of 1990), finds that a value of $N=10$ correlates
reasonably well with the majority of the data, regardless of the instability
mechanism.  It is this ``robustness'' which helps make the method a useful
design tool.

Given the apparent success of the \eton\ method, there is, however, a large
amount of scatter in the values of $N$ given in the literature.  It is not
unusual for $N$-factors to be reported in the range of $6-13$ depending on the
instability mechanisms, test environment, and flow geometry.  To put this in
perspective, this range of $N$-factors corresponds to a three
order-of-magnitude range of amplitude ratios.  Thus, large safety factors are
required in laminar flow wing design to account for this considerable
uncertainty in the $N$-factor.

The success of the \eton\ method is hampered by the assumptions of linearity,
parallel flow, and by the exclusion of the receptivity process which
determines the initial amplitude of disturbances within the boundary layer.
By including only the linear amplification stage of transition, the \eton\
method tacitly assumes that disturbances are of sufficiently low amplitude so
that there is a long region of linear growth which dominates the transition
process.  By ignoring the receptivity phase, the \eton\ method misses a
fundamental characteristic of the transition process: the transition location
moves upstream with increased disturbance levels.  Furthermore, nonlinear
effects in the form of nonlinear receptivity and nonlinear interactions are
not accounted for.  Thus, situations involving large initial disturbances
and/or modal (\eg\ TS and CF) interactions and mean flow modification are not
captured in the \eton\ predictions.

There are currently numerous research efforts underway to address the
limitations of the \eton\ method by developing an amplitude based transition
criterion which uses the PSE approach developed by Herbert and coworkers
\cite{Herbert:91}.  These efforts include studies of both nonlinear effects
and receptivity to address the deficiencies of the \eton\ method.  The
research discussed in this \thesis\ focuses on the numerical simulation of the
linear receptivity phase of transition with the hope that the results and
methodology discussed here can be used toward the development of an
amplitude-based transition prediction technique.

\section{Synopsis of Previous Receptivity Results \label{s:previous}}

Since 1969, when Morkovin \cite{Morkovin:69} first coined the term
``receptivity,'' there has been a tremendous amount of research to discover
and identify the mechanisms which lead to the formation of instability waves
in a boundary layer.  In this section, salient receptivity and transition
research is summarized including experimental, theoretical, and computational
results.  The discussion begins with a general overview of receptivity
research and then specializes to research involving the generation of
crossflow vortices in three-dimensional boundary layers.

Kerschen \cite{Kerschen:89} has identified two categories of receptivity:
forced and natural.  In forced receptivity, the external disturbance has the
appropriate length and time scales so that it can couple directly with an
instability wave in the boundary layer.  An early example of forced
receptivity is the landmark experiment of Schubauer \& Skramstad
\cite{ScSk:47} in which TS waves were generated by a vibrating ribbon placed
in the boundary layer.  Other examples include surface suction/blowing,
surface heating/cooling, and surface roughness.  We note in passing that both
the length and time scales of the disturbance must match that of the
instability wave such that surface roughness, which is generally regarded as
steady, may only directly excite stationary modes such as stationary crossflow
vortices or G\"ortler vortices.

In contrast to the forced case, natural receptivity involves external
disturbances which have the correct time-scale but require some form of
wavelength conversion to couple with an instability wave.  Examples include
freestream acoustic waves and freestream turbulence, which typically have
longer wavelengths then boundary-layer instability waves at the same
frequency.  The wavelength reduction process was first clarified in the
seminal work of Goldstein \cite{Goldstein:83} who showed, using high Reynolds
number asymptotic expansions, that the nonparallel mean flow near a
leading-edge can lead to the transfer of energy from long wavelength
freestream acoustic waves to Tollmien--Schlichting waves in the boundary layer.
Since then, natural receptivity has been found to occur near other forms of
nonparallel mean flows including: surface roughness
\cite{Ruban:84,Goldstein:85,ZhFe:87,ChKe:90}, marginal separation
\cite{GoLeCo:87}, sudden changes in surface curvature \cite{GoHu:87}, and
variations in surface admittance to pressure waves \cite{Choudhari:93}.  In
general, the results of experiments and the asymptotic theory have been in
good agreement \cite{Kerschen:89,KoRy:90}.  Most importantly, both have
demonstrated that regions of meanflow distortion which have a spatial spectrum
consonant with an instability wave are the key to the transfer of energy from
freestream disturbances to instability waves.

Readers interested in a more detailed view of receptivity research may wish to
consult the many available review articles.  In particular, the reviews by
Goldstein \& Hultgren \cite{GoHu:89} and Kerschen \cite{Kerschen:89} provide
excellent summaries of the high Reynolds number asymptotic theory results.
The recently popular finite Reynolds number theory is reviewed by Choudhari \&
Streett \cite{ChSt:94} and Crouch \cite{Crouch:94}, while, experimental
research as of 1986 is reviewed by Nishioka \& Morkovin \cite{NiMo:86} and
includes a summary of the extensive Soviet receptivity research.  The Soviet
experimental and theoretical research is also discussed in the paper by Kozlov
\& Ryzhov \cite{KoRy:90}.  More recent reviews of experimental research are
given by Wlezien \cite{Wlezien:94} and Saric \etal\ \cite{SaWeRa:95}.  Several
recent computational studies of natural receptivity have also been reported,
including the work of Reed and her colleagues \cite{SaReKe:94} and Corke \&
Haddad \cite{CoHa:95}.

Although most flows of practical relevance contain three-dimensional boundary
layers, the majority of receptivity and transition research, including most of
the work cited above, has been conducted for two-dimensional boundary layers.
This restriction was made necessary by the complex physical phenomenon which
occur during transition, even for two-dimensional boundary layers.
Fortunately, many of the observations made for two-dimensional boundary layers
also apply for three-dimensional boundary layers, including the importance of
meanflow distortion on receptivity.  As alluded to in \S\ref{s:motive}, the
renewed interest in laminar-flow technology has prompted considerable research
in the areas of receptivity, stability, and transition of three-dimensional
boundary layers.  Saric \& Reed \cite{SaRe:89} provide a review of the
research in stability and transition for three-dimensional boundary-layers up
to 1989.  Since then, considerable advances have been made both in the
receptivity of three-dimensional boundary layers and in the nonlinear
evolution of crossflow vortices.  In the remainder of this section, the recent
advances in the study of receptivity and transition for three-dimensional
boundary layers are discussed.

%
% Give a sense of size and Reynolds number for Saric's experiments,
% Indicate, later on, that your calculations are at about the same \Re
%
The early experiments of Saric and coworkers \cite{RaReSa:93} examined the
effects of micron-sized roughness elements on the transition location when the
roughness is placed near the leading-edge of a $45^\circ$ swept NLF(2)-0415
airfoil with a 1~m chord.  Their experimental results show that the transition
location is very sensitive to roughness near the leading-edge and that the
crossflow vortex pattern formed downstream of the roughness is directly
related to the applied surface roughness distribution.  By placing artificial
($6~\mu m$) roughness at different chordwise locations, it was demonstrated
that the transition location moved furthest upstream for roughness placed near
the point of first instability of the stationary crossflow vortices.
Variations in roughness height were also considered with the transition
location further upstream for larger roughness sizes.  Although these results
showed a strong influence of surface roughness on the generated crossflow
vortices, they did not provide a clear indication of the receptivity process
since measurements of the transition location are not a particularly sensitive
measure of the initial level of instability waves in the boundary
layer.\footnote{The transition location depends on the linear and nonlinear
evolution of instability waves in the boundary layer making it difficult to
infer receptivity characteristics directly from measurements of the transition
location.}  In particular, there was no observed effect on the transition
location when roughness was placed downstream of 4\% chord.  In subsequent
work \cite{RaReSa:94} measurement techniques were refined and the influence of
artificial roughness was detected from the attachment line all the way to 20\%
chord.  Their most recent experiments \cite{ReSaCaCh:96} demonstrate that the
transition process is still dominated by stationary crossflow vortices for a
surface that is polished to $0.25~\mu m$ RMS.  However, the focus of these
recent experiments has been to study the downstream evolution of crossflow
vortices in order to document and explain the failure of linear theory to
predict the growth rate.  We return to this issue of early nonlinear
development of crossflow vortices below.

Recently, Deyhle \& Bippes \cite{DeBi:96} have performed a detailed
experimental study of the receptivity characteristics of a three-dimensional
boundary layer to roughness of varying shape, size, location, and
distribution.  By considering swept flat-plate models with different surface
finishes, they find that the overall level of stationary CF vortices is
directly related to the average peak-to-peak roughness amplitude.  High levels
of initial roughness lead to high initial amplitudes of the stationary
crossflow vortices and to an earlier saturation at higher amplitudes.  In
comparison, the effect of roughness level on traveling crossflow vortices was
minor with the primary effect related to earlier nonlinear interaction (see
below) due to the enhanced stationary vortices at high roughness levels.  For
all roughness levels the crossflow vortices observed in the experiment had
wavelengths corresponding to the most amplified linear mode.

Deyhle \& Bippes \cite{DeBi:96} also considered the effect of discrete
artificial roughness elements to ascertain whether different wavelengths could
be excited.  The discrete roughness elements had a circular planform with a
height of $10~\mu m$ and were added to the polished surface ($1.8~\mu m$
peak-to-peak) of the plate at fixed chordwise locations near the leading edge.
Unfortunately, measurements where taken far downstream, well after the
vortices had saturated so that the results are extensively influenced by
nonlinearities.  Given this, however, the greatest crossflow response was
measured for roughness placed near the upstream neutral point and for
roughness with a spanwise spacing close to the most amplified wavelength
predicted by stability analysis.  These results are consistent both with the
experiments of Saric and coworkers \cite{RaReSa:93,RaReSa:94,ReSaCaCh:96} and
with theoretical receptivity predictions \cite{Crouch:93}, discussed in detail
below.  For roughness spacing of half the most unstable wavelength, enhanced
crossflow vortices were still observed at the most unstable wavelength but
with roughly half the amplitude.
%
% Could do a little Fourier analysis here to verify the response amplitudes
%
% Lele wants to be sure that the recent conference papers on 3-D 
% boundary layers stability, transition and nonparallel flow stability 
% are referenced.  These include:
%
%	Transition, turbulence and combustion:  TA357.T713
%	Numerical simulation of unsteady flows and transition to turbulence
%		TA357.5.U57
%	IUTAM Laminar-Turbulent transition
%

The recent experiments by both Saric's group \cite{ReSaCaCh:96} and Bippe and
coworkers \cite{DeBi:96} have established that surface roughness is of primary
importance in the generation of stationary CF vortices.  These experiments,
along with complementary PSE analyses \cite{MaLiCh:94,HaRe:96}, have focused
on the nonlinear evolution of CF vortices.  The results of these
investigations show that nonlinearity is primarily due to the interaction of
stationary CF vortices with the meanflow.  This interaction leads to meanflow
distortion which alters the stability characteristics of the boundary layer,
resulting in lower growth-rates and eventual saturation of the stationary
crossflow modes.
%
%Furthermore, nonlinearity is observed, in the experiments, to occur at the 
%furthest upstream stations where measuremenst are possible.
%
Deyhle \& Bippes \cite{DeBi:96} have also shown that the nonlinear interaction
of stationary and traveling modes can lead to an additional reduction in the
growth-rate of the stationary modes.  This nonlinearity is most pronounced for
wind tunnel experiments with modest to high free-stream turbulence, indicating
that the traveling modes are generated by a receptivity process which converts
the freestream turbulence into traveling CF vortices.
%
%Although this receptivity process has not, to date, been studied 
%theoretically, it is certainly a plausible receptivity path that has been 
%analyzed for two-dimensional boundary layers \cite{Saric_aps,Kendall}.  
%
Most importantly, however, is the fact that freestream turbulence levels are
generally much lower under flight conditions than in wind tunnels.  Because of
this, the stationary crossflow vortices are expected to be the dominant
instability mechanism in low disturbance environments such as a wing in
flight.

Comparisons of the experimental results \cite{ReSaCaCh:96,DeBi:96} with
nonlinear PSE \cite{HaRe:96,MaLiCh:94} have shown that the nonlinear effects,
including the saturation amplitude, are well predicted by PSE and, according
to Deyhle \& Bippes \cite{DeBi:96}, it is the saturation amplitude that
determines the transition location.  In forming a complete amplitude-based
transition prediction method, it remains to reliably predict the receptivity
process for stationary crossflow modes due to surface roughness.  It should be
pointed out that in performing the PSE calculations which have been compared
to experiments, the initial condition for the PSE marching procedure is based
empirically on the experimental results at the furthest upstream station
\cite{HaRe:96}.  These comparisons are hampered by that fact that the
disturbance environment is not completely known and that the flow evolution
may already be nonlinear at the first measurement station.  Recently there
have been attempts to include the receptivity process in the PSE analysis
\cite{HeLi:93}.  However, this approach is still in its infancy and, due to
the approximate nature involved in starting the PSE marching procedure, the
current results are only qualitative and further refinement is required to
incorporate quantitative receptivity predictions in PSE analysis.

%
% theory
%
In addition to the experimental work described above, there has also been
considerable theoretical research on the prediction of receptivity in
three-dimensional boundary layers.  A useful review of receptivity in
three-dimensional and high-speed boundary layers is given by Choudhari \&
Streett \cite{ChSt:90}.  In this article, the different receptivity mechanisms
for the generation of stationary and traveling crossflow vortices are
discussed.  In particular it is suggested that although the traveling
crossflow vortices have larger linear growth-rates then stationary modes,
local receptivity mechanisms preferentially excite stationary vortices.  This
hypothesis has been recently verified by Choudhari \cite{Choudhari:94} and
Crouch \cite{Crouch:93} who have independently shown that the initial
amplitude of traveling CF vortices due to the interaction of freestream
acoustic waves with surface roughness is much smaller then that of stationary
modes which are generated by the direct scattering of the meanflow by the
roughness.  This conclusion is directly related to the fact that the amplitude
of unsteady freestream disturbances (both sound and turbulence) is very low
under flight conditions.  In passing, it should be noted that both the
experiments of Radeztsky \etal\ \cite{RaReSa:93} and Deyhle \& Bippes
\cite{DeBi:96} show that freestream acoustic disturbances have no measurable
influence on the level of traveling crossflow modes or on the transition
location.

The analysis used by both Choudhari \cite{Choudhari:94} and Crouch
\cite{Crouch:93} is done in the same framework as the asymptotic theories of
Goldstein \cite{Goldstein:85} and Ruban \cite{Ruban:84}.  However, instead of
approximating the Navier--Stokes equations using high Reynolds number
expansions, the equations are replaced by the locally parallel equations in
the vicinity of the surface disturbance.  This approach was first used by
Zavol'skii \etal\ \cite{ZaReRy:83} to study the generation of TS waves in the
boundary layer over a wavy surface and several recent review articles
\cite{ChSt:94,Crouch:94} summarize the various applications of this technique.
The advantage of, what has been called Finite Reynolds Number Theory (FRNT),
is that solutions can be more easily obtained for a wide range of physical
problems at moderate Reynolds numbers.  Furthermore, FRNT is not limited to
the vicinity of the first neutral point as is the asymptotic theory.  As
indicated above, FRNT has been used to demonstrate that the initial amplitudes
of traveling crossflow modes due to the interaction of sound with roughness is
much smaller than that of stationary modes generated directly by surface
roughness.  Both of these studies where conducted for incompressible flow,
with the mean boundary layer flow given by the family of Falkner--Skan--Cooke
\cite{Cooke:50} swept-wedge flows.  Consistent with the experimental results
discussed above, the greatest effective receptivity is found to occur just
upstream of the first neutral point.  Although the receptivity increases
upstream of the neutral point, the actual amplitude of a generated instability
wave in the unstable region is reduced because the flow is highly damped
upstream of the neutral point.  As pointed out by Choudhari
\cite{Choudhari:94}, however, the results of the FRNT require numerical and/or
experimental verification in the vicinity of the first neutral point because
nonparallel effects may be important.  Additionally, in practical situations,
such as the leading-edge of a wing, the first neutral point for stationary
crossflow vortices will occur in a region of large surface curvature.
Recently the effects of both nonparallelism and surface curvature on the
stability characteristics of stationary crossflow vortices has been examined
by Malik and coworkers \cite{MaMa:94,MaBa:93,MaLi:93} using both PSE and
perturbation theory to account for the nonparallel effects.  These results
have generally shown that while curvature effects are stabilizing, nonparallel
effects are destabilizing for crossflow modes.  However, the impact of
non-parallelism and curvature remain open issues with regard to receptivity.

The present research aims to close the gap in transition prediction by
predicting the receptivity of crossflow vortices through numerical simulation
of the linearized Navier--Stokes equations.  In this way, the disturbance
environment can be carefully controlled and both the physical mechanisms of
the receptivity process and the success and limitations of the theoretical
approaches \cite{Choudhari:94,Crouch:93} can be established.  In particular,
both the effects of body curvature and nonparallel mean flow will be examined
in the context of a simplified model of a wing leading-edge.  Although beyond
the scope of the current study, our results can also be compared to PSE
receptivity predictions to evaluate the viability of that approach.

\section{Accomplishments and Overview \label{s:accomplishments}}

The accomplishments of this research include:

\begin{itemize}

\item A computational method for the simulation of receptivity phenomena near
the leading-edge of a swept-wing at high subsonic speeds has been developed.
To make the numerical simulation of receptivity problems more economical, only
the region in the immediate vicinity of the leading-edge is included in the
computational domain.  When computing mean-flow solutions, an outflow boundary
condition based on the parabolized Navier--Stokes equations has been
implemented which gives high-quality solutions up to the outflow boundary.
For disturbance calculations, extensive tests are reported on nonreflecting
outflow boundary conditions for problems where instability waves exit the
boundary.  Conditions based on the damping sponge \cite{IsOr:81} are found to
be adequate although more refined techniques are desirable.

\item A suite of model problems which represent the physical phenomena
expected in receptivity calculations have been solved.  These problems
include: unstable Tollmien--Schlichting waves in a two-dimensional boundary
layer, unstable crossflow vortices in a three-dimensional boundary layer, and
acoustic scattering from a circular cylinder.  Results from each of these
problems are in excellent agreement with exact solutions.

\item As further validation, a limited investigation of acoustic leading-edge
receptivity is performed for a flat-plate with a super-ellipse
leading-edge.  Results are compared to the incompressible calculation of Lin
\etal \cite{LiReSa:90} and the amplitude of the instability waves are
predicted to within 10\% of the incompressible solutions.

\item A series of two- and three-dimensional meanflow solutions are obtained
for a parabolic cylinder.  Low Mach number, two-dimensional results compare
favorably to a reference incompressible solution by Davis \cite{Davis:72}.  A
three-dimensional, high-speed boundary-layer over a swept parabolic-cylinder
is computed and thoroughly documented.

\item The development of stationary crossflow vortices in the
three-dimensional \linebreak boundary layer on the swept parabolic-cylinder is
compared to compressible, linear-stability-theory which incorporates the
effects of curvature and nonparallelism.  Curvature is included by writing the
Navier--Stokes equations in a curvilinear coordinate system.
%Unlike recent analysis on a circular cylinder, the metric terms for a general
%airfoil shape must account for the derivative in the body tangent direction 
%and these terms are included in our formulation.  
Nonparallel effects are included using a perturbation approach where the
solution is written in terms of a slowly varying chordwise coordinate.

\item Stability results indicate that surface curvature and nonparallel flow
are both important factors over the entire length of the unstable crossflow
region.  These results are in qualitative agreement with recent PSE and
stability analysis for a swept circular cylinder \cite{MaMa:94,MaBa:93}.  When
measured by the integrated disturbance kinetic energy, convex surface
curvature is found to be stabilizing while nonparallel effects are
destabilizing.  Including both effects in the stability theory leads to
growth-rate predictions which are in excellent agreement with linearized
Navier--Stokes solutions for relatively short spanwise wavelengths.  However,
it is demonstrated for a longer spanwise wavelength that the perturbation
approach is inadequate near the leading-edge.  This is unfortunate since this
longer wavelength mode has a larger $N$-factor making it considerably more
``dangerous.''  An approximate criterion for the failure of the nonparallel
perturbation approach is presented based on the normalized growth of the mean
boundary layer.

\item The impact of both curvature and nonparallel effects on the receptivity
of this flow to surface roughness are established by solutions of the
linearized Navier--Stokes equations.  When the results are compared to Finite
Reynolds Number Theory (FRNT), they indicate that curvature enhances
receptivity while nonparallel flow attenuates receptivity.  The net effect for
the conditions of this study are that the nonparallel attenuation dominates.
In a region near the leading-edge, the FRNT predictions are found to generally
over-predict the crossflow amplitude, with errors as large as $77\%$ when
curvature is included.  Although these predictions are conservative,
additional refinement of the theory to include nonparallel effects is
warranted.

\end{itemize}

We begin in Chapter \ref{c:eqn} with a summary of the governing equations for
an ideal, compressible fluid including nondimensionalization, and
linearization suitable for receptivity problems.  Chapter \ref{c:method}
details the numerical methods used to obtain high-accuracy meanflow solutions
and linearized disturbance solutions.  This chapter includes a discussion of
the potential flow solver, used to compute initial conditions, and boundary
conditions for both mean and disturbance equations.  Extensive validation of
the numerical method is performed in Chapter \ref{c:valid} using model
problems based on the physical phenomena observed during the early phases of
laminar-turbulent transition.  As further validation, a limited study of
leading-edge receptivity to sound is presented in \S\ref{c:2dle} including
comparisons to previous incompressible solutions.  The main focus of this
research is detailed in Chapter \ref{c:pcyl} where an extensive study of
instability and receptivity characteristics of the three-dimensional boundary
layer on a swept parabolic cylinder is performed.  Finally, the results of
this \thesis\ are summarized in Chapter \ref{c:conclude} with conclusions and
avenues for further investigation highlighted.

%==============================================================================
%
%  Tables and Figures
%
%==============================================================================

\begin{figure}[p]
\centering 
\figlab 0.25in 0.75in {(a)}
\epsfxsize=5.0in \epsfbox{figures/ch1/lfc.ai}
\vskip 1.0in
\figlab -0.1in 3.5in {(b)}
\epsfxsize=4.25in \epsfbox{figures/ch1/lv2.ai}
\vskip 0.25in
\caption [Laminar flow wing sections]{Laminar flow wing sections: (a) is a
full chord Laminar-Flow-Control airfoil (after \cite{HaHaBr:88}) showing the
ducting required to remove the suction air, while (b) is a
Natural-Laminar-Flow airfoil (after \cite{ReHoKoQu:86}) with the predicted
pressure distribution and transition locations on the upper and lower
surfaces. \label{f:lfc} }
\end{figure}

\begin{figure}[p]
\centering
\epsfxsize=6.0in \epsfbox{figures/ch1/transition.ai}
\vskip 0.5in
\caption {Routes to transition on a swept wing. \label{f:transition} }
\end{figure}
