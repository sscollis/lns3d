%%%%%%%%%%%%%%%%%%%%%%%%%%%%%%%%%%%%%%%%%%%%%%%%%%%%%%%%%%%%%%%%%%%%%%%%%%%%%%%
%
%  Chapter 3:  Numerical Method
%
%  S. Scott Collis
%
%  Written: 9-5-95
%
%  Revised: 7-19-96
%
%%%%%%%%%%%%%%%%%%%%%%%%%%%%%%%%%%%%%%%%%%%%%%%%%%%%%%%%%%%%%%%%%%%%%%%%%%%%%%%
\chapter [Numerical Method] {Numerical Method for \protect\\ 
Receptivity Calculations \label{c:method}}

This chapter presents the numerical methods developed to simulate receptivity
phenomena near a swept leading-edge.  We begin with a discussion of the issues
and challenges faced when undertaking receptivity calculations.

\section{Computational Issues and Challenges \label{s:issues} }

The computational simulation of receptivity requires a numerical scheme
capable of accurately and efficiently computing both the mean boundary-layer
flow about an arbitrary leading-edge and the evolution of unsteady
disturbances on that mean flow. The unsteady disturbances include inviscid
(acoustic, entropy, vorticity) waves in the far-field and viscous instability
waves in the boundary layer.  The dispersion relations for all types of waves
and the dynamics of their interaction (\ie\ receptivity) must be accurately
represented.  Although accurate and efficient numerical methods have been
developed to compute mean-flow and wave-propagation phenomena individually,
there is no method readily available for the simultaneous solution of both
types of problems in complex geometries.

For computing mean boundary-layer flows, there are a wide range of methods
available \cite{BeWa:76,Steger:77,vanLeer:77,StWa:81,Shakib:88} to name just a
few.  In general, these methods are second-order-accurate in space with
second-order-accurate, implicit time-advancement and incorporate some form of
(spatial) numerical dissipation to control spurious oscillations in the
solution.  These spurious oscillations arise for a host of reasons including,
but not limited to, insufficient resolution, approximate boundary conditions,
incomplete residual reduction when using iterative solvers, and mesh
stretching.  The type of numerical dissipation used depends on the particular
scheme but can take the form of an artificial viscosity term, upwind
differencing, or ``least-squares'' stabilization.  In certain limiting cases
each of these forms of numerical dissipation can be shown to be
equivalent. Although these flow solvers are generally found to converge
rapidly to accurate mean-flow solutions, their use for unsteady,
wave-propagation problems is relatively untested.
%
%.... TTD: check out the CAA workshop
%

Recently Fenno \etal \cite{Fenno:93} used a second-order finite-volume scheme
with artificial viscosity to study the receptivity of a two-dimensional
parabolic-cylinder to freestream acoustic waves at low Mach numbers.  They
found that the numerical dissipation of the code damped the incoming acoustic
waves by 56\% before the waves struck the leading-edge, even when using a
costly anisotropic dissipation technique.  Furthermore, instability wave
amplitudes and neutral point locations were in significant error when compared
to previous incompressible results obtained using a spectral method.  This was
also the case when resolution was increased to over 50 points per instability
wavelength---a value which resulted in prohibitively high CPU times.

Similar, although not as pessimistic, results were found when evaluating a
finite-element flow solver using the ``Galerkin Least-Squares'' method
\cite{Shakib:88} for computation of boundary layer instability waves
\cite{CoLe:96}.  When this method was properly formulated, by including the
full residual of the Navier--Stokes equations in the Least-Squares term, it was
found that 60 bilinear-elements per Tollmien--Schlichting wavelength and 250
elements normal to the wall are required to achieve sufficient accuracy in the
growth-rate.\footnote{With reference to \protect\S\ref{s:TS} we have defined a
growth-rate error of less than $0.2\%$ as sufficient for the receptivity
problems under consideration.}  For comparison, the method presented below,
achieves sufficient accuracy using 20 nodes per wavelength with 128 normal to
the wall.  To take full advantage of the finite-element method, an
unstructured grid is desirable to cluster elements in regions of high
gradients.  However, the error in the growth-rate for linear triangles is 3.5
times greater than that of bilinear elements using the same node distribution.
This loss in accuracy offsets the advantage of the unstructured grid.
Investigations were also made using bi-quadratic and bi-cubic elements and, as
expected, a significant reduction in the number elements was obtained with
increasing order for the same level of accuracy.  However, the greater expense
in terms of code complexity and computer time renders these higher-order
methods unsuitable for the present investigation.

In recent years, there has also been considerable method development for
computing wave-propagation problems.  These methods typically make use of
high-order finite-difference spatial discretization with a high-order explicit
time advancement scheme.  Examples include the sixth-order-accurate compact
scheme of Lele \cite{Lele:92} with fourth-order Runge-Kutta time advancement
used by Colonius \etal \cite{CoMoLe:95} and Mitchell \etal \cite{MiLeMo:96}
and the {\sf DRP} scheme of Tam and Webb \cite{TamWeb:93} which uses an
optimized 7-point finite-difference stencil in space with a multi-step,
explicit time-advancement scheme.  Although these schemes have excellent wave
propagation characteristics, they typically suffer from poor convergence to
steady-state since they rely on explicit time-advancement schemes with strict
stability limits.  Furthermore, converting to an implicit time-advancement
scheme for computing steady-states is complicated by the large stencils used
in computing spatial derivatives.

In an effort to develop a method suitable for both mean-flow and wave
propagation problems, we have retained a fully implicit time-advancement
scheme, but with a 5-point (up to 4th-order-accurate) finite difference
stencil in space.  With approximate factorization, this results in a series of
block penta-diagonal matrices which must be solved at each time-step or
iteration.  We have found that even with the rather large, block
penta-diagonal matrices, an efficient code results.  Furthermore, when
computing mean-flow solutions the block, penta-diagonal matrices can be
approximated by block, tri-diagonal matrices, thereby reducing expense while
retaining numerical stability.

In the following sections, details of the numerical method are presented
including the generalized coordinate system, spatial discretization, time
advancement, iterative solvers for both nonlinear and linearized equations,
and boundary treatments for mean and disturbance solutions.  But first, a
brief description of the receptivity problem and computational domain is
presented.

\section{Problem Description \label{s:problem} }

To render the discussion concrete, the numerical method is discussed in the
context of the swept leading-edge, receptivity problem; although the method is
generally applicable to a much wider set of problems.  Figure \ref{f:domain}
shows a schematic of the leading-edge geometry where the chordwise direction
is denoted by $x$, the vertical direction by $y$, and the spanwise direction
by $z$.  The arc-length along the surface of the wing is given by $s$ and
the distance normal to the wing surface is $n$.  The wing has an
angle-of-attack, $\alpha$, and sweep angle, $\theta$, and is idealized as
having infinite span with no taper.  This avoids fuselage and tip effects and
renders all $z$-derivatives zero in the mean.  The chordwise component of the
freestream velocity is denoted as $u_\infty$, the spanwise component as
$w_\infty$, and the magnitude of the freestream velocity is $U_\infty$.

In constructing the computational domain, only the region in the immediate
vicinity of the leading-edge is included.  By truncating the downstream
portion of the wing we avoid the unsteady and complicated flow features often
present near the trailing-edge: shock waves, flow separation, and the unsteady
wake.  In a subsonic flow, pressure disturbances associated with these
phenomena travel upstream and exert some influence on the flow near the
leading-edge.  To the extent that such upstream influence is important, the
current solutions are only an approximation of the flow near an actual wing
leading-edge.  For the problems presented in this \thesis, the impact of
domain truncation will be assessed by comparing solutions computed with
different domain lengths.

\section{Generalized Coordinates \label{s:gencoord} }

To represent the airfoil geometry and to cluster points in regions of high
spatial gradients (such as the boundary layer and the attachment-line), a
global, two-dimensional mapping is constructed from physical space $(x,y,z)$
to a uniform, Cartesian computational space $(\xi,\eta,z)$ with both $\xi$ and
$\eta$ in the range $[0,1]$.

Evaluation of first-derivatives in physical space in terms of derivatives in
computational space utilizes the Jacobian of the mapping
%
\begin{equation} \label{e:jacobian}
  \left\{ \matrix{\frac{\partial}{\partial x} \cr \noalign{\smallskip} 
                  \frac{\partial}{\partial y} \cr} \right\} = {\bf J}
  \left\{ \matrix{\frac{\partial}{\partial\xi} \cr \noalign{\smallskip} 
                  \frac{\partial}{\partial\eta} \cr} \right\} ,
\end{equation}
%
where the Jacobian, ${\bf J}$, is defined as
%
\begin{equation}
{\bf J} = \left [ \matrix{\xi_{,x} & \eta_{,x} \cr
                  \noalign{\smallskip}
                  \xi_{,y} & \eta_{,y} \cr } \right ]
\end{equation}
%
and $\xi_{,x}$, $\eta_{,x}$, $\xi_{,y}$, and $\eta_{,y}$ are the metrics of
the mapping transformation.  Similarly, second-derivatives in physical space
are computed using the Hessian of the mapping,
%
\begin{equation} \label{e:hessian}
  \left\{ \matrix{\frac{\partial^2}{\partial x^2} \cr 
                  \noalign{\smallskip}
                  \frac{\partial^2}{\partial x \partial y} \cr 
                  \noalign{\smallskip} 
                  \frac{\partial^2}{\partial y^2} \cr} \right\} = 
  \left [ \matrix{(\xi_{,x})^2 & \xi_{,x}\eta_{,x} & (\eta_{,x})^2 \cr
                  \noalign{\smallskip}
                  \xi_{,x}\xi_{,y} & \xi_{,x}\eta_{,y} + 
                  \xi_{,y}\eta_{,x} & \eta_{,x}\eta_{,y} \cr 
                  \noalign{\smallskip}
                  (\xi_{,y})^2 & \xi_{,y}\eta_{,y} & (\eta_{,y})^2 \cr} 
                  \right ]
  \left\{ \matrix{\frac{\partial^2}{\partial\xi^2} \cr 
                  \noalign{\smallskip} 
                  \frac{\partial^2}{\partial\xi\partial\eta} \cr
                  \noalign{\smallskip}
                  \frac{\partial^2}{\partial\eta^2} \cr} \right\} +
  \left [ \matrix{\xi_{,xx} & \eta_{,xx} \cr
                  \noalign{\smallskip}
                  \xi_{,xy} & \eta_{,xy} \cr 
                  \noalign{\smallskip}
                  \xi_{,yy} & \eta_{,yy} \cr} \right ]
  \left\{ \matrix{\frac{\partial}{\partial\xi} \cr 
                  \noalign{\smallskip} 
                  \frac{\partial}{\partial\eta} \cr} \right\} .
\end{equation}
%
The actual form of the mapping function depends on the particular problem
under investigation.

Typically, in CFD applications, the governing equations are converted to a
generalized coordinate space.  In so doing, the mapping metrics ($\xi_{,x}$,
$\eta_{,x}$, \ldots) are computed approximately using the same derivative
scheme used for the flow variables.  Errors in the mapping metrics can lead to
an increase in numerical errors.  To circumvent this, we use analytically
defined metrics whenever possible, which includes all of the problems
presented here. For geometries where analytical metrics are not possible, we
have developed a preprocessing routine which uses a high resolution mesh to
compute the mapping metrics.  Although the preprocessor computes finite
difference approximations to the metrics, the mesh spacing for the metric
computation can be selected to virtually eliminate errors in the metric
evaluation.  These `pseudo-analytical' metrics can then be used as input data
to the flow solver.

It is important to note that the numerical method presented here is based on
the primitive form of the governing equations (\ref{e:con})--(\ref{e:eng}).
As such, the numerical method is not conservative and therefore is not
appropriate for computations with shocks where shock-capturing is used.
However, since the equations are in primitive form, we are guaranteed that the
uniform freestream condition will be preserved regardless of the mesh metrics.
Overall conservation with this type of approach is only approximate but the
errors are small---on the order of the truncation error.

\section{Spatial Discretization \label{s:spacedisc}}

Consider derivative operators in the uniform computational space $(\xi,\eta)$
where $\xi_i,\eta_j$ represent the nodal locations with $\xi_i = \Delta \xi
(i-1)$ for $1 \leq i \leq N_\xi$ and $\eta_j = \Delta \eta (j-1)$ for $1 \leq
j \leq N_\eta$.  In presenting the spatial discretization, the difference
operators are given only for the $\xi$ direction since the $\eta$ operators
are completely analogous.

At interior nodes, the fourth-order-accurate central difference scheme is used
for first- and second-derivatives.  First-derivatives are approximated by the
expression
%
\begin{equation} \label{e:1st}
 \left( \frac{\partial f}{\partial \xi} \right)_i \approx 
 \frac{1}{12\Delta\xi} \Big[   \left( f_{i-2} - f_{i+2} \right) -
                             8 \left( f_{i-1} - f_{i+1} \right) \Big] ,
\end{equation}
%
while second-derivatives are computed using
%
\begin{equation} \label{e:2nd}
 \left( \frac{\partial^2 f}{\partial \xi^2} \right)_i \approx 
 \frac{1}{12 (\Delta\xi)^2} 
 \Big[  - \left( f_{i-2} - 2 f_{i} + f_{i+2} \right) +
       16 \left( f_{i-1} - 2 f_{i} + f_{i+1} \right) \Big] .
\end{equation}

Near the computational boundaries, finite difference operators that are biased
toward the interior are required.  Since the interior scheme uses a 5-point
stencil, biased difference operators are required at the boundary point and at
the first node in the interior, \ie\ nodes $(1,2,N_\xi-1,N_\xi)$.

First derivatives at $\xi_1$ are computed using the 4th-order-accurate
one-sided difference scheme
%
\begin{equation}
 \left( \frac{\partial f}{\partial \xi} \right)_1 \approx 
 \frac{1}{12\Delta\xi} \Big[ -25 f_{1} + 48 f_{2} - 
                              36 f_{3} + 16 f_{4} -
                               3 f_{5} \Big] ,
\end{equation}
%
while at node $\xi_2$ the 4th-order-accurate biased difference stencil is used
%
\begin{equation}
 \left( \frac{\partial f}{\partial \xi} \right)_2 \approx 
 \frac{1}{12\Delta\xi} \Big[ -3 f_{1} - 10 f_{2} + 
                             18 f_{3} -  6 f_{4} + f_{5} \Big] .
\end{equation}
%
Second derivatives at $\xi_1$ are computed using the 3rd-order-accurate
expression
%
\begin{equation}
 \left( \frac{\partial^2 f}{\partial \xi^2} \right)_1 \approx 
 \frac{1}{12(\Delta\xi)^2} \Big[ 11 f_{1} - 20 f_{2} + 
                                  6 f_{3} +  4 f_{4} - f_{5} \Big] ,
\end{equation}
%
while a 3rd-order-accurate one-side expression is used at node $\xi_2$
%
\begin{equation}
 \left( \frac{\partial^2 f}{\partial \xi^2} \right)_2 \approx 
 \frac{1}{12(\Delta\xi)^2} \Big[  35 f_{1} - 104 f_{2} +
                                 114 f_{3} -  56 f_{4} + 11 f_{5} \Big] .
\end{equation}
%
Similar expressions hold for the derivatives at nodes $N_\xi-1$ and $N_\xi$
but with the stencils reversed and the signs switched on the coefficients for
the first derivatives.

Notice that we have retained a 5-point stencil for all derivatives, even at
the boundaries.  This ensures that first-derivatives are computed to
4th-order-accuracy at every point in the domain, while second derivatives drop
only one-order near the boundary.  Although retaining 4th-order-accurate first
derivatives near the boundary is not required to obtain global 4th-order
convergence \cite{Gustafsson:75}, we have found, through numerical
experimentation, that the added accuracy is desirable especially near the
wall.  The use of 5-point stencils near the boundary does alter the structure
of the matrices used in the implicit solver and the impact of these changes is
discussed in sections \ref{s:base} and \ref{s:dist}.

\section{Temporal Discretization \label{s:temporal} }

For computing steady-state solutions (both nonlinear and linearized) we march
the unsteady equations to the steady-state using the implicit Euler scheme,
where the time derivative is approximated by
%
\begin{equation} \label{e:ieuler}
\left(\frac{\partial u}{\partial t}\right)^{(n+1)} =  
\frac{u^{(n+1)}-u^{(n)}}{\Delta t} + {\cal O}(\Delta t) \period
\end{equation}
%
This first-order accurate scheme has significant temporal dissipation which
helps speed the convergence to the steady-state solution.  Since time-accuracy
is not an issue for steady-state calculations, local-time-stepping is used to
further accelerate convergence.  With local-time-stepping, $\Delta t$ in
equation (\ref{e:ieuler}) is replaced by $\Delta t(x,y)$ which is a function
of the $(x,y)$ coordinates.  The value of the time-step is chosen to be
proportional to a measure of the local Courant--Friedrichs--Lewy ($\CFL$)
condition.  The techniques for estimating the $\CFL$ for base flow and
linearized calculations are presented in \S\ref{s:base} and \S\ref{s:dist}
respectively.  It is important to remember that the $\CFL$ is only used to
compute the local time-step as a means of enhancing convergence to the
steady-state solution.  In this context, the $\CFL$ has no relevance to
numerical stability since the time-advancement scheme is unconditionally
stable.

For unsteady calculations, we use the second-order accurate, two-step implicit
scheme given by
%
\begin{equation}
  \left(\frac{\partial u}{\partial t}\right)^{(n+1)} = 
  \frac{3u^{(n+1)}-4u^{(n)}+u^{(n-1)}}{2\Delta t} + {\cal O}(\Delta t^2)
\end{equation}
%
Since this time-advancement scheme requires two previous time-levels, the
integration is started at $t=0$ with one step of implicit Euler, equation
(\ref{e:ieuler}).  The two-step implicit scheme has been recently used to
study transitional flows by Rai \& Moin \cite{RaMo:93}.  Unlike the
Crank--Nicholson scheme which is the most accurate second-order accurate
implicit scheme, the two-step implicit scheme does introduce temporal
dissipation due to the biased stencil.  When coupled with the
central-difference operators used in space, we have found that the
Crank--Nicholson scheme eventually leads to numerical oscillations in the
solution which arise due to the variety of factors discussed in
\S\ref{s:issues}.  Using the two-step implicit scheme prevents the
accumulation of these numerical oscillations at the expense of slightly
reduced accuracy for a given time-step.

\section{Base Flow Solver \label{s:base}}

The base flow is given by solution of the steady, nonlinear Navier--Stokes
equations (\ref{e:bcon})--(\ref{e:bideal}) subject to appropriate boundary
conditions.  Since the wing is idealized as having infinite span the problem
is reduced to a two-dimensional problem in the $(x,y)$ plane but with three
nonzero components of velocity due to wing sweep.

To solve (\ref{e:bcon})--(\ref{e:bideal}) we retain the time terms and march
the unsteady NS equations to the steady-state using the implicit-Euler
time-integrator, equation (\ref{e:ieuler}).  With the time terms added,
equations (\ref{e:bcon})--(\ref{e:bideal}) can be written in the form
%
\begin{equation} \label{e:base}
  \bUm_{,t} + \bF(\bUm) = 0 ,
\end{equation}
%
where 
%
\begin{equation}
  {\bUm} = \left\{\matrix{\rhom\cr \um_1\cr \um_2\cr \um_3\cr \Tm\cr}\right\} .
\end{equation}
%
Applying backward Euler time discretization yields
%
\begin{equation}
  -\bUm^{(n+1)} + \bUm^{(n)} + \Delta t \bUm^{(n+1)}_{,t} \equiv 
  \bG\left(\bUm^{(n+1)};\bUm^{(n)}\right) = 0
\end{equation}
%
and applying Newton's method to solve this nonlinear problem leads to the
iteration
%
\begin{eqnarray}
  \bUm^{(n+1)}_1  &=& \bUm^{(n)} \\
  \delta \bUm_{i} &=& \bUm^{(n+1)}_{i+1} - \bUm^{(n+1)}_{i} \label{e:var} \\
  -\bigg(\frac{\partial \bG}{\partial \bUm}\bigg)_{i} \delta \bUm_{i} &=& 
  \bG(\bUm^{(n+1)}_{i};\bUm^{(n)}) \label{e:newt} ,
\end{eqnarray}
%
where (\ref{e:newt}) can be simplified to
%
\begin{equation} \label{e:newton}
  \bigg[ \bI + \Delta t \frac{\partial \bF}{\partial \bUm} \bigg]_{i} 
  \delta \bUm_{i} = \bG(\bUm^{(n+1)}_{i};\bUm^{(n)}) .
\end{equation}
%
The process of linearization in the Newton iteration is nearly the same as
that done in \S\ref{c:eqn} when deriving the disturbance equations.  The only
difference is that the linearization is now done about an unsteady field thus
requiring a slight modification to the definitions of the matrices in equation
(\ref{e:pNS}).  The required modifications are detailed in Appendix
\ref{a:nlinmat}.  With the understanding that the matrices are modified to
account for the unsteady flow, the tangent in the Newton iteration can be
obtained from (\ref{e:pNS}) by evaluating the matrices at the previous
iteration.

Converting to the generalized coordinate system the tangent becomes
%
\begin{equation} \label{e:tangent}
  -\bigg(\frac{\partial \bG}{\partial \bUm}\bigg)_{i} =
   \bigg[ \bI + \Delta t \Big (\check\bA \Delta_{\xi} + 
                               \check\bB \Delta_{\eta} + 
                               \check\bD -
                               \check{\bf V}_{\xi\xi}   \Delta_{\xi\xi}  - 
                               \check{\bf V}_{\xi\eta}  \Delta_{\xi\eta} -
                               \check{\bf V}_{\eta\eta} \Delta_{\eta\eta} \Big)
  \bigg]_{i} .
\end{equation}
%
In this equation the matrices are denoted with a check to indicate that the
mapping metrics are included in their definitions, \eg,\ 
%
\begin{equation} \label{e:ss}
\check\bA = \xi_{,x} \bA + \xi_{,y} \bB + \xi_{,xx} {\bf V}_{11} +
            \xi_{,xy} {\bf V}_{12} + \xi_{,yy} {\bf V}_{22} .
\end{equation}
%
The symbols ($\Delta_{\xi}$, $\Delta_{\eta}$, \ldots) in equation
(\ref{e:tangent}) represent finite difference operators in computational
space.  In computing the mean flow, it is computationally attractive to
approximate the tangent using second-order finite difference operators instead
of the full fourth-order operators used to form the residual.  This
approximation decreases the bandwidth of the tangent, reducing memory storage
and floating point operations.  However, from computational experience we have
found that convergence is degraded if the boundary differences are
approximated in the tangent.  So, in practice, second-order differences are
used in the tangent only for interior nodes and 5-point stencils are retained
at the three nodes nearest the boundaries.  It should be noted that the
approximation of the LHS finite difference operators with reduced-order
finite-differences has been used previously by Rai \& Moin \cite{RaMo:93} and
Beaudan \& Moin \cite{BeMo:94}.  However, in both these applications, the
interior scheme used upwind biased differencing and the LHS was approximated
by first-order upwind differencing with no report of poor convergence near the
boundaries.  However, it is well known that the use of biased differencing
introduces numerical dissipation.  The lack of numerical dissipation in the
present scheme may contribute to the sensitivity of convergence on the
boundary approximation in the LHS.

As a further approximation, (\ref{e:newton}) is approximately factored into
%
\begin{eqnarray}
  \bigg[ \bI + \Delta t \Big ( \check\bA \Delta_{\xi} + \check\bD - 
                               \check{\bf V}_{\xi\xi} \Delta_{\xi\xi} 
  \Big) \bigg]_{i} \bZ_{i} &=& \bG(\bUm^{n+1}_{i};\bUm^{n}) , \\
  \bigg[ {\bf I} + \Delta t \Big ( \check\bB \Delta_{\eta} - 
                                   \check{\bf V}_{\eta\eta} \Delta_{\eta\eta}
  \Big) \bigg]_{i} \Delta \bUm_{i} &=& \bZ_{i} , 
\end{eqnarray}
%
which greatly reduces the amount of memory required at the expense of
introducing errors of ${\cal O}(\Delta t^2)$ in the tangent.  The use of the
higher-order boundary closure in the tangent spoils the block tri-diagonal
structure of the approximately factored system near the boundaries resulting
in a matrix with a profile as shown in figure~\ref{f:tri}.  A special purpose
routine based on Gaussian elimination \cite{GoLo:89} has been written to
efficiently solve this, nearly, block tri-diagonal system on a vector
supercomputer.

By introducing the approximations described above into the tangent, the
quadratic convergence of the Newton iteration is sacrificed.  Assuming that
the iteration converges, the errors due to these approximations can be driven
to zero.  By analysis of the one-dimensional convection-diffusion equation and
by computational experimentation of the full equations we have found that
these approximations yield a stable and convergent scheme.

As discussed in \S\ref{s:temporal}, local time-stepping, based on the local
$\CFL$, is used to speed convergence to the steady-state solution.  To
estimate the local $\CFL$ we use the following approximate expression
\cite{Pulliam:94} which follows from a physical argument of enclosing the
physical domain of dependence within a computational cell \cite{MacCormack:88}
%
\begin{eqnarray} \label{e:cfl}
\CFL &=& \frac{\Delta t}{\Delta\xi \Delta\eta}
         \biggl\{ |{\bf J}_{1i} \bar u_i| \Delta\eta + 
                  |{\bf J}_{2i} \bar u_i| \Delta\xi +  \nonumber \\
  & &   \bar c \left[ ( {\bf J}_{11}^2 + {\bf J}_{12}^2 ) \Delta\eta^2 + 
                 ( {\bf J}_{21}^2 + {\bf J}_{22}^2 ) \Delta\xi^2  
          \right]^{1/2} \biggr\} \period
\end{eqnarray}
%
In this expression, ${\bf J}$ is the Jacobian of the mapping and $\bar c$ is
the local sound-speed.  By specifying a value of $\CFL$, equation
(\ref{e:cfl}) is solved for the local value of $\Delta t(x,y)$.  The $\CFL$
defined in equation (\ref{e:cfl}) includes convective and acoustic phenomena
but ignores viscous diffusion.  A similar expression can be derived to include
viscous effects, but we have found the current equation (\ref{e:cfl}) adequate
for accelerating convergence to steady-state.

\section{Linear Disturbance Solver \label{s:dist} }

Two approaches are used to obtain solutions to the linearized disturbance
equations (\ref{e:pNS}).  In the first method, the unsteady equations are
advanced in time using the second-order accurate, implicit, time-advancement
scheme presented in \S\ref{s:temporal}.  This technique is suitable for the
solution of problems in which the transient temporal evolution of disturbances
is desired.  In the second method, the equations are converted to the
frequency domain and are marched to a steady-state solution using a technique
similar to that used for the base-flow solution.  This method is ideal when a
time asymptotic solution is desired.

\subsection{Unsteady Approach \label{ss:unsteady} }

In the unsteady approach, the disturbance equations are solved using a
time-accurate technique to determine the transient evolution of disturbances.
Given a mean flow solution, the solution to the disturbance equations follows
using a combination of high-order finite difference and spectral schemes to
accurately compute spatial derivatives.  In the $(x,y)$ plane, derivatives are
computed in the same manner as for the base flow.  Since the spanwise
direction, $z$, is periodic, we write the solution as
%
\begin{equation}
  \bU' = \bhU(x,y,t) \: e^{i k_z z }
\end{equation}
%
where $k_z = 2\pi/\lambda_z$ is the spanwise wavenumber and the real-part
convention is used.  Substituting this expression into the linearized
Navier--Stokes equation (\ref{e:pNS}) one obtains
%
\begin{eqnarray} \label{e:fNS}
  \bhU_{,t} + \bA \bhU_{,1} + \bB \bhU_{,2} + i k_z \bC \bhU + \bD \bhU =
  \nonumber \\
  \bVxx \bhU_{,11} + \bVxy \bhU_{,12} +
  i k_z \bVxz \bhU_{,1} + \bVyy \bhU_{,22} +
  i k_z \bVyz \bhU_{,2} - k_z^2 \bVzz \bhU .
\end{eqnarray}
%
Converting to computational space produces the equation
%
\begin{eqnarray}
  \bhU_{,t} + \bigg[ 
              \check\bA \Delta_{\xi} + 
              \check\bB \Delta_{\eta} + 
              \check\bD -
              \check{\bf V}_{\xi\xi}   \Delta_{\xi\xi}  - 
              \check{\bf V}_{\xi\eta}  \Delta_{\xi\eta} -
              \check{\bf V}_{\eta\eta} \Delta_{\eta\eta} 
              \bigg] \bhU = 0
\end{eqnarray}
%
where the check on the matrices indicates that they include the mapping
metrics and contributions from the spanwise direction.  In this equation, the
derivative operators in computational space are given by the fourth-order
interior scheme with fourth-order boundary closure for the first-derivatives
and third-order boundary closure for second derivatives (see
\S\ref{s:spacedisc}).  It is useful to rewrite this equation using operator
notation as
%
\begin{equation}
  \bhU_{,t} + \bL(\bhU) = 0 .
\end{equation}
%
The time derivative is discretized with the second-order, two-step implicit
scheme, giving
%
\begin{eqnarray}
  - \bhU^{n+1} + \frac{4}{3} \bhU^{n} - \frac{1}{3} \bhU^{n-1} + 
    \frac{2\Delta t}{3} \bhU^{n+1}_{,t} \equiv
    \bG(\bhU^{n+1};\bhU^{n},\bhU^{n-1}) = 0 . \label{e:2step}
\end{eqnarray}
%
Although (\ref{e:2step}) is a linear equation, we formally apply Newton's
method to solve it, leading to the iteration
%
\begin{eqnarray}
  \bhU^{n+1}_1 &=& \bhU^{n} \\
  \delta \bhU_{i} &=& \bhU^{n+1}_{i+1} - \bhU^{n+1}_{i} \\
  \bigg[ \bI + \frac{2\Delta t}{3} \frac{\partial \bL}{\partial \bhU} 
  \bigg]_{i} \delta \bhU_{i} &=& \bG(\bhU^{n+1}_{i};\bhU^{n},\bhU^{n-1})  .
  \label{e:pnewton}
\end{eqnarray}
%
If (\ref{e:pnewton}) is solved exactly then only one iteration is required.
However, for computational efficiency the disturbance equations are
approximately factored to obtain
%
\begin{eqnarray}
  \bigg[ \bI + \frac{2\Delta t}{3} \Big ( \check\bA \Delta_{\xi} + 
                               \check\bD - 
                               \check{\bf V}_{\xi\xi} \Delta_{\xi\xi} 
  \Big) \bigg]_{i} \bZ_{i} = 
  \bG(\bhU^{n+1}_{i};\bhU^{n},\bhU^{n-1}) \label{e:itera} \\
  \bigg[ {\bf I} + \frac{2\Delta t}{3} \Big ( \check\bB \Delta_{\eta} - 
                               \check{\bf V}_{\eta\eta} \Delta_{\eta\eta}
  \Big) \bigg]_{i} \delta \bhU_{i} = \bZ_{i} . \label{e:iterb}
\end{eqnarray}
%
The approximate factorization introduces an ${\cal O}(\Delta t^2)$ error in
the tangent which is consistent with the second-order-accurate time
advancement scheme. By taking additional iterations, the factorization error
can be eliminated thus recovering the solution to equation (\ref{e:2step}).

In contrast to the base-flow solver, the fourth-order spatial differences for
the interior nodes are retained in the tangent.  This ensures second-order
temporal accuracy in one iteration per time-step while also accelerating
convergence when multiple iterations are used.  With the exception of the
boundaries, the approximate factorization of this system of equations results
in a series of block, penta-diagonal linear systems.  Near the boundaries the
penta-diagonal structure is spoiled due to the high-order boundary closure
(see figure~\ref{f:penta}) and a special solver has been developed to
efficiently obtain solutions to this system on vector supercomputers.  Since
the matrices depend only on the mean-flow, a single LU factorization is
performed for the first time-step and subsequent time-steps only require
forward and backward substitutions.  This renders the disturbance code roughly
six times faster then the nonlinear base-flow solver.

\subsection{Frequency-Domain Approach \label{ss:frequency} }

The frequency-domain approach is used when only the time-asymptotic behavior
of the solution is required.  This includes cases when the disturbance
solution is steady or when the solution is periodic with a single frequency,
$\omega$.  In this case, the solution is assumed to be of the form
%
\begin{equation}
  \bU' = \bhU(x,y) \: e^{i (k_z z - \omega t)} .
\end{equation}
%
Substituting this expression into equation (\ref{e:pNS}) yields
%
\def\bhU{{\hat{\bf U}}}
\begin{eqnarray} \label{e:omegaNS}
  -i \omega \bhU + \bA \bhU_{,1} + \bB \bhU_{,2} + i k_z \bC \bhU + \bD \bhU =
  \nonumber \\
  \bVxx \bhU_{,11} + \bVxy \bhU_{,12} +
  i k_z \bVxz \bhU_{,1} + \bVyy \bhU_{,22} +
  i k_z \bVyz \bhU_{,2} - k_z^2 \bVzz \bhU .
\end{eqnarray}
%
When converted to computational space, this equation takes the form
%
\begin{eqnarray}
  \bigg[ 
  \check\bA \Delta_{\xi} + 
  \check\bB \Delta_{\eta} + 
  \check\bD -
  \check{\bf V}_{\xi\xi}   \Delta_{\xi\xi}  - 
  \check{\bf V}_{\xi\eta}  \Delta_{\xi\eta} -
  \check{\bf V}_{\eta\eta} \Delta_{\eta\eta} 
  \bigg] \bhU = 0
\end{eqnarray}
%
where the check on the matrices indicates that they include the mapping
metrics and contributions from the spanwise direction and time-derivative.  As
in the unsteady approach, the derivative operators in computational space are
given by the fourth-order interior scheme with fourth-order boundary closure
for the first-derivatives and third-order boundary closure for second
derivatives.  Rewriting this equation using operator notation and introducing
a pseudo-time, $\tau$-derivative leads to
%
\begin{equation}
  \frac{\partial\bhU}{\partial\tau} + \bL_\omega (\bhU) = 0 .
\end{equation}
%
The solution to this equation is obtained using the same technique described
for the unsteady approach (see \S\ref{ss:unsteady}) but with the second-order
time advancement replaced with implicit Euler.  

To speed convergence to the steady-state solution, local-time-stepping is used
as discussed in \S\ref{s:temporal}.  In this case, we approximate the local
$\CFL$ by the following expression
%
\begin{eqnarray} \label{e:cfllinear}
  \CFL &=& \frac{\Delta\tau}{\Delta\xi \Delta\eta}
           \biggl\{ |{\bf J}_{1i} \bar u_i| \Delta\eta + 
                  |{\bf J}_{2i} \bar u_i| \Delta\xi +  
                  \frac{\bar w}{\lambda_z} \Delta\xi \Delta\eta \nonumber \\
       & & \bar c \left[ ( {\bf J}_{11}^2 + {\bf J}_{12}^2 ) \Delta\eta^2 + 
                      ( {\bf J}_{21}^2 + {\bf J}_{22}^2 ) \Delta\xi^2  +
                      \frac{\Delta\xi^2\Delta\eta^2}{\lambda_z^2}
                  \right]^{1/2} \biggr\} \comma
\end{eqnarray}
%
where the velocities and sound-speed are based on mean-flow quantities and
$\lambda_z = 2\pi/k_z$ is the wavelength of disturbances in the $z$-direction.
Letting $\lambda_z \rightarrow \infty$ yields the form appropriate for
two-dimensional linear disturbances.  By specifying a value of $\CFL$,
equation (\ref{e:cfllinear}) is solved for the local value of
$\Delta\tau(x,y)$.  Although this estimate of $\CFL$ does not include viscous
effects, we have found it to be adequate for accelerating convergence to
steady-state.

\section{Potential Flow Solutions \label{s:initial} }

The initial conditions for the base-flow calculations can significantly
influence the convergence to the steady state.  For the base-flow computations
presented here, a potential flow solution is used for the initial condition.
Since we are considering only infinite-span wings, the spanwise component of
velocity, $w$, is constant.  Therefore the steady, three-dimensional potential
equation in nonconservative form reduces to the two-dimensional equation
%
\begin{equation} \label{e:potential}
(1-\M_x^2) \phi_{,xx} + (1-\M_y^2) \phi_{,yy} - 2\M_x\M_y \phi_{,xy} = 0
\end{equation}
%
where $\phi$ is the velocity potential function defined as
%
\begin{equation}
u = \phi_{,x} \qquad v = \phi_{,y} ,
\end{equation}
%
with $\M_x = u / c$, and $\M_y = v / c$.  In these expressions, $c = \sqrt{T}
/ \M$ with $T$ computed from the isentropic relation
%
\begin{equation}
  T = 1 + \frac{(\gamma-1)\M^2}{2} \left( 1 - u^2 - v^2 \right)
\end{equation}
%
and with the reference Mach number given by $\M = u_\infty / c_\infty$ which
corresponds to the chordwise Mach number.  The freestream Mach number is
related to the chordwise Mach number by $\M_\infty = \M \sqrt{1 +
\tan^2(\theta)}$ where $\theta$ is the sweep angle.  In these expressions, the
reference velocity is $u^*_r = u^*_\infty$ such that the nondimensional
spanwise component of velocity in the freestream is given by $w_\infty =
\tan(\theta)$.  The advantage of this approach is that an unswept potential
solution at $\M_\infty = \M$ is identical to a swept solution at $\M_\infty =
\M \sqrt{1 + \tan^2(\theta)}$ except with $w = \tan(\theta)$.

The method used to solve equation (\ref{e:potential}) is similar to that used
for the Navier--Stokes equations above.  Fourth-order accurate finite
differences (see \S\ref{s:spacedisc}) are used to approximate the spatial
derivatives and the solution is marched to the steady-state using implicit
Euler time-advancement.  At each time-step the system of equations is
approximately factored resulting in a series of (nearly) block, penta-diagonal
systems.

On the wall and symmetry boundaries the normal derivative of the velocity
potential function is enforced to be zero,
%
\begin{equation}
  \frac{\partial \phi}{\partial n} = 0
\end{equation}
%
corresponding to the no-penetration condition, $\vn = 0$, where $n$ is the
coordinate normal to the boundary under consideration.  On the inflow and
outflow boundaries, the potential function is set to the value corresponding
to the freestream solution.  Specific details on boundary conditions, mesh
generation, and solution accuracy are given in \S\ref{s:pcylpot} for the
parabolic-cylinder geometry.  In addition to using the potential solution as
an initial condition, we also use it to specify inflow and outflow boundary
conditions for the base-flow NS calculation.  Details are provided in
section~\ref{ss:meanbc}.

\section{Boundary Treatments}

In this section, the various boundary treatments used to obtain mean and
disturbance solutions are discussed.  Before doing so, however, the general
framework used to implement boundary constraints in the implicit solver is
presented.  For this purpose, we adopt an approach which is based on a direct
application of Newton's method to linearize the boundary constraints in terms
of the primitive variables which results in linearized expressions for
the boundary conditions that are compatible with Newton's method used to
solve the Navier--Stokes equations [see equation (\ref{e:newton}) and
(\ref{e:pnewton})].  The advantage of this technique is that it provides a
consistent framework for deriving the appropriate implicit boundary constraints
given any desired boundary condition.  Of course, the resulting constraints
will be identical to those obtained using the so-called ``delta-form''
\cite{Hirsch:90}, however we feel that the present approach is easier to
apply and more intuitive, especially for complicated boundary conditions.

\subsection{Implicit Boundary Constraints}

To preserve convergence of the implicit solver, linearized expressions for all
boundary constraints are required in terms of $\delta\bU$.  Consider an
arbitrary constraint, 
%
\begin{equation}
  g_k(\bU^{(n+1)}) = 0 \comma
\end{equation}
%
which is some function (possibly nonlinear) of the primitive variables at
time-step, $n+1$.  In this expression, $k$ denotes the particular primitive
variable under constraint and can take the values: $\rho$, $u$, $v$, $w$, $T$.
Applying Newton's method to the constraint equation yields
%
\begin{equation} \label{e:bnewt}
  \bigg(\frac{\partial g_k}{\partial \bU}\bigg)_{i} \delta \bU_{i} = 
  -g_k\big(\bU^{(n+1)}_{i}\big)
\end{equation}
%
where $\delta \bU_{i} = \bU^{(n+1)}_{i+1} - \bU^{(n+1)}_{i}$.  When applying a
boundary constraint, equation (\ref{e:bnewt}) is used in place of the
discrete, linearized equations-of-motion for the constrained quantity.  In
presenting the boundary conditions, we define the constraints, $g_k$, and the
variation of the constraints,
%
\begin{equation}
  \delta g_k \equiv \bigg(\frac{\partial g_k}{\partial \bU}\bigg)_{i} 
                    \delta \bU_{i}
\end{equation}
%
in terms of the primitive variables.

For example, consider the application of the no-slip condition on the first
velocity component, $u = u_w$, where, for generality, the wall has velocity
$u_w$.  For this case
%
\begin{equation}
  g_u(\bU) = u - u_w = 0
\end{equation}
%
and
%
\begin{equation}
  \delta g_u = \delta u \period
\end{equation}
%
Thus, equation (\ref{e:bnewt}) simply becomes
%
\begin{equation}
  \delta g_u = -g_u \comma
\end{equation}
%
which in terms of the primitive variables is just
%
\begin{equation}
  \delta u = -(u - u_w) \period
\end{equation}

\subsection{Base Flow Boundary Treatments \label{ss:meanbc}}

The base-flow solutions are obtained using no-slip boundary conditions on the
wall with either isothermal or adiabatic temperature boundary conditions,
symmetry conditions upstream of the leading edge, Riemann extrapolation on the
inflow boundary, and a parabolic approximation on the outflow boundary.  In
the following, the detailed implementation of each of these boundary
treatments is presented.

\subsubsection{Wall boundary conditions \label{sss:wallbc}}

On the wall, $(n=\eta=0)$, the no-slip boundary conditions are implemented by
replacing the momentum equations with constraints on the velocities: $\um_i =
0$. Likewise, the energy equation is replaced by either the isothermal
condition, $\Tm = \Tm_w$, or the adiabatic condition ${\partial \Tm}/{\partial
n} = 0$.  For the body orthogonal meshes used here, the adiabatic condition
can be written in computational space as ${\partial \Tm}/{\partial
\eta}(\eta=0) = 0$ and the derivative is approximated by a fourth-order
accurate one-sided difference (see \S\ref{s:spacedisc}).  In all cases the
density at the wall is determined using the continuity equation.

For the implicit solver, the boundary constraints at the wall, $(\eta=0)$, can
be written as:
%
\begin{equation}
  g_{u_i} = \um_{i} = 0 \comma
\end{equation}
%
\begin{equation}
  g_T = \Tm - \Tm_w = 0 \comma \quad \mbox{or} \quad g_T = \Delta_\eta \Tm = 0
\end{equation}
%
and the variation of the constraints with the primitive variables are
%
\begin{equation}
  \delta g_{u_i} = \delta \um_i \comma
\end{equation}
%
\begin{equation}
  \delta g_T = \delta \Tm \comma \quad \mbox{or} \quad 
               \delta g_T = \Delta_\eta (\delta \Tm) \period
\end{equation}

\subsubsection{Symmetry boundary}

Near the symmetry boundary, $(\xi=0$), the finite difference stencil used in
the interior is adjusted to account for the symmetry (or antisymmetry for the
boundary normal velocity) of the variables at the boundary.  Since a
five-point stencil is used, modification is required for the three rows of
nodes adjacent to the boundary when computing both the RHS and LHS.

\subsubsection{Inflow boundary \label{s:inflow} }

On the inflow boundary, $(\eta=1)$, the boundary conditions are based on the
locally one-dimen\-sional Riemann invariants.  On a subsonic inflow, four
quantities must be specified.  Here we constrain the entropy, spanwise and
tangential velocities, and the incoming Riemann invariant and each of these
quantities are determined from the potential flow solution (see
\S\ref{s:initial}).

The locally one-dimensional Riemann invariants are defined as
%
\begin{equation}
  R_1 = \vn - \frac{2 c}{\gamma - 1} , \quad R_2 = \vn + \frac{2 c}{\gamma-1}
\end{equation}
%
where $c$ is the local sound-speed and $\vn$ is the velocity normal to the
boundary.  Then, on the inflow boundary
%
\begin{equation} \label{e:Riemann}
  \frac{p}{\rho^\gamma} = \left(\frac{p}{\rho^\gamma}\right)_\infty \equiv 
                          \frac{1}{\gamma\M_\infty^2} \comma \quad
  w = w_p \comma \quad
  V_t = {V_t}_p \comma \quad
  R_1 = {R_1}_p \comma \quad
  R_2 = {R_2}_{int}
\end{equation}
%
where $V_t$ is the velocity tangent to the boundary, the subscript ``$p$''
refers to the potential flow solution, and the subscript ``$int$'' refers to a
quantity extrapolated from the interior.

The conditions (\ref{e:Riemann}) are used to form the following constraints on
the primitive variables at the boundary:
%
\begin{equation}
  g_\rho = \rho - \left( \frac{(\gamma-1)\M_\infty}{4}
        \left[ {R_2}_{int} - {R_1}_p \right] \right)^\frac{2}{\gamma-1} 
	= 0\comma
\end{equation}
%
\begin{equation}
  g_u = u - n_1 \left(\frac{{R_1}_{p} + {R_2}_{int}}{2}\right) + 
            n_2 {V_t}_p = 0 \comma
\end{equation}
%
\begin{equation}
  g_v = v - n_2 \left(\frac{{R_1}_{p} + {R_2}_{int}}{2}\right) - 
            n_1 {V_t}_p = 0 \comma
\end{equation}
%
\begin{equation}
  g_w = w - w_p = 0 \comma
\end{equation}
%
\begin{equation}
  g_T = T - \left(\frac{(\gamma-1)\M_\infty}{4}
            \left[ {R_2}_{int} - {R_1}_p \right ] \right)^2 = 0 \comma
\end{equation}
%
where $\{n_1,n_2\}^{\rm T}$ is the unit, boundary-normal vector.  

In Beaudan \& Moin \cite{BeMo:94}, zeroth-order extrapolation is used to
determine ${R_2}_{int}$ from the interior values.  We have found that
first-order (\ie\ linear) extrapolation yields an improved boundary
approximation and, since we use a block penta-diagonal matrix structure, this
is easily incorporated in the LHS.  With the modified block penta-diagonal LHS
discussed in \S\ref{s:base} we can actually implement up to cubic
extrapolation.  However, numerical experiments show that quadratic and cubic
extrapolation are both highly unstable.  Even linear extrapolation has been
found to be slightly unstable for large \CFL\ and techniques to remove this
instability are discussed below.  Assuming the boundary to be at node $N$, the
outgoing Riemann invariant is given by
%
\begin{equation}
  {R_2}_{int} = 2 {R_2}_{N-1} - {R_2}_{N-2} \period
\end{equation}

In constructing the implicit LHS operator at the boundary, the variation of
the outgoing Riemann variable with-respect-to the primitive variables is
required:
%
\begin{equation}
  \delta R_2 = n_1 \delta u + n_2 \delta v + \frac{\delta T}{(\gamma-1)
               \M_\infty \sqrt{T}} \period
\end{equation}
%
With this expression, the variation of the boundary constraints becomes
%
\begin{equation}
  \delta g_\rho = 
  \delta\rho - \left( \frac{(\gamma-1)\M_\infty}{4} \right)^\frac{2}{\gamma-1}
               \left( \frac{2}{\gamma-1} \right) 
	       \left( {R_2}_{int} - {R_1}_p \right)^\frac{3-\gamma}{\gamma-1} 
	       \\ \left( 2 \delta{R_2}_{N-1} - \delta{R_2}_{N-2} \right) \comma
\end{equation}
%
\begin{equation}
  \delta g_u = 
  \delta u - \frac{n_1}{2} \left( 2 \delta{R_2}_{N-1} - 
                                    \delta{R_2}_{N-2} \right) \comma
\end{equation}
%
\begin{equation}
  \delta g_v = 
  \delta v - \frac{n_2}{2} \left( 2 \delta{R_2}_{N-1} - 
                                    \delta{R_2}_{N-2} \right) \comma
\end{equation}
%
\begin{equation}
  \delta g_w = \delta w \comma
\end{equation}
%
\begin{equation}
  \delta g_T = 
  \delta T - 2 \left( \frac{(\gamma-1)\M_\infty}{4} \right)^2
             \left( {R_2}_{int} - {R_1}_p \right)
	     \left( 2 \delta{R_2}_{N-1} - \delta{R_2}_{N-2} \right) \period
\end{equation}

\subsubsection{Boundary damping \label{sss:damping} }

As indicated in the previous section, the use of the Riemann extrapolation
inflow condition results is a slight instability and the generation of
numerical noise at the inflow boundary.  This noise can eventually corrupt the
entire solution, even near the wall.  To control the instability and prevent
numerical noise from affecting the rest of the field, a boundary damping term
is added to the RHS of equation (\ref{e:base}).  This term takes the form of a
fourth-derivative dissipation term in computational space and can be written
as
%
\begin{equation} \label{e:fourth_diss}
  \bD_b = -\epsilon_d \sigma_d(\eta) \left( 
          \Delta_\xi^4  \frac{\partial^4 \bU}{\partial \xi^4} + 
          \Delta_\eta^4 \frac{\partial^4 \bU}{\partial \eta^4} \right) 
\end{equation}
%
where $\epsilon_d$ controls the level of dissipation and $\sigma_d(\eta)$
takes the form \cite{GuAd:93}
%
\begin{equation} \label{e:sigma}
  \sigma_d(\eta) = 1 - 10^{-\left[a_d \left( \frac{\eta - \eta_s}
                           {1 - \eta_s} \right) \right]^{n_d} }
\end{equation}
%
with $a_d = 2.57$, $n_d = 4$, and $\eta_s = 0.8$.  Using these parameters,
$\sigma_d$ varies smoothly from one at the inflow boundary to zero at $\eta_s$
such that the dissipation term acts over 20\% of the domain in $\eta$.  It is
important to note that the $\sigma_d$ term is constructed so that the
dissipation term only affects the flow near the inflow, Riemann boundary.
This guarantees that the dissipation term is zero inside the actual viscous
boundary layer.  A typically value of $\epsilon_d$ is $0.0125$ and this is the
value used for the mean-flow calculations described in Chapter \ref{c:pcyl}.

Considering the $\eta$ direction (analogous expressions hold for $\xi$), the
fourth-de\-riv\-a\-tive operator is approximated in the interior of the domain
by the fourth-order accurate equation
%
\begin{equation}
 (\Delta\eta)^4 \left( \frac{\partial^4 f}{\partial \eta^4} \right)_i \approx 
 \frac{1}{6} 
 \Big[  - \left( f_{i+3} + f_{i-3} \right) +
       12 \left( f_{i+2} + f_{i-2} \right) -
       39 \left( f_{i+1} + f_{i-1} \right) +
       56 f_{i} \Big] \period
\end{equation}
%
This expression has a seven-point stencil which requires boundary closures for
the boundary node and the two nodes adjacent to the boundary.  

On the second node in from the boundary, the second-order accurate
fourth-derivative is used:
%
\begin{equation} \label{e:fourth}
 (\Delta\eta)^4 \left( \frac{\partial^4 f}{\partial \eta^4} \right)_3 \approx 
 \Big[   \left( f_{5} + f_{1} \right) -
       4 \left( f_{4} + f_{2} \right) +
       6 f_{3} \Big] \period
\end{equation}
%
On the first node in from the boundary, the fourth-derivative dissipation term
(\ref{e:fourth_diss}) is replaced, locally, with a second-derivative
dissipation term of the form
%
\begin{equation}
  \bD_b = -\epsilon_d \sigma_d(\eta) \left( 
          \Delta_\xi^2  \frac{\partial^2 \bU}{\partial \xi^2} + 
          \Delta_\eta^2 \frac{\partial^2 \bU}{\partial \eta^2} \right) \comma
\end{equation}
%
where $\epsilon_d$ and $\sigma_d(\eta)$ are the same as that used in equation
(\ref{e:fourth_diss}).  The second derivative operators are then approximated
by second-order accurate differences, where, for example
%
\begin{equation} 
 (\Delta\eta)^2 \left( \frac{\partial^2 f}{\partial \eta^2} \right)_2 \approx  
 \Big[ \left( f_{3} - f_{1} \right) + 2 f_{2} \Big] \period
\end{equation}
%
Finally, on the actual boundary node, no dissipation term is used.

The seven-point stencil, used for the dissipation term (\ref{e:fourth_diss}) in
the interior, is not compatible with the block penta-diagonal structure of the
implicit LHS.  Because of this, we use the the second-order, fourth-derivative
stencil (\ref{e:fourth}) for the interior nodes in the LHS and with this
approximation, we have experienced no noticeable reduction in convergence of
the iterative solver.

As an example of the success of the boundary damping term in stabilizing the
Riemann extrapolation inflow boundary condition, figures \ref{f:M=0.1p3} and
\ref{f:M=0.1p4} show contours of the pressure coefficient [see equation
(\ref{e:Cp})] in the far-field and near the wall boundary for an unswept
parabolic-cylinder at $\M=0.1$ and $\Re=1000$ (see \S\ref{s:pcylmean} for
details) without boundary damping.  For such low Mach number flows, the
pressure field is a sensitive indicator of numerical oscillations.  Because of
the approximate nature of the Riemann extrapolation used on the inflow
boundary, numerical oscillations are generated that corrupt the entire
solution.  These oscillations are particularly acute near the wall, where the
mesh is highly stretched and oscillations inside the boundary layer are
particularly deleterious since first and second derivatives of the mean
boundary-layer profiles are required for linear stability analysis.  Figures
\ref{f:M=0.1p1} and \ref{f:M=0.1p2} show the pressure field for the same
conditions with boundary damping turned on [$\epsilon_d = 0.0125$ and
$\sigma_d$ given by (\ref{e:sigma})].  The pressure field is now smooth in both
regions with no sign of numerical oscillations.  Figure \ref{f:uderiv} shows a
streamwise velocity profile from the same simulation along with the
derivatives needed for stability analysis.  With boundary damping, both first
and second derivatives are smooth.

%Another quantity that is very sensitive to numerical oscillations is the
%divergence of the velocity field, or dilatation.  Figure \ref{f:M=0.1dil}
%shows contours of dilatation, both in the far-field and near the wall.  Near
%the wall the dilatation is smooth with no signs of numerical noise.  In the
%far-field, the dilatation is generally smooth except for a wiggle near the
%inflow boundary.  This wiggle is likely due to a combination of the one-sided
%differencing used to compute the divergence, and the approximate Riemann
%boundary condition.  In either case, the wiggle is local to the inflow
%boundary and the magnitude of the dilatation in that region is very
%small---less than $1\times 10^{-8}$.

In summary, the boundary damping term kills numerical oscillations at their
source and prevents them from corrupting the solution near the wall.  Since
the boundary damping term is nonzero only near the inflow boundary, no
noticeable effects on boundary-layer properties have been observed.

\subsubsection{Parabolized outflow boundary \label{sss:ParabolicBC} }

In constructing the computational domain for leading-edge receptivity
problems, the computational expense can be significantly reduced by truncating
the downstream portion of the domain (\eg\ the trailing edge of a wing).
Doing so, however, introduces difficulties in specifying accurate and stable
boundary conditions on the downstream, outflow boundary.  This difficulty is
particularly acute, since the outflow boundary intersects not only a potential
flow region, but also the viscously dominated boundary layer.  Use of
inviscid, nonreflecting boundary conditions, such those based on Riemann
invariants (similar to \S\ref{s:inflow}) or characteristics
\cite{PoLe:92,Giles:90,Thompson:90}, work well for the outer potential flow;
but cause large errors in the viscous layer even when viscous corrections
\cite{PoLe:92} are employed.  In a similar context, Fenno \cite{Fenno:93}
recently used zeroth- and first-order extrapolation on an outflow boundary.
However, numerical tests using these boundary conditions with the current
low-dissipation numerical scheme show them to be highly unstable with large
numerical errors generated.

These difficulties motived the development of a new, outflow
boundary-condition (for steady-state flows) in which the Parabolized
Navier--Stokes (PNS) equations are used to predict the flow on the outflow
boundary.  The PNS equations are derived from the NS equations by removing the
streamwise viscous derivatives and suppressing the upstream propagating
acoustic waves in subsonic regions.  The PNS were first used by Vigeron \etal
\cite{ViRaTa:78} who modified the pressure gradient in the streamwise
direction to remove the upstream propagating acoustic wave.  More recently,
\cite{ChMe:89} show that flux vector splitting naturally leads to a family of
parabolized Navier--Stokes equations.  Regardless of the approach, the PNS
equations have been shown to yield accurate solutions for high Reynolds number
flow as long as the freestream Mach number is supersonic \cite{ViRaTa:78}.
However, when applied to fully subsonic flow, poor solutions can be obtained
since the external inviscid field (which drives the boundary layer) is often
inaccurate.\footnote{The inaccuracies are greatest for geometries with curved
walls where the streamwise pressure gradient is significant.  Accurate
solutions can be obtained for flat-plate, zero pressure-gradient
boundary-layers since Vigeron's method naturally removes the streamwise
pressure-gradient in subsonic flow.}

The basic weakness of the PNS is that the streamwise pressure gradient is
incorrectly predicted when the equations are parabolized.  Thus, directly
applying the PNS on the outflow boundary can lead to large errors since the
local streamwise pressure gradient is incorrect.  However, if the streamwise
pressure gradient on the boundary can be reasonably estimated {\it a priori},
then it can be treated as a known source term in the equations.  This is the
technique used here to implement a parabolized outflow boundary condition.
The streamwise pressure gradient is estimated from the potential solution
about the body and used as a source term in the PNS at the outflow
boundary.\footnote{A better estimate of the pressure gradient would require
viscous/inviscid interaction to be accounted for.  However, for high Reynolds
number flows with thin boundary layers, relative to a characteristic body
length scale, the displacement thickness is small.}  This method is very
successful in providing an accurate boundary treatment for both the inviscid
and viscous regions of the flow.

This is demonstrated in figure \ref{f:Cp-outBC} which shows contours of the
pressure coefficient [see equation (\ref{e:Cp})] near the outflow boundary for
a typical meanflow over a parabolic cylinder (see \S\ref{s:pcylmean} for more
details).  There is no indication of numerical oscillations in the far-field.
Note that this solution does use boundary damping (discussed in the previous
section) on the inflow boundary, but this does not affect the lower half of
the outflow boundary.  Figure \ref{f:d2-outBC} shows the evolution of the
displacement thickness squared plotted as a function of the arc-length along
the parabola.  Eventually a nearly linear distribution is established
downstream and there is no indication of upstream influence due to the outflow
boundary.  This result is to be compared with figure \ref{delta} which shows a
similar plot (although for a different geometry and nondimensionalization)
which shows significant upstream influence when a characteristic based
boundary condition \cite{PoLe:92} with viscous corrections is used.

For most of the computations reported here, we have noticed no instability or
convergence problems associated with the parabolic outflow boundary treatment.
However, for the high Reynolds number cases reported in Chapter \ref{c:pcyl}, a
slight instability was present when the residual approached $1\times 10^{-7}$
in magnitude.  However, the solution at this point was judged to be
sufficiently converged so that further investigation of the instability was
not performed.

\subsection{Disturbance Boundary Treatments \label{ss:distbc}}

Disturbance-flow solutions are obtained using no-slip boundary conditions on
the wall with either isothermal or adiabatic temperature boundary conditions,
symmetry conditions upstream of the leading edge, and sponge layer boundary
treatments on the inflow and outflow boundaries.  The symmetry conditions are
implemented in the finite-difference stencil in the same manner used for the
base-flow.  The detailed implementation of each of the other boundary
treatments is presented below.

\subsubsection{Wall boundary conditions \label{sss:distwallbc} }

For disturbance calculations we consider both undisturbed surface boundaries
and surface boundaries with small perturbations to model surface roughness.
The undisturbed surface boundary conditions follow directly from the wall
boundary conditions used for the mean-flow in \S\ref{sss:wallbc}.

When the wall boundary is perturbed to model roughness, the no-slip and
temperature boundary conditions must be applied on the perturbed surface.
Denoting the smooth wall by $n(s,z)=0$ the rough wall is obtained by adding a
small, spanwise-periodic perturbation given by
%
\begin{equation}
  \tilde n(s,z) = \ew h_w(s) e^{i \bw z}
\end{equation}
%
where $\ew = \ew^* / L^* \ll 1$ is the nondimensional roughness height,
$h_w(s)$ is the streamwise shape of the bump, and $\bw = \bw L^*$ is the
spanwise wavenumber of the bump.

Similar to the expansions used in section~\ref{s:linNS}, the flow variables
can be written as
%
\begin{eqnarray}
    u_j(x_i) &=& \bar u_j(x_i) + \ew \tilde u_j(x_i) + 
                 \O(\ew^2) \comma \\
      T(x_i) &=& \bar T(x_i)   + \ew \tilde T(x_i) +   
                 \O(\ew^2) \comma
\end{eqnarray}
%
where perturbations are denoted with a tilde.  For all the cases considered
here, we assume that $\ew$ is sufficiently small such that the $\O(\ew^2)$ and
higher terms can be neglected and that the surface boundary conditions can be
transfered by Taylor series expansion to inhomogeneous boundary conditions on
the undisturbed surface.  The detailed derivation of these inhomogeneous
boundary conditions is given in appendix~\ref{a:bumpbc}.  Summarizing those
results for an isothermal boundary, the appropriate linearized wall boundary
conditions for both the mean and disturbances quantities are given by
%
\begin{eqnarray}
  \bar u_j(s,0,z) &=& 0, \\
  \bar T(s,0,z)   &=& T_w, \\
  u'_j(s,0,z)     &=& -h_w(s) e^{i \bw z} 
                       \frac{\partial \bar u_j}{\partial n}(s,0,z), \\
  T'(s,0,z)       &=& -h_w(s) e^{i \bw z} 
                       \frac{\partial \bar T}{\partial n}(s,0,z) .
\end{eqnarray}
%
Not surprisingly, the mean boundary conditions are unchanged from the
unperturbed case, while the disturbance boundary conditions are now
inhomogeneous and depend on the wall-normal gradients of the mean flow.  The
linearized temperature boundary condition for an adiabatic wall is
considerably more complicated then the isothermal condition and is given by
equation (\ref{e:bump.tn}) in Appendix \ref{a:bumpbc}.  The constraint
equations for the implicit LHS are formed similar to \S\ref{sss:wallbc}.

\subsubsection{Sponge Boundary Treatment \label{sss:sponge} }

When performing receptivity calculations one often encounters situations where
waves of some type (inviscid and/or viscous) must pass through a computational
boundary without producing significant reflections.  Much work has been
performed to develop nonreflecting boundary conditions for hyperbolic systems
of equations such as the Euler equations, and the interested reader is
referred to the review article by Givoli \cite{Givoli:91} and the references
therein for further information.  However, these boundary conditions are
typically only valid in regions where the mean flow is uniform. Furthermore,
when used with the Navier--Stokes equations, these nonreflecting conditions are
only useful away from viscous regions (like boundary layers and free-shear
layers).  For receptivity calculations the viscous layers typically contain
instability waves which also must exit the computational domain.
%
%.... Could use another reference to a NRBC review article.
%

For boundary layer instability waves, various outflow conditions have been
used in the past with varying degree of effectiveness.  Many of these
conditions have been developed for incompressible flow.  Recently Guo and
Adams \cite{GuAd:93} have reported results for compressible flow in which they
attempted a variety of outflow boundary conditions and determined the impact of
these conditions on the growth of TS waves in a boundary layer.  They find
that the sponge-layer boundary treatment gives adequate results compared to
buffer regions and characteristic-based conditions.  Given this, and the fact
that the sponge layer has proven to be rather robust, we have adopted the
sponge layer as our primary outflow boundary treatment.  We have also
implemented the sponge layer on inflow boundaries and details are provided
below.

Following Orszag \& Israeli \cite{IsOr:81}, the sponge layer is implemented by
adding the following term to the RHS of the NS equations
%
\begin{equation} \label{e:sponge}
  \bF_s = \sigma_s(x,y) \left[ \bU(x,y,z,t) - \bU_{ref}(x,y,z,t) \right]
\end{equation}
%
where $\bU_{ref}$ is an arbitrary reference state which can be a function of
the spatial coordinates and of time.

The sponge function is designed to vary smoothly from zero in the interior to
a finite value, $A_s$, on the boundary.  For this purpose, we use the
following function which was suggested by Mahesh \cite{Mahesh:96} as a
simplification to the functions used by Adams \cite{Adams:94}:
%
\begin{equation} \label{e:spgfun}
  \sigma_s(x) = \left\{ \begin{array}{ll}
                A_s \left( {{x - x_s}\over{x_t - x_s}} \right )^{N_s} & 
                x \in (x_s,x_t] \\
	        0 & \mbox{otherwise}
	        \end{array}
	        \right. \label{spgfun} \period
\end{equation}
%
In this expression, $x_s$ denotes the start of the sponge, $x_t$ the end of
the sponge, and $N_s$ is the exponent (generally taken to be 3) which
determines the rate at which the function transitions from 0 to $A_s$.  The
effectiveness of this sponge is documented in \S\ref{ss:TSspace} for
Tollmien--Schlichting waves, \S\ref{ss:CFspace} for cross-flow vortices, and
\S\ref{s:scat} for acoustic waves.

We note in passing that recent work by Colonius \etal \cite{CoMoLe:95} makes
use of a ``sponge-layer'' outflow boundary treatment to damp reflections due
to the passage of large vortical disturbances.  Instead of adding a damping
term to the equations, this technique uses an exit zone comprised of a
stretched mesh combined with explicit filtering to damp disturbances before
they reach the outflow boundary.  Although shown to produce acceptable
results, this method has the disadvantage of coupling the mesh point
distribution to the boundary treatment.  With the current method, the mesh can
be constructed independent of the form of the sponge term, with the caveat that
the spatial gradients which occur in the sponge region must be adequately
resolved.
%
%.... TTD: Could show some plane-wave acoustic tests here ?
%

%==============================================================================
%
%  Tables and Figures
%
%==============================================================================
%
%.... Computational domain
%
\begin{figure}[p]
\centering \epsfxsize=5.0in \epsfbox{figures/ch3/schematic.ai}
\caption {Generic geometry for leading-edge receptivity
calculations. \label{f:domain}}
\end{figure}
%
%.... Block matrix structure
%
\begin{figure}[p]
\centering
\epsfxsize=4.0in \epsfbox{figures/ch3/tri.ai}
\caption[Modified, block tri-diagonal matrix structure.] {Modified, block
tri-diagonal structure for mean-flow LHS matrix.  Each block is a $5 \times 5$
dense matrix and the shaded blocks denote the boundary nodes which spoil the
interior block, tri-diagonal structure. \label{f:tri}}
\end{figure}
%
\begin{figure}[p]
\centering
\epsfxsize=4.0in \epsfbox{figures/ch3/penta.ai}
\caption[Modified block, penta-diagonal matrix structure] {Modified block,
penta-diagonal structure for disturbance-flow LHS matrix. Each block is a $5
\times 5$ dense matrix and the shaded blocks denote the boundary nodes which
spoil the interior block, penta-diagonal structure. \label{f:penta}}
\end{figure}
%
%.... Pressure without boundary damping
%
\begin{figure}[p]
\centering
\figlab 3.3in 0in {$x$}
\figlab -0.1in 3.225in {$y$}
\epsfxsize=3.25in \epsfbox{figures/ch5/cp_nd.eps}
\caption[Contours of the pressure coefficient for the unswept parabolic
cylinder with no boundary damping] {Contours of $C_p$ for the unswept
parabolic cylinder with no boundary damping at $\M=0.1$, $\Re=1000$,
$\Pr=0.7$.  The contours range from 0 to 1.04 with increments of 0.04; tics
every 100 nose radii. \label{f:M=0.1p3}}
\end{figure}
%
\begin{figure}[p]
\centering
\figlab 3in 0in {$x$}
\figlab -0.1in 2.9in {$y$}
\epsfxsize=5.5in \epsfbox{figures/ch5/cp_nd_le_cu.ai}
\caption[Contours of the pressure coefficient near the leading-edge for the
unswept parabolic cylinder without boundary damping] {Contours of $C_p$ near
the leading-edge for the unswept parabolic cylinder without boundary damping
at $\M=0.1$, $\Re=1000$, $\Pr=0.7$.  The contours range from 0 to 1.04 with
increments of 0.04; tics every nose radii. \label{f:M=0.1p4}}
\end{figure}
%
%.... Pressure with boundary damping
%
\begin{figure}[p]
\centering
\figlab 3.3in 0in {$x$}
\figlab -0.1in 3.225in {$y$}
\epsfxsize=3.25in \epsfbox{figures/ch5/cp.eps}
\caption[Contours of the pressure coefficient for the unswept parabolic
cylinder] {Contours of $C_p$ for the unswept parabolic cylinder at $\M=0.1$,
$\Re=1000$, $\Pr=0.7$.  The contours range from 0 to 1.04 with increments of
0.04; tics every 100 nose radii. \label{f:M=0.1p1}}
\end{figure}
%
\begin{figure}[p]
\centering 
\figlab 3in 0in {$x$} 
\figlab -0.1in 2.9in {$y$} 
\epsfxsize=5.5in
\epsfbox{figures/ch5/cp_le_cu.ai}
\caption[Contours of the pressure coefficient near the leading-edge for the
unswept parabolic cylinder] {Contours of $C_p$ near the leading-edge for the
unswept parabolic cylinder at $\M=0.1$, $\Re=1000$, $\Pr=0.7$.  The contours
range from 0 to 1.04 with increments of 0.04; tics every nose
radii. \label{f:M=0.1p2}}
\end{figure}
%
%.... Boundary layer derivative examples
%
\begin{figure}[p]
\centering \sethlabel{$x$} 
\setvlabel{$u, u_{,n}, u_{,nn}$} 
\epsfxsize=5.5in
\epsfbox{figures/ch3/uderiv.eps}
\caption[Derivatives of the streamwise velocity at $x = 172$ for the unswept
parabolic cylinder] {Derivatives of the streamwise velocity at $x = 172$ for
the unswept parabolic cylinder with boundary damping at $\M=0.1$, $\Re=1000$,
$\Pr=0.7$: \solid $u$, \dashed $u_{,n}$, \dotted $u_{,nn}$.  \label{f:uderiv}}
\end{figure}
%
%.... Parabolic outflow boundary condition
%
\begin{figure}[p]
\centering
\figlab 3.3in 0in {$x$}
\figlab -0.1in 3.225in {$y$}
\epsfxsize=3.25in 
\epsfbox{figures/ch3/pcyl-ns.eps}
\caption[Contours of the pressure coefficient for $\M=0.1$, $\Re=1000$ flow
over a parabolic cylinder] {Contours of $C_p$ for $\M=0.1$, $\Re=1000$ flow
over a parabolic cylinder. Contours are from 0 to 1.2 in increments of 0.01;
tics every $100$ nose radii. \label{f:Cp-outBC}}
\end{figure}
%
\begin{figure}[p]
\centering
\sethlabel{$s$}
\setvlabel{$\delta_1^2$}
\epsfxsize=5.5in \epsfbox{figures/ch3/pcyl-d2.eps}
\caption {Evolution of the displacement thickness squared for a $\M=0.1$ and
$\Re=1000$ boundary layer on a parabolic cylinder. \label{f:d2-outBC}}
\end{figure}

