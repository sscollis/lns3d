%%%%%%%%%%%%%%%%%%%%%%%%%%%%%%%%%%%%%%%%%%%%%%%%%%%%%%%%%%%%%%%%%%%%%%%%%%%%%%%
%
%  Abstract
%
%  S. Scott Collis
%
%  Written: 9-5-95
%
%  Revised: 
%
%%%%%%%%%%%%%%%%%%%%%%%%%%%%%%%%%%%%%%%%%%%%%%%%%%%%%%%%%%%%%%%%%%%%%%%%%%%%%%%

Laminar-flow wings have the promise of reducing viscous drag forces in cruise
for commercial aircraft.  However, the success of a laminar-flow wing depends
critically on the external disturbance environment and how these disturbances
influence the transition from laminar to turbulent flow.  The process by which
external disturbances are converted into instability waves, which are the
precursors to turbulence, is called receptivity.

The research described in this \thesis\ focuses on the receptivity of the
three-dimensional boundary layer near the leading edge of a high-speed swept
wing.  In particular, the influence of surface roughness near the leading edge
is examined as it relates to the formation of stationary crossflow vortices in
the boundary layer.  Recent experiments indicate that surface roughness is the
primary cause of stationary crossflow vortices which are observed to dominate
the laminar to turbulent transition process on swept wings.  Therefore, the
mechanisms responsible for the formation of crossflow vortices and the
accurate prediction of their initial amplitude are essential for the
development of laminar-flow wings.

The receptivity and early evolution of stationary crossflow vortices is
investigated \linebreak through numerical solutions of the linearized
Navier--Stokes equations for a swept leading-edge.  Consistent with findings
for other geometries, convex surface curvature stabilizes crossflow vortex
growth while nonparallel effects are destabilizing.  In contrast, the initial
amplitude of crossflow vortices downstream of a localized surface roughness
site is found to be greater in the presence of convex surface curvature, while
the nonparallel meanflow near a leading-edge is found to strongly reduce the
initial amplitude of crossflow vortices.  These competing effects---curvature
and nonparallelism---tend to counteract one another, but, for the conditions
studied here, the nonparallel effect is dominant.

Comparisons between linearized Navier--Stokes solutions and recent theoretical
receptivity analysis, based on the parallel-flow equations, show that the
theoretical method over-predicts the initial amplitude of stationary crossflow
vortices by as much as $77\%$ for long wavelength disturbances.  The error in
the theoretical prediction is reduced for shorter wavelengths, since
nonparallel effects are relatively less important, but remains as high as
$30\%$ near the leading-edge.  Although the parallel theory provides a
conservative estimate of the initial amplitude of crossflow vortices, it is
concluded that accurate theoretical prediction of crossflow receptivity near
the leading-edge of a swept wing requires the inclusion of nonparallel
effects.