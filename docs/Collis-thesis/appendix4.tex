%%%%%%%%%%%%%%%%%%%%%%%%%%%%%%%%%%%%%%%%%%%%%%%%%%%%%%%%%%%%%%%%%%%%%%%%%%%%%%%
%
%  Appendix 4
%
%  S. Scott Collis
%
%  Written: 9-5-95
%
%  Revised: 9-18-96
%
%%%%%%%%%%%%%%%%%%%%%%%%%%%%%%%%%%%%%%%%%%%%%%%%%%%%%%%%%%%%%%%%%%%%%%%%%%%%%%%
\chapter{Compressible Stability Solver \label{a:LST}}

The Linear Stability Theory (LST) solver described in this appendix is based
on the compressible, linearized-disturbance-equations developed in Chapter
\ref{c:eqn}.  Here, these equations are extended to account for surface
curvature in the context of an infinite-span swept-wing.  The quasi-parallel
assumption is then used to form the stability equations and the numerical
methods utilized to obtain the eigensolutions to the stability equations are
presented.  Finally, a perturbation method is introduced to correct the
stability results for nonparallel effects and the numerical method implemented
to solve for the nonparallel correction is discussed.  With both surface
curvature and nonparallel effects included, the stability solver described in
this appendix represents the state-of-the-art in local linear stability
analysis.

For further information on linear stability, the reader is referred to the
very thorough review of the compressible stability equations and their
solution by Mack \cite{Mack:84a}.  Another useful review of stability
analysis, especially as applied to three-dimensional boundary layers, is given
by Malik \cite{Malik:90a}.

\section{Disturbance Equations on a Curved Surface \label{s:LSTeqn} }

Recent stability analyses \cite{MaMa:94,MaBa:93} have shown that surface
curvature has a significant quantitative effect on the growth-rates of
stationary crossflow vortices.  To account for this effect, we introduce the
body-fitted, orthogonal coordinate system where, $s$ is the coordinate tangent
to the body, $n$ is normal to the body, and $z$ is the spanwise coordinate.
The disturbance equations (\ref{e:pcon})--(\ref{e:peng}) corresponding to the
linearized Navier--Stokes (LNS) equations are then rewritten using
coordinate-system independent, vector notation
%
\begin{equation} \label{e:vcon}
  \rho'_{,t} + \nabla \rhom \cdot \bu' + \nabla \rho' \cdot \bum +
  \rhom \nabla \cdot \bu' + \rho' \nabla \cdot \bum = 0 \comma
\end{equation}
%
\begin{eqnarray} \label{e:vmom}
  \rhom \bu'_{,t} + \rho' ( \bum \cdot \nabla ) \bum + 
  \rhom ( \bu' \cdot \nabla ) \bum + \rhom ( \bum \cdot \nabla ) \bu' +
  \nabla p' = \\ \nonumber 
  \frac{1}{\Re}\Big[ \nabla ( \bar\lambda \nabla \cdot \bu' +
                              \lambda' \nabla \cdot \bum ) + 
                   2 \nabla \cdot ( \bar\mu {\bf S'} + 
		                    \mu' \bar{\bf S} ) \Big] \comma
\end{eqnarray}
%
\begin{eqnarray} \label{e:veng}
  \rhom T'_{,t} + \rho' ( \bum \cdot \nabla ) \Tm + 
  \rhom ( \bu' \cdot \nabla ) \Tm + \rhom ( \bum \cdot \nabla ) T' + 
  \\ \nonumber \rhom(\gamma-1) \Tm \nabla \cdot \bu' + 
  \rhom (\gamma-1) T' \nabla \cdot \bum + 
  \rho' (\gamma-1) \Tm \nabla \cdot \bum = \\ \nonumber
  \frac{\gamma}{\Pr\Re} \Big[ \nabla\bar\kappa \cdot \nabla T' +
  \nabla\kappa' \cdot \nabla\Tm + \bar\kappa\nabla^2 T' + 
  \kappa'\nabla^2\Tm \Big] + \\ \nonumber
  \frac{\gamma(\gamma-1)\M^2}{\Re} \Big[ (2 \bar\lambda \nabla \cdot \bum)
  \nabla \cdot \bu' + \lambda' (\nabla \cdot \bum)^2 + 
  2 \mu' \bar{\bf S} \cdot \bar{\bf S} + 
  4 \bar\mu \bar{\bf S} \cdot {\bf S'} \Big] \comma
\end{eqnarray}
%
where $\bu$ denotes the velocity vector $\{v_s,v_n,w\}^{\rm T}$ (in the
body-fitted coordinate system), and $\nabla$ is the generalized gradient
operator.  Differential distances in the $s$-direction are given by $h(s,n)
ds$ where $h(s,n)$ is the coordinate system metric and metrics in the other
coordinate directions are unity.  Each of the differential operators in
equations (\ref{e:vcon})--(\ref{e:veng}) can be written as a combination of
partial derivatives with respect to $(s,n,z)$ and additional terms involving
the metric and its derivatives.  For example, the gradient of a scalar
function, $f$, is given by
%
\begin{equation}
  \nabla f = \frac{1}{h} f_{,s} \hat s + f_{,n} \hat n + f_{,z} \hat z
\end{equation}
%
where $\hat s$, $\hat n$, $\hat z$ are unit vectors in the coordinate
directions.  Likewise, the divergence of a vector is written as
%
\begin{equation}
  \nabla \cdot \bu = \frac{1}{h} ( v_{s,s} + v_n h_{,n} ) + v_{n,n} + u_{z,z} .
\end{equation}
%
A complete description of all the generalized, differential operators required
for the Navier--Stokes equations is given in Appendix~A of Sherman
\cite{Sherman:90}.  Using these expressions, the LNS equations can be converted
to a compact matrix form, analogous to equation (\ref{e:pNS}),
%
\begin{eqnarray} \label{e:cNS}
  \bU'_{,t} + \bA \bU'_{,s} + \bB \bU'_{,n} + \bC \bU'_{,z} + \bD \bU' =
  \bVss \bU'_{,ss} + \bVsn \bU'_{,sn} + \nonumber \\
  \bVsz \bU'_{,sz} + \bVnn \bU'_{,nn} + \bVnz \bU'_{,nz} + 
  {\bf V}_{zz} \bU'_{,zz}
  \period 
\end{eqnarray}
%
The matrices appearing in equation (\ref{e:cNS}) are defined in Appendix
\ref{a:curvmat}.  When curvature is removed, by setting $h=1$, there is a
direct correspondence between $(s,n,z)$ and $(x,y,z)$ and equations
(\ref{e:cNS}) and (\ref{e:pNS}) are equivalent.  The reader should note that
unlike recent stability analysis of the flow over the circular cylinder
\cite{MaMa:94}, our formulation allows for the mesh metric, $h$, to vary in
both $n$ and $s$, not just $n$.  This greatly complicates the structure of the
matrices given in Appendix \ref{a:curvmat} but with the benefit that arbitrary
two-dimensional bodies can be analyzed.

\section[Quasi-Parallel Stability Analysis for an Infinite-Span 
Swept-Wing] {Quasi-Parallel Stability Analysis for an \protect \\
Infinite-Span Swept-Wing \label{s:QPLST} }

Given the body-fitted coordinate system described in the previous section,
additional coordinate systems used for stability analysis of an infinite-span
swept-wing are shown in figure \ref{f:LSTcoord} where it is assumed that all
quantities are nondimensionalized as in \S\ref{s:eqn}.  The external
freestream velocity makes an angle $\theta$ with the $s$-axis.  The coordinate
system aligned with the local edge-velocity is called the streamline
coordinates, $(s_s,n,z_s)$, and the local streamwise direction, $s_s$, makes
an angle $\theta_e$ with the chordwise direction.  The angle between the local
streamwise direction and the freestream velocity is $\theta_p = \theta_e -
\theta$.

For an infinite-span swept-wing, all mean $z$-gradients are zero by
definition.  By further making the parallel flow assumption, all mean
$s$-derivatives are neglected and the disturbance solution can be written in
terms of normal modes
%
\begin{equation} \label{e:normalmode}
   \bU'(s,n,z,t) = \bhU(n) \: e^{i( k_s s + k_z z - \omega t)}
\end{equation}
%
where $\bhU(n) = \{ \hat\rho, \hat v_s, \hat v_n, \hat w, \hat T\}^{\rm T}$ is
the complex valued eigenfunction which is a function only of the wall normal
coordinate, $k_s$ is the chordwise wavenumber, $k_z$ is the spanwise
wavenumber, and $\omega$ is the angular frequency.  In writing equation
(\ref{e:normalmode}) it is understood that the actual solution is given by the
real part of the left-hand-side.  Substituting the normal-mode expression into
the LNS equations (\ref{e:cNS}) results in the following ordinary differential
equation
%
\begin{eqnarray} \label{e:OS}
  -i \omega \bhU + i k_s \bA \bhU + \bB \bhU_{,n} + i k_z \bC \bhU + \bD \bhU
  k_s^2 \bVss \bhU \ - & &\cr i k_s \bVsn \bhU_{,n} +
  k_s k_z \bVsz \bhU - \bVnn \bhU_{,nn} -
  i k_z \bVnz \bhU_{,n} + k_z^2 \bVzz \bhU  &=& 0 \period
\end{eqnarray}
%
Given a mean-flow profile $\bar\bU(n) = \{\bar\rho,\bar v_s,\bar v_n,\bar
w,\bar T\}^{\rm T}$, and its first and second derivatives in the wall-normal
direction, $\bar\bU_{,n}$, $\bar\bU_{,nn}$, the matrices in equation
(\ref{e:OS}) are determined with the streamwise and spanwise derivatives of
the mean flow set explicitly to zero.  Equation (\ref{e:OS}) is solved subject
to the homogeneous boundary conditions
%
\begin{equation}  \label{e:LST-BC-u-wall}
  \hat v_s(0) = 0 \comma \quad \hat v_n(0) = 0 \comma \quad 
  \hat w(0) = 0 \comma
\end{equation}
\begin{equation} \label{e:LST-BC-t-wall}
  \hat T(0) = 0 \quad \mbox{or} \quad \hat T_{,n}(0) = 0 \comma
\end{equation} 
\begin{equation} \label{e:LST-BC-infty}
  \hat v_s, \hat v_n, \hat w, \hat T \rightarrow 0 \quad \mbox{as} 
  \quad n \rightarrow \infty \period
\end{equation}

The wavenumbers and frequency, $(k_s, k_z, \omega)$, form a set of six real
parameters since they are complex quantities in general.  With the homogeneous
boundary conditions (also mapped to computational space), equation
(\ref{e:OS}) reduces to an eigenvalue problem for any two of the six
parameters, given the other four.  Typically the eigenproblems are formulated
in either a temporal or spatial approach.  For the temporal problem the
wavenumbers are specified real values and the eigenvalue is the complex
frequency, $\omega$.  Thus a temporal eigenmode has fixed spatial wavelengths
and evolves in time with the growth rate, $\sigma$, given by the imaginary
part of $\omega$:
%
\begin{equation}
  \sigma = {\rm Im}(\omega) \period
\end{equation}
%
In the spatial problem, the frequency and one of the wavenumbers ($k_z$
herein) are specified real values and the other wavenumber ($k_s$) is the
eigenvalue.  In this case, the waves evolve in space with the growth-rate
given by
%
\begin{equation}
  \sigma = -{\rm Im}(k_s) 
\end{equation}
%
and the local chordwise wavenumber is given by $\alpha = {\rm Re}(k_s)$.

Before proceeding, the reader should note that fixing $k_z$ is a natural and
reasonable constraint for the problems at hand.  This is because we are
interested in comparing the LST predictions with LNS solutions which have a
value of $k_z$ prescribed through a receptivity process (either direct or
indirect).  The instability waves that result from this receptivity process
will have fixed $k_z$ and grow only in the chordwise direction.  This approach
is in contrast to the methods used in transition prediction where the value of
$k_z$ (possibly complex) varies in the downstream direction in such a way as
to maximize the growth-rate \cite{Malik:90,Nayfeh:80}.

With the real parts of the spatial wavenumbers denoted by $\alpha={\rm
Re}(k_s)$ and $\beta={\rm Re}(k_z)$ some useful relations can be summarized.
The magnitude of the wavenumber vector is
%
\begin{equation}
  k = \sqrt{ \alpha^2 + \beta^2 } \comma
\end{equation}
%
the angle between the $s$-axis and the wave-vector is given by
%
\begin{equation}
  \psi = \tan^{-1}\left( \frac{\beta}{\alpha} \right) \comma
\end{equation}
%
and the phase velocity is
%
\begin{equation}
  c_p = \frac{\omega}{k} \period
\end{equation}

%
%.... \section{Discrete Eigen-Solutions}
%

Before solving, equation (\ref{e:OS}) is converted to computational space
using the mapping $n \mapsto \eta$ which yields
%
\begin{eqnarray} \label{e:dOS}
  \bigg[ 
  (\bB - i k_s \bVsn - i k_z \bVnz ) \eta_{,n} \Delta_{\eta} + 
  i k_s \bA + i k_z \bC + \bD \ + & & \nonumber \\
  k_s^2 \bVss + k_z^2 \bVzz - i \omega - \bVnn ( \eta_{,n}^2 
  \Delta_{\eta\eta} + \eta_{,nn} \Delta_{\eta} ) 
  \bigg] \bhU &=& 0 \comma
\end{eqnarray}
%
where the derivatives in $\eta$ have been replaced with discrete
approximations.  We have implemented two different discretizations for the
wall normal derivatives: fourth-order accurate finite differences and
spectrally accurate Chebyshev collocation.

The finite difference operators are the same fourth-order accurate
approximations described in \S\ref{s:spacedisc} for the LNS solver.  This
method is useful when a LST solution is desired using the same or similar mesh
as used in a NS calculation.  Usually when using the finite difference method
the infinite domain in $n$ is truncated to a finite domain and the far-field
boundary condition is enforced at the finite domain boundary.  This is only an
approximation, but we have found that when the boundary is at least $25
\delta_1$ above the wall, the eigensolutions are not adversely affected.  The
finite domain in $n$ is then mapped to uniform computational space using the
algebraic mapping function defined by
%
\begin{equation}  
  a = {{(n_{max} n_s)}\over{(n_{max} - 2 n_s)}} \comma
\end{equation}
\begin{equation}
  b = 1 + {a \over n_{max}} \comma
\end{equation}
\begin{equation}  \label{e:tsmap}
  n(\eta) = {{a \eta}\over{b - \eta}} \period
\end{equation}
%
In these expressions, $n_{max}$ is the maximum vertical distance above the
wall and $n_s$ controls the minimum node spacing at the wall.  As a general
rule for subsonic mean boundary-layer profiles, we have found that $n_s
\approx 2.5 \delta_1$ provides an adequate point distribution when $n_{max}
\approx 40 \delta_1$.  In practice, the actual values of $n_s$ and $n_{max}$
are set to ensure that the eigenfunction is adequately resolved since the
quality of the solution is rather sensitive to the value of $n_s$.

Equation (\ref{e:dOS}) can also be discretized using the spectrally accurate
Chebyshev collocation operators (see Canuto \etal \cite{CaHuQuZa:88} on
page~69).  This method is used to provide reference eigensolutions for
comparison with LNS calculations (see Chapter~\ref{c:valid}) and also to
provide accurate solutions which are more economical then the FD method since
fewer points are required for the same level of accuracy.  Typically the
Chebyshev solutions are obtained by mapping the semi-infinite domain in $n$ to
a finite domain in $\eta$ using the transformation
%
\begin{equation}
  n(\eta) = n_s \frac{1+\eta}{1-\eta} \comma
\end{equation}
%
where $n_s$ again controls the minimum node spacing at the wall and a typical
value of $n_s = 2 \delta_1$.  Unlike the FD method, the Chebyshev solutions
are not particularly sensitive to the value of $n_s$ which makes this method
more robust.

Regardless of the particular discretization, the boundary conditions are
applied to the discrete equations in the same manner.  At the wall, the
momentum equations are replaced with equations (\ref{e:LST-BC-u-wall}) and for
the case of an isothermal wall, the energy equation is replaced with $\hat
T(0) = 0$.  If, instead, an adiabatic wall is used, the energy equation is
modified by explicitly setting the wall-normal derivative of both the mean and
disturbance temperature to zero.  For the disturbances, this amounts to
removing terms which multiply $\hat T_{,n}$ at the wall.  Although this
``weak'' formulation for the adiabatic constraint does not guarantee that
$\hat T_{,n}$ is strictly zero, the deviation from zero will be on the order
of the truncation error of the scheme.  In all cases, the discrete continuity
equation is used at the wall to obtain density.  At the far-field boundary,
all disturbance quantities are set to zero by replacing the equations of
motion with the constraint $\bhU = 0$.

With the boundary conditions applied to the discrete equation (\ref{e:dOS})
and the appropriate parameters set, the resulting problem is reduced to a
matrix eigenvalue problem.  If the temporal approach is used then the
eigenvalue problem can be written as
%
\begin{equation}
  \bM \bhU = \omega \bhU
\end{equation}
%
which is a $(5N_\eta \times 5N_\eta)$ regular, complex eigenvalue problem that
is solved using the IMSL routine {\tt EVCCG} \cite{IMSL:91}.

For the spatial problem, the eigenvalue, $k_s$, appears nonlinearly and the
eigenvalue problem can be written in the form
%
\begin{equation}
  ( \bC_0 + k_s \bC_1 + k_s^2 \bC_2 ) \bhU = 0 \period
\end{equation}
%
Bridges \& Morris \cite{BrMo:84a} show that this can be converted to the
following extended problem
%
\begin{equation} \label{e:extend}
  \left[ \matrix{ -\bC_0^{-1} \bC_1 & -\bC_0^{-1} \bC_2 \cr
                  \bI & {\bf 0} \cr } \right ] 
  \left\{ \matrix{\bhU_1 \cr \bhU} \right\} = 
  k_s \left\{ \matrix{\bhU_1 \cr \bhU} \right\}
\end{equation}
%
where $\bhU_1 = k_s \bhU$.  In practice, the inverse $\bC_0^{-1}$ is not
directly computed.  Instead, the $LU$-factorization of $\bC_0$ is computed
using the LAPACK routine {\tt CGETRF} \cite{Lapack:92}, and the products,
$\bC_0^{-1} \bC_1$ and $\bC_0^{-1} \bC_2$, are obtained by forward-solve and
back-substitution using the routine {\tt CGETRS} \cite{Lapack:92}.

Equation \ref{e:extend} is a $(10 N_\eta \times 10 N_\eta)$ regular, complex
eigenvalue problem which is again solved using the {\tt EVCCG} routine
\cite{IMSL:91}.  The computational expense of the $QR$ algorithm, which is the
basis of IMSL eigenvalue routine, is on the order of $N^3$, where $N$ is the
dimension of the matrix \cite{GoLo:89}.  Thus the spatial eigensystem is eight
times as expensive to solve as the temporal eigensystem and this fact
motivates the need for high-order accurate spatial differencing.
%
%.... TTD: Could put in a couple of validation cases: 
%
%   (1) Nearly incompressible, Blasius
%   (2) One of Malik's supersonic cases
%

\section{Nonparallel Effects \label{s:NPLST} }

Along with surface curvature, nonparallel effects are also found to have a
significant impact on the stability characteristics of stationary crossflow
vortices \cite{MaMa:94}.  We have included nonparallelism in our stability
calculations using a perturbation approach, pioneered by Ling \& Reynolds
\cite{LiRe:73} and Saric \& Nayfeh \cite{SaNa:75}.  Here, we follow the
approach of Saric \& Nayfeh \cite{SaNa:75} applied to the compressible linear
stability equations in curvilinear coordinates.

For high Reynolds numbers, the mean boundary layer flow, $\bar\bU$, can be
considered a slowly varying function of the streamwise coordinate, $s$.
Defining a slow coordinate, $s_1 = \enp s$, the base flow can be written as an
expansion in the parameter $\enp$
%
\begin{equation} \label{e:npbase}
  \bar\bU(s,n) = \bar\bU_0(s_1,n) + \enp \bar\bU_1(s_1,n) + \dots
\end{equation}
%
where $\enp$ is a small dimensionless parameter representing the
nonparallelism of the base flow.  Here, $\bar\bU_0(s_1,n)$ is the
quasi-parallel base flow, and $\bar\bU_1(s_1,n)$ is the first-order
nonparallel correction which, for the class of flows considered here, consists
only of the scaled, wall-normal velocity, $\bar\bU_1 = \{0, 0, \bar v_n /
\enp, 0, 0\}^{\rm T}$.

Utilizing the method of multiple scales, linear disturbances about the base
flow can be expanded in $\enp$ as
%
\begin{eqnarray} \label{e:npdist}
  \bU'(s,n,z,t) &=& \left\{ A(s_1) \bhU_0(s_1,n) + 
     \enp \bhU_1(s_1,n) + \dots\right\} \cr 
     & & \exp\left( \int i k_s(s_1) ds + i k_z z - i \omega t \right) \period
\end{eqnarray}
%
From the following analysis, we will show that $\bhU_0(s_1,n)$ is the same
quasi-parallel eigenfunction discussed is \S\ref{s:QPLST}, $A(s_1)$ is the
local wave amplitude, and $\bhU_1(s_1,n)$ is the nonparallel correction at
$\cal{O}(\enp)$.

Equations (\ref{e:npbase}) and (\ref{e:npdist}) are substituted into the
linearized Navier--Stokes equations in curvilinear coordinates (\ref{e:cNS})
and terms are grouped in powers of $\enp$.  In so doing, we note that the
matrices ($\bA$, $\bB$, $\bC$, $\bD$) can be split into parallel and
nonparallel portions, \ie
%
\begin{equation}
  \bA(\bar \bU) = \bA_0(\bar\bU_0) + \enp \bA_1\left( \frac{\partial \bar\bU_0}
                {\partial s_1},\bar\bU_1 \right) + \dots \comma
\end{equation}
%
whereas, the matrices due to viscous effects ($\bVss$, $\bVsn$, \etc) only
contain parallel flow contributions.  To $\O(\enp^0)$ we obtain the parallel
linear stability equations (\ref{e:OS})
%
\begin{eqnarray} \label{e:para}
  \L_0( \bhU_0 ) &=& -i \omega \bhU_0 + i k_s \bA_0 \bhU_0 + 
      \bB_0 \bhU_{0,n} + i k_z \bC_0 \bhU_0 + \bD_0 \bhU_0 + \nonumber \\
  & & k_s^2 \bVss \bhU_0 - i k_s \bVsn \bhU_{0,n} +
      k_s k_z \bVsz \bhU_0 - \bVnn \bhU_{0,nn} - \\
  & & i k_z \bVnz \bhU_{0,n} + k_z^2 \bVzz \bhU_0 \nonumber \\ 
  &=& 0 \nonumber \period
\end{eqnarray}
%
This equation is subject to the homogeneous boundary conditions
(\ref{e:LST-BC-u-wall})--(\ref{e:LST-BC-infty}) leading to an eigenvalue
problem for ($k_s$,$\bhU_0$) which is identical to that discussed in
\S\ref{s:QPLST}.

At $\O(\enp^1)$ the following equation is obtained
%
\begin{eqnarray} \label{e:npOS}
  \L_0(\bhU_1) &=& \Biggl\{ \bigl(2 i k_s \bVss + i k_z \bVsz - \bA_0 \bigr) 
       \bhU_0 + \bVsn \bhU_{0,n} \Biggr\} \frac{dA}{ds_1} + \nonumber \\
  & &  \Biggl\{ \bVss\left( - \frac{dk_s}{ds_1} \bhU_0 + 
       2 i k_s \bhU_{0,s_1} \right) + \bVsn \bhU_{0,s_1 n} + \\
  & & (i k_z \bVsz - \bA_0) \bhU_{0,s_1} - \L_1(\bhU_0) \Biggr\} A  \period
\end{eqnarray}
%
Here, $\L_0$ is the parallel-flow operator introduced in equation
(\ref{e:para}) and $\L_1$ an operator due to the nonparallel meanflow which
takes the form
%
\begin{eqnarray} \label{e:npcorr}
  \L_1(\bhU_0) &=& i k_s \bA_1 \bhU_0 + \bB_1 \bhU_{0,n} + 
                   i k_z \bC_1 \bhU_0 + \bD_1 \bhU_0  \period
\end{eqnarray}
%
Equation (\ref{e:npOS}) is also subject to the homogeneous boundary conditions
(\ref{e:LST-BC-u-wall})--(\ref{e:LST-BC-infty}) which are now written in terms
of the $\bhU_1$ variables.  Thus, the equations at $\O(\enp^1)$ for $\bhU_1$
are an inhomogeneous version of the parallel-flow stability equations where
the RHS forcing term depends on the eigensolution to the homogeneous problem
(\ref{e:para}) and an as-yet undetermined amplitude function $A(s_1)$.  Since
the homogeneous problem given by $\L_0(\bhU_1) = 0$ has a nontrivial
solution (\ie\ the quasi-parallel eigensolutions) the solution to the
inhomogeneous problem (\ref{e:npOS}) only has a solution if a solvability
condition is met \cite{SaNa:75}.

To simplify the construction of the solvability condition, we first convert
the equations (\ref{e:npOS}) to a system of first-order equations.  This is
accomplished using the linearized continuity equation to remove the explicit
dependence on $v'_{n,nn}$ (see Appendix \ref{a:NPmat}).  Doing so in equation
(\ref{e:npOS}) and introducing additional unknowns for first-derivatives of
$\bhU_1$ in the wall-normal direction, leads to a system of eight, first-order
ordinary differential equations of the form
%
\begin{equation} \label{e:eOS}
  \bhQ_{,n} + \bF \bhQ = \bG \frac{dA}{ds_1} + \bH A
\end{equation}
%
where 
%
\begin{equation}
  \bhQ =\left\{ \matrix{ \hat Q_1\cr \hat Q_2\cr \hat Q_3\cr \hat Q_4\cr \hat
         Q_5\cr \hat Q_6\cr \hat Q_7\cr \hat Q_8 } \right\} = 
        \left\{ \matrix{ \hat\rho\cr \hat v_s\cr \hat v_n\cr \hat w\cr \hat
         T\cr \hat v_{s,n}\cr \hat w_{,n}\cr \hat T_{,n} } \right\}_1
\end{equation}
%
is the extended vector of unknowns (note that we have dropped the subscript 1
on the definition of $\bhQ$ for brevity).  The RHS has two vector
coefficients, $\bG$ and $\bH$, where the nonparallel meanflow terms,
$\L_1(\bhU_0)$, contribute to $\bH$.  The definitions of the matrices
appearing in equation (\ref{e:eOS}) are given in Appendix \ref{a:NPmat}. The
boundary conditions for (\ref{e:eOS}) are given by
%
\begin{equation}  \label{e:npLST-BC-u-wall}
  \hat Q_2(0) = 0 \comma \quad \hat Q_3(0) = 0 \comma \quad 
  \hat Q_4(0) = 0 \comma
\end{equation}
\begin{equation} \label{e:npLST-BC-t-wall}
  \hat Q_5(0) = 0 \quad \mbox{or} \quad \hat Q_{8}(0) = 0 \comma
\end{equation} 
\begin{equation} \label{e:npLST-BC-infty}
  \hat Q_2, \hat Q_3, \hat Q_4, \hat Q_5 \rightarrow 0 \quad \mbox{as} 
  \quad n \rightarrow \infty \period
\end{equation}

The solvability condition is constructed by multiplying equation (\ref{e:eOS})
by the vector $\bhZ^{\rm T} = \{\hat Z_1,\hat Z_2,\hat Z_3,\hat Z_4,\hat
Z_5,\hat Z_6,\hat Z_7,\hat Z_8\}$ (later identified as the adjoint
eigenfunction) and integrating in the normal direction
%
\begin{equation}
  \int_0^\infty \bhZ^{\rm T} \bhQ_{,n} dn + \int_0^\infty \bhZ^{\rm T} \bF
  \bhQ dn = \frac{dA}{ds_1} 
            \int_0^\infty \bhZ^{\rm T} \bG dn + 
	    A \int_0^\infty \bhZ^{\rm T} \bH dn \period
\end{equation}
%
Integrating by parts leads to
%
\begin{equation} \label{e:ibyp}
  -\int_0^\infty ( \bhZ^{\rm T}_{,n} - \bhZ^{\rm T} \bF ) \bhQ dn +
  \bhZ^{\rm T} \bhQ \Big |_0^\infty = \frac{dA}{ds_1} 
            \int_0^\infty \bhZ^{\rm T} \bG dn + 
	    A \int_0^\infty \bhZ^{\rm T} \bH dn \period
\end{equation}
%
The solvability condition follows by requiring that both terms on the LHS of
(\ref{e:ibyp}) be zero.  Thus, we arrive at the adjoint equation (after
transposing),
%
\begin{equation} \label{e:adjoint}
  \bhZ_{,n} - \bF^{\rm T} \bhZ  = 0
\end{equation}
%
and the adjoint boundary conditions which follow from
%
\begin{equation} \label{e:adjoint2}
  \bhZ^{\rm T} \bhQ \Big |_0^\infty = 0 \period
\end{equation}
%
Since each term in (\ref{e:adjoint2}) must be zero, the boundary conditions on
$\bhQ$ require that the following conditions be imposed on the adjoint problem
%
\begin{equation}  \label{e:aLST-BC-u-wall}
  \hat Z_1(0) = 0 \comma \quad \hat Z_6(0) = 0 \comma \quad \hat Z_7(0) = 0
\end{equation}
\begin{equation} \label{e:aLST-BC-t-wall}
  \hat Z_8(0) = 0 \quad \mbox{or} \quad \hat Z_5(0) = 0 \comma
\end{equation} 
\begin{equation} \label{e:aLST-BC-infty}
  \hat Z_1, \hat Z_6, \hat Z_7, \hat Z_8 \rightarrow 0 \quad \mbox{as} 
  \quad n \rightarrow \infty \period
\end{equation}
%
Thus, equations (\ref{e:adjoint}) and boundary conditions
(\ref{e:aLST-BC-u-wall})--(\ref{e:aLST-BC-infty}) constitute the adjoint
eigenvalue problem and when these equations are used in equation
(\ref{e:ibyp}) the solvability condition becomes
%
\begin{equation} \label{e:solvability}
  h_1(s_1) \frac{dA}{ds_1} + h_2(s_1) A = 0
\end{equation}
%
where
\begin{equation}
  h_1(s_1) = \int_0^\infty \bhZ^{\rm T} \bG dn
\end{equation}
%
and
%
\begin{equation}
  h_2(s_1) = \int_0^\infty \bhZ^{\rm T} \bH dn \period
\end{equation}
%
In practice, before solving equation (\ref{e:solvability}), we convert back
from the slow variable $s_1$ to $s$ by multiplying both sides by $\enp$,
leading to
%
\begin{equation} \label{e:solve}
  h_1(s) \frac{dA}{ds} + h_2(s) A = 0 \period
\end{equation}
%
In transforming back to $s$, all the terms requiring derivatives with respect
to $s_1$ are converted to derivatives in the physical coordinate $s$, and the
nonparallel meanflow terms are now of the form $\enp\bar\bU_1 = \{0,0,\bar
v_n,0,0\}$.  Thus, all quantities are returned to their unscaled values and
the solution to equation (\ref{e:solve}) is simply
%
\begin{equation}
  \frac{A(s)}{A(s_0)} = \exp\left( -\int_{s_0}^{s} \frac{h_2}{h_1} ds \right)
\end{equation}
%
where $s_0$ is an arbitrary  reference location.  Using this result in equation
(\ref{e:npdist}), yields to lowest order
%
\begin{equation}
  \bU'(s,n,z,t) = A(s_0) \bhU_0(s,n) \: \exp\left[  \int_{s_0}^s 
                  \left(i k_s(s) - \frac{h_2}{h_1} \right) ds +
                  i k_z z - i \omega t \right] \period
\end{equation}
%
The nonparallel growth-rate and chordwise wavenumber are defined as
%
\begin{equation}
  \sigma = {\rm Re} \left[ \frac{\partial}{\partial s} \ln( U' ) \right] \comma
\end{equation}
%
\begin{equation}
  \alpha = {\rm Im} \left[ \frac{\partial}{\partial s} \ln( U' ) \right]
\end{equation}
%
where $U'$ is any desired disturbance quantity.  Using these expressions, one
obtains
%
\begin{equation} \label{e:NPsigma}
  \sigma = -{\rm Im}( k_s ) - {\rm Re}\left(\frac{h_2}{h_1}\right) +
            {\rm Re}\left( \frac{\partial}{\partial s} \ln (\hat U) \right)
  \comma
\end{equation}
%
\begin{equation} \label{e:NPalpha}
  \alpha = {\rm Re}( k_s ) - {\rm Im}\left(\frac{h_2}{h_1}\right) +
           {\rm Im}\left( \frac{\partial}{\partial s} \ln (\hat U) \right)
  \period
\end{equation}
%
Note that $\hat U$ is any particular disturbance eigenfunction which can be a
function of both $s$ and $n$.  Thus, both $\sigma$ and $\alpha$ depend on the
particular disturbance quantity used and the wall normal coordinate.  The
three terms which contribute to the nonparallel growth-rate in equation
(\ref{e:NPsigma}) are, in order of appearance: the quasi-parallel growth-rate
with curvature, the nonparallel meanflow term, and the eigenfunction growth
term.

%
%.... Solution of nonparallel problem
%
The adjoint eigenfunction must be computed in order to construct the
solvability condition and to determine the nonparallel corrections to the
growth-rate and wavenumber.  Since the eigenvalues for the regular and adjoint
problems are the same, the first-step is to solve the regular eigenvalue
problem (\ie\ equation \ref{e:para}) using the techniques described in
\S\ref{s:QPLST}.  Given the appropriate eigenvalue, the adjoint eigenfunction
is obtained by direct integration of equation (\ref{e:adjoint}) subject to
boundary conditions (\ref{e:aLST-BC-u-wall})--(\ref{e:aLST-BC-infty}) with no
iteration required on the eigenvalue.  The integration is started at a fixed
height above the wall, $n=n_{max}$, and the initial condition at $n_{max}$ is
constructed by solving the uniform-flow version of equation (\ref{e:adjoint}).
Mack \cite{Mack:84a} describes the analytical procedure used to determine the
initial condition for the compressible equations without curvature.  For
simplicity, we formulate the problem numerically by substituting the
uniform-flow conditions at $n_{max}$ into (\ref{e:adjoint}) and solving the
resulting $(8\times 8)$ eigensystem using the LAPACK routine {\tt CGEEV}
\cite{Lapack:92}.  Doing so, one obtains 8 eigensolutions, four that are
damped for large $n$ and four that are unbounded for large $n$.  Consistent
with the farfield boundary conditions (\ref{e:aLST-BC-infty}), the bounded
solutions are used as four linearly-independent initial-conditions at
$n_{max}$.  Each of these initial conditions are integrated to the wall using
standard fourth-order accurate, fixed-step Runge-Kutta integration with the
orthonormalization procedure of Conte \cite{Conte:66} used to prevent
contamination from the undamped solutions.  At the wall, a linear
superposition is constructed that satisfies the wall boundary conditions.
With curvature, the metric terms vary over the entire wall-normal domain so
that the initial condition described above is only approximate.  However, in
practice, $n_{max} \ge 2$ is found to be adequate for cases with and without
curvature.  Once the adjoint is obtained, $h_1$ and $h_2$ are constructed
given the full (nonparallel) meanflow, and the parallel eigenfunctions.  Note
that in determining $h_1$ and $h_2$, chordwise derivatives of both the
meanflow and the regular eigenfunctions are required.  The chordwise
derivatives of the meanflow come directly from the full mean solution, while
the streamwise derivatives of the eigenfunctions are computed by solving the
regular eigensystem over a range in $s$ and forming fourth-order-accurate
finite difference approximations (see \S\ref{s:spacedisc}) for the chordwise
derivatives.

%==============================================================================
%
%  Tables and Figures
%
%==============================================================================
%
\begin{figure}[p]
\centering \epsfxsize=5.0in \epsfbox{figures/ap4/coord.ai}
\caption {Coordinate systems for LST analysis of an infinite-span swept
wing\label{f:LSTcoord}}
\end{figure}
