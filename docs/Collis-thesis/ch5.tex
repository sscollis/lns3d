%%%%%%%%%%%%%%%%%%%%%%%%%%%%%%%%%%%%%%%%%%%%%%%%%%%%%%%%%%%%%%%%%%%%%%%%%%%%%%%
%
%  Chapter 5:  Three-Dimensional Leading-Edge Receptivity
%
%  S. Scott Collis
%
%  Written: 9-5-95
%
%  Revised: 11-8-96
%
%%%%%%%%%%%%%%%%%%%%%%%%%%%%%%%%%%%%%%%%%%%%%%%%%%%%%%%%%%%%%%%%%%%%%%%%%%%%%%%
\chapter{Receptivity on a Swept Parabolic-Cylinder \label{c:pcyl} }

This chapter focuses on the receptivity of the three-dimensional
boundary-layer on a swept parabolic-cylinder subject to surface roughness.
The discussion begins with a description of the geometry followed by the
potential flow solution, mesh generation for Navier--Stokes calculations, and
mean boundary-layer flow characteristics---both for two-dimensional and
three-dimensional boundary layers.  The linear stability characteristics are
then documented for a three-dimensional boundary-layer which serves as a model
of the flow near the leading-edge of a swept-wing in flight.  Finally, forced
receptivity calculations are presented for surface roughness located near the
leading-edge including results from Linearized Navier--Stokes (LNS)
calculations and comparisons to receptivity predictions based on Finite
Reynolds Number Theory (FRNT).

\section{Geometry: the Parabolic Cylinder}

The leading-edge geometry is given by a swept parabolic-cylinder shown in
figure \ref{f:pcyl}.  The reference length-scale is the leading-edge radius,
$L^*=r_n^*$ and the reference velocity is the chordwise component of the
freestream velocity, $u^*_r = u^*_\infty$ where the total magnitude of the
freestream velocity is denoted by $U_\infty$.  Unless otherwise indicated,
free-stream values are used for all other reference quantities.

In many previous experimental and computational investigations of receptivity,
the geometry consists of a flat-plate with some type of smooth leading-edge
attached in front.  The parabolic cylinder has several advantages over these
types of geometries: (1) the surface is continuous and infinitely
differentiable, (2) a conformal map to a rectangular domain exists, (3) a
theoretical investigation \cite{HaKe:96} of two-dimensional, acoustic
receptivity is available, and (4) the pressure gradient is everywhere
favorable and relatively insensitive to Mach number (see \S\ref{s:pcylmean}).

For hybrid flat-plate/leading-edge geometries, there is often a lack of
continuity (\ie\ smoothness) between the plate and leading-edge.  This can
lead to an additional source of receptivity \cite{GoHu:87} which may cloud the
receptivity mechanisms actually under investigation.  Attempts to rectify this
situation have been made \cite{LiReSa:92} including polynomial smoothing at
the junction and the so-called Modified-Super-Ellipse which can be designed to
mimic the pressure distribution of an elliptical leading-edge while retained
higher continuity at the junction.  The parabolic cylinder avoids these
difficulties by providing a continuous geometry (with no arbitrary junction)
that serves as a first-order approximation \cite{vanDyke:64} to the
leading-edge of an airfoil.  Use of this geometry for receptivity
investigations dates back to the work of Murdock \cite{Murdock:81} who
performed two-dimensional, acoustic receptivity computations.  More recently,
Hammerton and Kerschen \cite{HaKe:96} have investigated the same problem using
high Reynolds number asymptotic theory.  Their results show that TS-wave
response is reduced as the nose radius becomes more blunt.  These results have
been qualitatively validated by the recent numerical simulations of Corke \&
Haddad \cite{CoHa:95}.

Here, we investigate receptivity of the three-dimensional boundary-layer on a
swept parabolic-cylinder due to surface roughness and our discussion begins
with the computation of potential flow solutions.

\section{Potential Flow Solutions \label{s:pcylpot} }

Before computing the base-flow solutions, a potential flow solution is
required both as an initial condition to the NS solver and to supply
information for the farfield boundary conditions as described in
\S\ref{ss:meanbc}. The potential flow solution is determined using the method
presented in \S\ref{s:initial}.  The no-penetration condition is used on the
wall and on the symmetry boundary upstream of the leading-edge.  On the inflow
and outflow boundaries, the velocity potential function is set to the value
corresponding to the exact incompressible solution given by van Dyke
\cite{vanDyke:64}
%
\begin{equation}
\tilde\eta = \left( \frac{1}{2} - x + \frac{1}{2} 
                    \sqrt{(2x-1)^2 + 4 y^2} \right)^{1/2},
\end{equation}
\begin{equation}
\tilde\xi  = \frac{y}{\sqrt{2} \tilde\eta} ,
\end{equation}
\begin{equation}
\phi = \tilde\xi^2 - \frac{1}{2} \tilde\eta^2 + \tilde\eta ,
\end{equation} 
%
where $(\tilde\xi,\tilde\eta)$ are the parabolic coordinates and $\phi$ is the
velocity potential.  Use of this solution on the inflow and outflow boundaries
is obviously only an approximation for $\M \ne 0$.  To overcome this, we
typically solve for the potential flow solution on a mesh that is much larger
than the mesh used in our receptivity calculations.

In constructing the mesh for the potential flow solution, an exponential
mapping of the conformal coordinates to the uniform coordinates $(\xi,\eta)$
is used to place the inflow/outflow boundaries far away from the leading-edge.
Given the minimum mesh spacing, $\Delta\tilde\xi_{min}$ and the number of node
points, $N_\xi$, the exponential mapping is given by
%
\begin{equation} \label{e:expmap}
  \tilde\xi = \frac{1}{c_1} \exp\left( c_1 \xi + c_2 \right) - 
              \exp\left( c_2 \right) ,
\end{equation}
%
where $c_2 = \log\left( \Delta\tilde\xi_{min} (N_\xi - 1) \right)$ and 
%
\begin{equation}
  c_1 = (N_\xi - 1) \left[ \log\left( s_\xi \Delta\tilde\xi_{min} 
                                      (N_\xi-1) \right) - c_2 \right] .
\end{equation}
%
The stretching factor, $s_\xi$, is determined from equation (\ref{e:expmap})
by requiring that $\tilde\xi = \tilde\xi_{max}$ at $\xi=1$.  A detailed
derivation of the metrics required by the potential flow solver is provided in
Appendix \ref{a:metrics}.  With this mapping, we have constructed meshes for
potential flow solutions that extend to $x = \pm 10^8$ which prevents the
boundary approximation from corrupting the solution near the body.

To demonstrate this, figure \ref{f:Cp-pot} shows contours of the pressure
coefficient, $C_p$, defined as
%
\begin{equation} \label{e:Cp}
  C_p \equiv \frac{p^* - p_\infty^*}{\frac{1}{2}\rho_\infty^* {u_\infty^*}^2} =
             2 \left( p - \frac{1}{\gamma \M^2} \right) \comma  
\end{equation}
%
for the potential flow over a parabolic cylinder at $\M=0.8$, $\theta=0^\circ$
in the region $x \in [-500,500]$.  Solutions from three domains are shown: $x
\in \pm 10^8$, $x \in \pm 10^7$, and $x \in \pm 10^6$.  The results for the
two larger domains are identical, to graphical accuracy, indicating that the
solution is independent of the farfield boundary.  In figure \ref{f:pg-pot},
the pressure gradient along the wall is plotted for the solutions from the two
largest meshes and, again, the results are in excellent agreement.  It is
important to point out that the domain size required to obtain a converged
solution is quite large.  This is, of course, due to the approximate
boundary-condition imposed on the farfield boundary.  Compared to a closed
body with a similar nose radius (\eg\ an airfoil or circular cylinder) the
parabolic cylinder requires a much larger domain to obtain a mesh independent
potential flow solution given a similar farfield boundary-condition.  This is
due to the fact that domain truncation at some finite location results in a
disregard of the upstream influence associated with the ever expanding
parabola downstream.

To convert the potential flow solutions to a mesh suitable for NS calculations
we use fourth-order accurate B-splines \cite{IMSL:91} to interpolate the
solutions in computational space.  Then using the mapping functions, the
interpolated solution is converted to physical space on the new grid.  After
interpolating to a typical NS mesh, we have re-plotted the pressure contours
from the original and interpolated fields in figure \ref{f:Cp-interp}
demonstrating excellent agreement.

%
%.... This is not quite correct!  I really map 
%     (\xi,\eta) -> (s,n) -> (\tilde\xi,\tilde\eta) -> (x,y)
%
\section{Mesh Generation for Navier--Stokes Solutions \label{s:meshmap} }

The conformal mapping of the parabolic-cylinder to the half-plane is also used
in forming a mesh suitable for Navier--Stokes solutions.  The conformal mapping
can be written as
%
\begin{eqnarray}
  x &=& \tilde\xi^2 + \frac{1 - \tilde\eta^2}{2} \comma \\
  y &=& \sqrt{2} \tilde\xi \tilde\eta \comma
\end{eqnarray}
%
% On the body: y = \sqrt{2} \xi
%
% s = sqrt{2}/2 \xi \sqrt{1+2\xi^2} + 1/2 \log{\sqrt{2}\xi + \sqrt{1+2\xi^2}
%
where $(x,y)$ are the nondimensional physical coordinates and $(\tilde\xi,
\tilde\eta)$ are the nondimensional conformal coordinates.  In this
expression, the body is given by the curve $\tilde\eta=1$.  The conformal
coordinates are further transformed to computational coordinates $(\xi,\eta)$
using hyperbolic mapping functions \cite{MaMoLe:96} designed to cluster points
near the leading-edge and in the boundary layer.  An example of a typical mesh
that results from these transformations is shown in figure~\ref{f:pcylmesh}.

The same mapping function is used for both the $\xi$- and $\eta$ directions.
Considering the $\xi$-direction, the transformation takes the form
%
\begin{equation} \label{e:map}
\tilde\xi = \frac{\tilde\xi_{max}}{\Xi} \left\{
            c_m \xi + \log\left( \frac{\cosh[b(\xi-\xi_c)]}
                                      {\cosh[b(\xi+\xi_c)]} \right) \right\},
\end{equation}
%
where
%
\begin{equation}
  \Xi = c_m + \log\left( \frac{\cosh[b(1-\xi_c)]}
                              {\cosh[b(1+\xi_c)]} \right) ,
\end{equation}
%
and
%
\begin{equation} \label{e:cm}
c_m = \frac{ 2b\tanh(b\xi_c) +
             \frac{(N_\xi-1)\Delta\tilde\xi_{min}}{\tilde\xi_{max}} \log\left(
             \frac{\cosh[b(1-\xi_c)]}{\cosh[b(1+\xi_c)]} \right) } {1 -
             \frac{(N_\xi-1)\Delta\tilde\xi_{min}}{\tilde\xi_{max}}} .
\end{equation}
%
In these equations, $\Delta\tilde\xi_{min}$ is the minimum desired mesh
spacing in $\tilde\xi$, $N_\xi$ is the number of nodes in the $\xi$-direction,
$\tilde\xi_{max}$ is the maximum $\tilde\xi$ node location, $\xi_c$ is the
location of the transition point in computational space, and $b$ controls the
rate of stretching at the transition.  The derivatives of the mapping function
which are required to compute the mesh metrics (see Appendix \ref{a:metrics})
are given by
%
\begin{equation} \label{e:d1map}
\frac{\partial \tilde\xi}{\partial \xi} = \frac{\tilde\xi_{max}}{\Xi}
      \left( c_m + b\tanh[b(\xi-\xi_c)] - b\tanh[b(\xi+\xi_c)] \right)
\end{equation}
%
and
%
\begin{equation} \label{e:d2map}
\frac{\partial^2 \tilde\xi}{\partial \xi^2} = \frac{\tilde\xi_{max}}{\Xi}
      \left( -b^2\tanh^2[b(\xi-\xi_c)] + b^2\tanh^2[b(\xi+\xi_c)] \right)
\end{equation}
%
Figure~\ref{f:map} shows the mapping function and its derivatives for
parameters typical of those used in the wall-tangent direction (see the figure
caption for the parameter values).  The node spacing near the leading-edge is
small to resolve the viscously dominated stagnation region.  The first
derivative of the mapping, which is proportional to the local mesh spacing, is
approximately constant at the leading-edge, indicating that the mesh is nearly
uniform there.  The second derivative at the leading-edge is exactly zero
which ensures that the mesh metrics are symmetric about the $(-x)$-axis.
Downstream of the leading-edge, the mesh transitions smoothly to a second
region of uniform spacing, but with a spacing roughly 32 times larger than
that at the leading-edge.  Note that a uniform spacing in the conformal
coordinate, $\tilde\xi$, still grows like $\sqrt{s}$ in the chordwise
direction so that the mesh spacing in the physical space continues to grow
downstream in a manner consistent with the growing boundary layer.  As
indicated above, the same mapping function is also used in the wall-normal
direction.  For this application, the transition point, $\eta_c$, is set to
$3/4$ instead of $1/4$.  For an appropriate value of $\Delta\eta_{min}$, this
has the effect of shifting approximately half of the mesh points into the
boundary layer.  Using the mapping functions presented above, the metrics
required by the Navier--Stokes solver can be determined using the equations in
Appendix \ref{a:metrics}.

Before closing this section, we point out that the mapping function given by
equation (\ref{e:map}) has several advantages over the algebraic, hyperbolic
tangent \cite{Lin:92}, and exponential \cite{FeStHa:92,Fenno:93} mappings used
previously.  In particular, the first derivative of the mapping function
approaches a constant at the boundaries implying that the mesh is nearly
uniform there.  This ensures that the mesh and its metrics are smooth at the
boundaries and that the mesh has the proper symmetry characteristics for both
solid walls and symmetry boundaries.  With a region of nearly uniform mesh
near the boundaries, second derivative metrics are also better scaled in these
regions.  In our applications, this is important at both the wall and in the
stagnation region on the symmetry-line boundary where viscous effects are
important.
%
%.... Demonstrate the advantages of the current the mapping function?
%

\section{Base Flow Results \label{s:pcylmean}}

Mean-flow results for the parabolic cylinder have been obtained for a variety
of conditions.  For unswept cases at low Mach numbers, the results are
compared to the reference incompressible solutions of Davis \cite{Davis:72}.
After doing so, the influence of compressibility and wing sweep are examined.
%
% Aside:  the adiabatic wall temperature for a flat-plate is given by
%
% \begin{equation}
%   T_{aw} = T_e + r U_e^2/(2 c_p)
% \end{equation}
%
% where $r = \sqrt{\Pr}$ for $0.1 < \Pr < 3.0$.

\subsection{Two-Dimensional Solutions \label{ss:2Dmean} }

Our discussion begins with the mean-flow solution at $\M=0.1$, $\Re=1000$, and
$\Pr=0.7$ using constant fluid-properties with an adiabatic wall-temperature
boundary-condition.  A Mach number of 0.1 was selected to mimic the reference
incompressible solutions, and the appropriateness of this choice will be
assessed below.  The mean flow is computed using the method discussed in
\S\ref{s:base} with the boundary conditions of \S\ref{ss:meanbc}.  Figure
\ref{f:conv} shows the convergence history for a steady-state calculation on a
$(384 \times 127)$ mesh with a potential flow solution as the initial
condition.  Points where a rapid change in the residual occurs correspond to
times during the computation when the timestep was altered to improve
convergence.  The norm of the residual at the end of the computation is
approximately $1.4\times 10^{-7}$ and this is typical of the convergence
achieved for all the mean-flow computations presented here.  The computation
required 4234 iterations using $1.15$~hr of CPU time on a single Cray C-90
processor which corresponds to $20~\mu$-sec/node/iteration.

%
%.... For low mach number flows, global time stepping at very large
%.... CFL is the best road to convergence.
%

%The primary goal of our mean-flow solver is to obtain accurate, clean
%steady-stage solutions using the same spatial differencing technique used for
%disturbance calculations.  By ``clean'' we mean that numerical errors present
%in the solution must be sufficiently small so that they do not adversely
%affect the accuracy of the solution.  This goal is particularly difficult to
%achieve using high-order central difference methods with no inherent numerical
%dissipation, since these schemes are known to possess spurious numerical
%solutions which can be generated by a variety of sources: insufficient
%resolution, approximate boundary conditions, and rapid mesh stretching.  In
%\S\ref{sss:damping}, pressure contours---a very sensitive indicator for low
%Mach number flows---were shown for this mean solution, demonstrating that the
%solutions are free of numerical oscillations using the boundary conditions
%described in \S\ref{ss:meanbc}.

To assess accuracy, the mean solutions are compared to the incompressible
results of Davis \cite{Davis:72}.  Davis solves for the mean flow over an
infinitely long parabolic cylinder by mapping the infinite conformal space to
a finite domain.  In this manner, his method treats the farfield boundaries
exactly.  The equations are cast in streamfunction/vorticity form and
solutions are obtained with a iterative scheme based on a parabolic
approximation.  Davis' results are presented using a scaled wall vorticity
defined as
%
\begin{equation} \label{e:omegaD}
  \omega_D \equiv  -\frac{(2 x + 1)}{\sqrt{2 x \Re}} \omega_w
\end{equation}
%
where $\omega_w$ is the wall vorticity.  This quantity is plotted versus a
viscously-scaled, streamwise, conformal-coordinate defined as
%
\begin{equation} 
  \xi_D \equiv \sqrt{2 x \Re} \period
\end{equation}

By scaling the wall vorticity by the factor in equation (\ref{e:omegaD}), the
first-order variation of the incompressible solution near the leading-edge and
far downstream is accounted for.  Thus $\omega_D$ approaches a constant both
near the leading-edge and far downstream, making it a very sensitive measure
of solution accuracy.  Figure \ref{f:davis} shows a comparison of results
using the current method, at $\M=0.1$ and $\Pr=0.7$, compared to Davis'
incompressible solutions for $\Re=100$ and $1000$.  The current results agree
well with the incompressible solutions for both Reynolds numbers.  This
indicates that both the inflow and outflow boundary conditions are quite
successful even for very thick viscous layers.  That this isn't surprising is
demonstrated by Davis' figure~3 which shows that for $\xi_d > 50$ the
parabolized Navier--Stokes equations are a good approximation to the full
Navier--Stokes equations (also see figure \ref{f:parabolic} discussed below).

% What is somewhat surprising is the fact
% that the inflow boundary condition is also able to adjust to the rather large
% displacement thickness of the low Reynolds number boundary layers.  However,
% even when $\Re=1$ the inflow boundary is at least $30 \delta_1$ above the 
% wall so that it is always well away from the boundary layer.

%
%.... The vorticity function at the LE goes as g = 1.146 - 0.248 M^2
%
%	M^2	g(0)
%=========================
%	0.01	1.141805
%	0.25	1.083930
%	0.64	0.988099

A close examination of the vorticity function near the leading-edge shows that
the current results slightly under-predict the incompressible results.  To
determine if this difference is due to compressibility, mean flow solutions
have been obtained over a range of subsonic Mach numbers at $\Re=1000$,
$\Pr=0.7$, again using constant fluid-properties and an adiabatic wall
temperature condition.  The use of constant fluid properties at high subsonic
Mach numbers is obviously an approximation, but doing so is not expected to
have a qualitative impact on the present results.  The scaled wall vorticity
from these solutions is shown in figure \ref{f:mach} which gives results for
Mach numbers of 0.1, 0.5 and 0.8 along with Davis' incompressible solution.
As the Mach number is decreased, $\omega_D$ near the leading-edge increases,
approaching the incompressible solution.  In figure \ref{f:attach}, $\omega_D$
at the attachment-line is plotted as a function of Mach number squared.  The
relationship is linear and extrapolating to zero Mach number gives a value
1.146, which is in very good agreement with Davis' value of 1.15.

Returning to figure \ref{f:mach}, we see that $\omega_D$ also approaches the
incompressible solution over the entire domain as Mach number is lowered.
But, near the end of the computational domain, there is a slight difference
between the current solution at $M=0.1$ and Davis' incompressible solution.
However, the symbols plotted corresponding to Davis' solution were extracted
from a rather small and crowded graph in \cite{Davis:72} so that there could
be some error in the deduced incompressible values.  This is particularly the
case for high $\xi_D$ where many solutions overlap on Davis' plot.  However,
Davis demonstrated that beyond $\xi_D \approx 50$ a fully parabolic
approximation to the governing equations agrees very well with the full
Navier--Stokes solutions.  In figure \ref{f:parabolic} the incompressible
parabolic solution is compared to our $\M=0.1$ solution.  The parabolic result
was obtained using a method identical to that presented by Davis.  Our
solution compares very well with the parabolic solution downstream of $\xi_D =
50$, but, as expected, the parabolic solution near the leading-edge is poor
\cite{Davis:72}.

As the final result for two-dimensional boundary layers, figure
\ref{f:pg_mach} shows the evolution of the wall-tangent pressure gradient,
evaluated at the wall, for a range of Mach numbers.  As Mach number is
increased, the favorable pressure gradient, which exists over the entire
surface, increases slightly over most of the chordwise extent.  This is in
contrast to the results presented in \S\ref{s:recep} for the super-ellipse
which showed a dramatic increase in the magnitude and length of the region of
adverse pressure gradient, associated with the leading-edge/plate junction, as
Mach number was increased.  As alluded to earlier, the more benign influence
of Mach number on the pressure gradient (and therefore stability
characteristics) for the parabolic cylinder makes it a more useful geometry
for the study of receptivity at high subsonic Mach numbers.

To summarize the two-dimensional results, it is concluded that the current
mean-flow solutions converge to the reference incompressible solution of Davis
as Mach number is reduced.  Furthermore, at $\M=0.1$ the current solution is
in good agreement with the incompressible solution over the full length of the
parabolic-cylinder, and, as Mach number is increased, the vorticity at the
leading-edge varies like $\omega_D(0) = 1.146 - 0.248 \M^{2}$.

\subsection{Three-Dimensional Solutions \label{ss:3Dmean} }

To create a three-dimensional boundary layer, the parabolic-cylinder is swept
with respect to the incoming uniform flow.  The conditions are selected to
generate a flow which, in the vicinity of the leading-edge, roughly models
that of a subsonic transport-aircraft in cruise.  For this purpose, we use a
sweep-angle of $\theta = 35^\circ$ with $\M_\infty = 0.977$ and $\Pr = 1$.
Using the chordwise component of the freestream velocity as the reference
velocity gives a Mach number of $\M = 0.8$ and Reynolds number of $\Re =
1\times 10^5$, where the nose radius is the reference length-scale.  Assuming
that the nose radius of a typical airfoil section is approximately 1\% chord
\cite{AbDo:59}, these conditions correspond to a chord Reynolds number of
approximately $1\times 10^7$, which is in the realm of flight Reynolds
numbers.

Unlike the two-dimensional results presented above, an isothermal
wall-temp\-er\-ature boundary-condition of the form $T_w = T_0$ is now used,
where
%
\begin{equation}
  T_0 = 1 + \frac{\gamma-1}{2} \M^2 ( 1 + \tan^2(\theta) ) 
\end{equation}
%
is the freestream stagnation temperature.  For $\Pr = 1$, this corresponds to
the adiabatic wall temperature for a flat-plate boundary layer.  Since the
boundary layer flow over a parabolic cylinder approaches a flat-plat boundary
layer downstream, the wall temperature boundary condition asymptotically
approaches an adiabatic condition downstream.  The use of the isothermal
boundary-condition greatly simplifies the wall-roughness boundary condition
(see below), while using $\Pr=1$ allows more direct comparisons to available
similarity solutions (see Appendix \ref{a:FSC}).  As before, the meanflow
is computed using constant fluid properties since variations in fluid
properties are not expected to cause qualitative differences in receptivity
results.
%
% The computational parameters are summarized in table~\ref{t:swept}.
%

To extract boundary layer profiles and non-dimensional parameters
characteristic of its state from the Navier--Stokes solution, the original
solution on the conformal mesh, described in \S\ref{s:meshmap}, is
interpolated onto a body-fitted coordinate system.  Recall that the
body-fitted coordinate system is denoted by $(s,n,z)$ where $s$ is the
arc-length in the chordwise direction, $n$ is the distance normal to the body,
and $z$ is the spanwise direction.  The body-fitted mesh is constructed by
forming level-sets off the parabolic body.  To cluster points near the
leading-edge and in the viscous layer, the mesh mapping presented in equation
(\ref{e:map}) is used in both the body normal and tangent directions.  Since
the body-fitted mesh is used only for analysis, the top boundary can be placed
much closer to the wall to reduce the number of points in the far-field.  To
interpolate the solution from the conformal to body-fitted mesh, fourth-order
accurate B-spline interpolation is used in computational space.  Each point,
in the body-fitted mesh, is converted using the mapping of \S\ref{s:meshmap}
to the uniform computational space where the IMSL\ B-spline routines
\cite{IMSL:91} are used to determine the interpolated value for each of the
primitive variables.  This interpolation procedure, which is similar to that
used to interpolate the potential flow solution, has been found to yield
excellent results and is also used to convert solutions to different grids
when performing mesh resolution studies and when computing local quantities
such as $\delta_{0.999}$ and local profile maxima.

Given the meanflow solution on a body-fitted mesh, the first step in analyzing
the boundary layer is to determine the edge conditions.  Conceptually, the
edge conditions are given by the local inviscid-flow outside of the boundary
layer.  Figure \ref{f:3dbl} is a schematic of a three-dimensional boundary
layer near a swept leading-edge which aids in the definition of the various
boundary layer parameters.  In this study, the boundary-layer edge is defined
as the point where the spanwise, $w$, component of velocity reaches $99.9\%$
of its freestream value.  Since the wing has infinite span, the $w$-velocity
profile increases monotonically from zero at the wall to its freestream value,
$\tan(\theta)$.  This is in contrast to the more traditionally used $\vs$
velocity.  Recall that $\vs$ is the velocity locally tangent to the body at a
given station while $\vn$ is the body-normal velocity.  These velocity
components are related to the global Cartesian velocity components by
%
\begin{equation}
  \vs = n_2 u - n_1 v \comma
\end{equation}
%
\begin{equation}
  \vn = n_1 u + n_2 v \comma
\end{equation}
%
where the outward unit-normal to the body is ${\bf n} = \{n_1,n_2,0\}^{\rm
T}$.  Since the flow is accelerated in the chordwise direction, the $v_s$
velocity over-accelerates near the leading-edge making an unambiguous
definition of the boundary-layer edge difficult.  Although the particular
choice of edge-conditions does have an influence on the reported
boundary-layer parameters, the effects are minor and the current approach
yields adequate results.  Throughout the following discussion, the reader may
wish to consult table \ref{t:results} which itemizes the boundary-layer
edge-conditions and parameters for selected stations near the leading-edge.

The magnitude of the edge-velocity, projected to the plane tangent to the
body, is given by $U_e = \sqrt{ \vs^2_e + w_e^2}$.  In practice, $U_e$ is
referred to as the total edge-velocity since $v_n$ is negligible except in the
immediate vicinity of the attachment-line.  The edge-velocity makes an angle
$\theta_e$ with respect to the $x$-axis and this angle is called the local
sweep-angle.  When presenting boundary-layer profiles, it is useful to convert
the velocity to a coordinate system that is locally aligned with the
edge-velocity. Velocity components in this ``streamwise'' coordinate system
are denoted by $(u_s,v_n,w_s)$, where $u_s$ is the local, streamwise velocity;
$v_n$ is the body-normal velocity; and $w_s$ is the crossflow velocity.  These
components are related to the velocities in the boundary-fitted coordinate
system by
%
\begin{equation}
  u_s = \cos(\theta_e) \vs + \sin(\theta_e) w \comma
\end{equation}
%
\begin{equation}
  w_s = -\sin(\theta_e) \vs + \cos(\theta_e) w \period
\end{equation}

With these definitions we can begin to examine the characteristics of this
three-dimensional boundary-layer.  Figures \ref{f:pro1}--\ref{f:pro3} show
streamwise, $u_s$, and crossflow, $w_s$, velocity profiles at three stations
near the leading-edge where the profiles have been nondimensionalized by
$U_e$.  The streamwise profile is ``Blasius-like'' while the crossflow profile
is inflectional with a maximum, crossflow velocity of $\approx 11\%$ of the
local edge velocity at $s=0.802$. The inflectional crossflow profile gives
rise to the inviscid, crossflow instability.

For this flow, the standard attachment-line length-scale $\delta^*_l \equiv
\sqrt{\nu_e^* / (\partial U^*_e / \partial s^*)}$ is $0.00353 \, r^*_n$ and
this yields an attachment-line Reynolds number $\bar R \equiv \delta^*_l
w^*_\infty / \nu^*_\infty$ of 247.  This value of $\bar R$ is less than the
critical value of $\bar R_{crit} = 583$, which ensures that the flow is
linearly stable along the attachment-line \cite{Spalart:89}.

Figure \ref{f:r4.wmax} shows the chordwise evolution of the maximum crossflow
component of velocity normalized relative the local edge-velocity.  The
crossflow component is everywhere negative and reaches its greatest magnitude
of $-0.107$ at $s=0.802$.  At the point of maximum crossflow, the crossflow
Reynolds number
%
\begin{equation} \label{e:chi}
  \chi \equiv \delta^*_{0.999} {w_s^*}_{max} / \nu^*
\end{equation}
%
is 161.  The results of Poll \cite{Poll:85} for a swept circular-cylinder
indicate that the critical crossflow Reynolds number is 125 and our value of
161 suggests that significant crossflow instability will be present.  Table
\ref{t:results} shows the crossflow Reynolds number reaches a maximum of 256
at $s \approx 20.3$.

The variation of the local sweep-angle, $\theta_e$, of the edge-velocity is
shown in figure \ref{f:r4.thetae}.  At the point of maximum crossflow, the
local sweep-angle is $54.5^\circ$ relative to the chord compared to a value of
$90^\circ$ at the attachment-line.  Downstream, the local sweep-angle
asymptotes to the global sweep-angle $\theta=35^\circ$ as the chordwise
pressure gradient decays.  The pressure gradient along the parabola is
documented using the Hartree pressure-gradient parameter which is defined as
%
\begin{equation} \label{e:betah}
  \beta_h = \frac{2 m}{m+1} 
\end{equation}
%
where
%
\begin{equation}
  m = \frac{s}{u_s} \frac{\partial u_s}{\partial s} \period
\end{equation}
%
Figure \ref{f:r4.betah} shows the evolution of $\beta_h$ along the length of
the plate.  At the attachment line, $\beta_h = 1$ corresponding to a
stagnation region while downstream of the leading edge, $\beta_h$ decays
monotonically to zero.  At the point of maximum crossflow, $\beta_h = 0.765$
and by 30 nose radii downstream $\beta_h$ is reduced to $0.042$ indicating
that the flow is approaching a zero-pressure gradient boundary layer.
%
% For M=0.8, R=1e5, th=35,  M_\infty = 0.97661967
%
% \partial U_e / \partial s = 0.80276598
%
% \delta_l = 0.0035294377,     \bar \Re = 247.13389
%
% Spalart says that at the attachment line, \theta_z = 0.404
%
% I get \delta_2 / \delta_l = 0.33774
%
% \delta_2 = 0.00119
%
% For M=0.73, R=1e5, th=35,  M_\infty = 0.8911655
%
% \partial U_e / \partial s = 0.83242
%
% \delta_l = 0.003466,     \bar \Re = 242.7
%
% \delta_2 / \delta_l = 0.3494021
%
% \delta_2 = 0.00121
%
% For M=0.5, R=1000, th=45,  M_\infty = 0.70710678
%
% \partial U_e / \partial s = 0.820
%
% \delta_l = 0.03492,     \bar \Re = 34.9
%
% \delta_2 / \delta_l = 0.3695
%
% If you plot \delta_2 / \delta_l vs (total Mach)^2 you get a linear
% relationship.  In fact, if you extrapolate back against M^2 you get
% \delta_2 / \delta_l = 0.407 at M=0

Figure \ref{f:r4.thick} shows the displacement and momentum thickness
evolution for the local streamwise velocity profile.  The displacement
thickness is defined by
%
\begin{equation}
  \delta_1 \equiv \int_0^{\delta_{0.999}} \left( 1 - 
  \frac{ \rho u_s }{\rho_e U_e } \right) \, dn
\end{equation}
%
while the momentum thickness is given by
%
\begin{equation}
  \delta_2 \equiv \int_0^{\delta_{0.999}} \frac{ \rho u_s }{\rho_e U_e} 
                           \left( 1 - \frac{u_s}{U_e} \right) \, dn \period
\end{equation}
%
For incompressible flow at the attachment-line, Spalart gives a value of
$\delta_2 / \delta_l $ = 0.404 \cite{Spalart:89}.  From the current data, we
get a value of $0.337$ which suggests that this quantity decreases with Mach
number.  To verify this, a separate calculation at $\M=0.73$, with all other
conditions the same, was performed which gives a value of $\delta_2/\delta_l =
0.349$.  Linearly extrapolating these results against the {\it freestream}
Mach number squared gives a value of 0.407 at $\M_\infty = 0$ which is in good
agreement with the incompressible value.  The evolution of the boundary-layer
shape-factor, defined as $H = \delta_2 / \delta_1$, is shown in figure
\ref{f:r4.shape}.  At the attachment-line $H = 0.368$ and quickly increases to
a maximum of $H = 0.377$ at $s = 0.63$ before beginning a slow decay,
eventually asymptoting to $H = 0.317$.

%The final base-flow result is given in figure \ref{f:r4.epsi} which presents
%$\epsilon_i$, the flow angle at the inflection point of the crossflow velocity
%profile relative to the local sweep-angle.  At the attachment line
%$\epsilon_i$ is zero by definition.  Downstream, $\epsilon_i$ decreases and
%reaches a minimum value of $-4.72^\circ$ at the point of maximum crossflow.
%Further downstream, $\epsilon_i$ increases and eventually asymptotes to zero.
%The inflection point flow-angle is approximately related to the wave-angle of
%the most unstable crossflow mode as described by Mack \cite{Mack:84a}, and, as
%such, provides a useful means of estimating the crossflow vortex wavenumbers
%for stability analysis.

\section{Linear Stability Analysis \label{s:LSTpcyl} }

Before performing receptivity calculations, it is necessary to determine the
stability characteristics of the mean flow.  This is accomplished by using the
spatial, Linear Stability Theory (LST) solver described in Appendix
\ref{a:LST} in conjunction with several LNS stability calculations.  Recent
investigations \cite{MaMa:94,MaBa:93} have shown that the growth of stationary
crossflow vortices is significantly influenced by nonparallel and curvature
effects. The boundary-layer over a parabolic cylinder is an ideal flow for
studying these effects since both curvature and nonparallelism occur naturally
and are attenuated downstream.  In this section, we examine the influence of
curvature and nonparallel flow on the stability characteristics of stationary
crossflow vortices, since these effects are also likely to impact the
receptivity characteristics.

Our discussion begins by considering the boundary-layer profiles at the
location of maximum crossflow velocity, $s=0.802$.  These profiles are shown
in figure \ref{f:profile} and are extracted using a body-fitted mesh as
described in \S\ref{s:pcylmean}.  Since we are interested in comparing the LST
results with LNS calculations, we break with the traditional
nondimensionalization of LST and instead retain the global
nondimensionalization used in the LNS solutions. In particular, the reference
velocity is the chordwise component of the freestream velocity and the
reference length is the leading-edge nose radius. Use of the global
nondimensionalization makes comparison of results at different stations more
convenient.  However, our results can be easily converted to the local
nondimensionalization based on the displacement thickness and edge-velocity
using the data supplied in Table \ref{t:results}.

Growth-rate predictions from quasi-parallel LST at $s=0.802$ are given in
figure \ref{f:sigma} for zero-frequency crossflow vortices.  The range of
amplified spanwise wavenumbers is from roughly 32 to 160 and the maximum
predicted growth-rate at this station occurs for $\beta = 90$.  The predicted
chordwise wavenumber for stationary modes is shown in figure \ref{f:alpha}
where it is seen that $\alpha$ varies almost linearly with $\beta$ in the
unstable range. In figure \ref{f:psi}, the wave-angle, $\psi$, with respect to
the body-fixed coordinate system is shown as a function of the spanwise
wavenumber, demonstrating that across the unstable range the wave-angle is
nearly fixed at $140^\circ$.  This means that the axis of the crossflow
vortices is at approximately $50^\circ$ and remains within $6^\circ$ of the
local streamwise direction.

Considering the spanwise wavenumber, $\beta = 35$, figure \ref{f:QPNCsigma}
shows the variation in the LST growth-rate and wavenumber as a function of the
distance along the plate.  The growth-rate of this crossflow mode quickly
increases from the first neutral-point at $s = 0.65$ reaching a maximum at $s
= 2.05$ before slowly decaying to zero with the second neutral-point located
at $s = 27.7$.  This extended region (about 27 nose radii) of crossflow
amplification is due to the slowly decaying favorable pressure gradient (see
figure \ref{f:r4.betah}) of the meanflow.  We note that typical airfoil
sections have a pressure minimum on the upper surface followed by an adverse
pressure gradient that destabilizes TS waves.  The relatively simpler
situation here, combined with the extended region of crossflow amplification,
makes the parabolic-cylinder an ideal geometry for the study of crossflow
vortex receptivity.  For a discussion of combined crossflow/TS instability
modes for a flow with both favorable and adverse pressure gradients, the
reader is referred to the recent computations by Spalart \etal\
\cite{SpCrNg:94}.

To verify the accuracy of the quasi-parallel LST prediction, several LNS
calculations have been performed in order to determine the actual stability
characteristics including nonparallel and curvature effects.  These stability
calculations are performed using the techniques introduced in
\S\ref{s:CFtest}.  To decrease the cost of the calculations, conformal grids
are constructed that truncate the extensive inviscid region used in the
meanflow calculations.  The steady-state meanflow is then interpolated using
fourth-order-accurate B-splines to the new mesh.
%
% An example of a mesh used for stability calculations is shown in figure 
% \ref{f:stabmesh}.  
%
Consistent with the findings of \S\ref{s:CFtest}, the chordwise mesh spacing
is chosen to ensure at least 20 node points per wavelength based on the
chordwise wavenumber predicted by LST.  The LST eigenfunction is forced on the
inflow boundary using the same technique discussed in \S\ref{ss:CFspace}.
This eigenfunction is only an approximation, since curvature and
nonparallelism are not accounted for, and we therefore expect a small
transient before a dominant mode is established.  The nonparallel meanflow
introduces another complication since the growth-rate is now a function of
both the quantity measured and the wall normal location, $n$.  In general, the
growth-rate for any linear disturbance quantity, $\hat f$, is given by
%
\begin{equation}
  \sigma = {\rm Re} \left[ \frac{d \ln( \hat f(s,n) )}{ds} \right]
\end{equation}
%
while the wavenumber is determined from
%
\begin{equation}
  \alpha = {\rm Im} \left[ \frac{d \ln( \hat f(s,n) )}{ds} \right] \period
\end{equation}
%
To remove the $n$ dependence from the growth-rate, experimental investigations
often define the growth-rate based on the (first) maximum, in $n$, of the
chordwise velocity, $\vs$.  Similar quantities can be computed based on other
disturbance variables.  Another way to remove the variation in the wall-normal
location is to base the growth-rate on the integrated disturbance kinetic
energy, defined as
%
\begin{equation}
  E'_k \equiv \int_0^{n_{max}} \frac{1}{2} \left( |\hat u|^2 + 
                                     |\hat v|^2 + |\hat w|^2 \right) dn
\end{equation}
%
where $n_{max}$ is sufficiently large so that additional contributions to the
integral are negligible.  Here, $n_{max}$ is taken to be the the maximum
wall-normal location in the computational domain.  The growth-rate, based on
$E'_k$, is obtained from
%
\begin{equation}
  \sigma = \frac{1}{2} \frac{d \ln(E'_k) }{ds} \period
\end{equation}

The first LNS calculation is performed for a domain starting at $s=5.6$ and
extending to $s=25.8$.  This domain covers the later portion of the unstable
region, and, since it is located far from the leading-edge, offers the best
chance that curvature and nonparallel effects will be small.  Figure
\ref{f:x5gr} shows the growth-rate extracted from the calculation, based on
$E'_k$, compared to the quasi-parallel LST prediction.  As expected, there is
a slight transient in the LNS solution since the function forced on the inflow
is only an approximation to the actual eigenfunction.  After approximately two
nose radii downstream, the transient is no longer noticeable and a single
instability wave dominates the flow.  Interestingly, the LNS and LST
growth-rates cross-over at $s = 13$ and the errors in the LST prediction
increase upstream and downstream of this point.  This implies that there may
be competing errors in the quasi-parallel LST prediction.

To identify the source of this discrepancy, we have incorporated both surface
curvature and nonparallel effects in the LST predictions.  Surface curvature
is accounted for by solving the stability equations in a general orthogonal
coordinate system that is coincident with a body-fitted mesh.  Nonparallel
effects are included through a perturbation approach where the meanflow,
chordwise wavenumber, and eigenfunction are allowed to vary slowly in the
chordwise direction.  The detailed implementation of curvature and nonparallel
effects is discussed in Appendix \ref{a:LST}.

Figure \ref{f:QPWC} compares the growth-rate for $\beta=35$ from the LNS
calculation with two predictions from LST.  These two LST predictions are
denoted by QPNC, which stands for Quasi-Parallel with No Curvature, and QPWC,
which is short for Quasi-Parallel With Curvature.  The growth-rate for the
QPWC case shows that curvature is stabilizing which is consistent with recent
analysis of the flow over a circular cylinder \cite{MaMa:94}.  However the
stabilization due to curvature leads to a greater discrepancy between LST and
LNS, as compared to QPNC.  As expected, the effect of curvature increases
upstream as the surface curvature increases.  Figure \ref{f:radius} shows the
normalized curvature $K \delta_1$,
%
% K = K^* \delta_1^* = ( 2 x + 1 )^{-3/2} \delta_1
%
where $K$ is the surface curvature, which demonstrates that the curvature
decays as $1/s$ downstream of $s=1$.  Upstream of $s=1$, the normalized
curvature plateaus at a value of $K\delta_1 = 3.23\times 10^{-3}$.  This
behavior is in contrast to results using a circular cylinder where the
normalized curvature increases downstream of the leading-edge since the radius
of curvature for that geometry remains constant.  However, based on the
results from QPNC, even a very small normalized curvature on the order of
$10^{-4}$ still has a significant effect on the predicted growth-rates.  This
result is qualitatively consistent with recent PSE analysis \cite{HaRe:96} of
the crossflow experiments performed by Saric's group \cite{ReSaCaCh:96} which
indicate that curvature continues to influence the growth-rate of stationary
CF vortices over the entire upper surface of an airfoil up to transition.

The influence of nonparallel effects are shown in figure \ref{f:NPWC} which
includes the nonparallel correction with curvature (NPWC) in addition to the
previous results.  It is obvious from the figure that including both
nonparallel and curvature effects leads to growth-rate predictions that are in
excellent agreement with the LNS solution.  This level of agreement gives
further confidence that both the LNS and LST implementations are free of
errors.  In general, nonparallelism is seen to destabilize the flow, returning
the growth-rate to a level similar to quasi-parallel theory over most of the
plate.  Near the leading-edge, the flow becomes highly nonparallel causing a
qualitative change in the shape of the NPWC growth-rate distribution when
compared to the quasi-parallel results.  This difference is in the form of a
local maxima in the NPWC growth-rate which occurs at $s = 0.7$.

Based on the analysis in Appendix \ref{a:LST}, there are three contributions
to the nonparallel growth-rate: the quasi-parallel growth-rate with curvature
(QPWC), the eigenfunction distortion term, and the nonparallel meanflow term.
These contributions are shown in figure \ref{f:NPterms} both in an expanded
view near the leading-edge and over the entire unstable region.  Downstream of
$13 r^*_n$, the nonparallel meanflow term becomes negligible in comparison to
the other contributions indicating that the meanflow is, for all practical
purposes, parallel.  However, the eigenfunction distortion term remains
significant beyond the second neutral-point resulting in a downstream shift of
the neutral-point (relative to QPWC) to $s = 30.54$.  As shown in
Appendix~\ref{a:LST}, the eigenfunction distortion term can be computed
directly from the quasi-parallel eigenfunctions so that it can be accounted
for without resorting to the full perturbation approach.  However, the small
influence of the nonparallel meanflow term at the second neutral-point is not
a general result, and conditions (either $\beta$ or $\Re$) could be selected
such that this is not the case.  The second frame of figure \ref{f:NPterms}
shows the contributions to the nonparallel growth-rate near the leading-edge.
Upstream of $s \approx 5$, the eigenfunction distortion term begins to
increase slowly to a maximum at $s=1$; decaying further upstream.  Upstream of
$s \approx 1$, (which, coincidently, is the approximate location of the first
neutral-point for QPWC) the nonparallel meanflow term undergoes a sudden
increase, quickly dominating the other terms.  The increase in both
nonparallel terms causes the local maxima in the total NPWC growth-rate and
results in an upstream shift (relative to QP theory) in the first
neutral-point to $s = 0.471$.  Since the nonparallel correction is based on a
perturbation expansion, the results near the leading-edge may be inaccurate.
This is addressed below by comparison to LNS calculations.

But first, the effect of curvature and nonparallelism on the chordwise
wavenumber is shown in figure \ref{f:NPwavenumber} (note that a log scale is
used in $s$ to more clearly show the leading-edge behavior).  The NPWC
wavenumber is computed based on the maximum in the wall-normal direction of
the $|\hat\vs|$ velocity.  As seen in the figure, both QPNC and QPWC give
almost identical results, indicating that curvature has little influence on
the wavenumber.  However, NPWC shows that nonparallel effects tend to decrease
the wavenumber magnitude---when based on $|\hat\vs|_{max}$.  This effect is
most pronounced near the leading-edge, but it also occurs to a lesser degree
over the entire unstable range.  This is demonstrated in figure
\ref{f:NPwaveLNS} which shows a comparison of the wavenumber extracted from
LNS (also based on $|\hat\vs|_{max}$) with the QPWC and NPWC predictions.
After a brief transient in the LNS solution, the NPWC wavenumber is in good
agreement with LNS while QPWC slightly over-predicts the wavenumber magnitude.

As indicated above, nonparallel effects in the LST are accounted for using a
perturbation expansion so that the results may become inaccurate near the
leading-edge where the meanflow is highly nonparallel.  To evaluate this, a
second LNS calculation is performed on a domain from $s = 0.2$ to $s=5.25$ in
which the QPWC eigenfunction is forced on the inflow.  The measured
growth-rate from this calculation is shown in figure \ref{f:LNSle} compared to
the NPWC prediction where both are based on $E'_k$.  Clearly, the NPWC results
do not agree with the LNS solution near the leading-edge indicating that the
perturbation approach is inaccurate under these conditions.  However, by $s=3$
the two approaches are in excellent agreement.  The failure of the nonparallel
perturbation approach leads to both qualitative and quantitative errors in the
predicted growth-rate near the leading-edge.  The peak growth-rate is grossly
under-predicted and there is the appearance of a spurious local maxima.  In
addition, the first neutral point is also in error with a predicted value of
$0.471$ compared to the actual value of $0.395$.

The failure of the perturbation approach, at first, appears at odds with
recent stability analysis which have shown the nonparallel perturbation
results to be in good agreement with PSE for a swept circular cylinder
\cite{MaMa:94}.  Likewise, PSE has been demonstrated to compare well with LNS
calculations \cite{MaLiCh:94} for the swept Hiemenz flow.  Compared to the
current results, the stability characteristics in these studies are, by
necessity, limited to the immediate vicinity of the attachment-line.  For this
reason, the spanwise wavenumbers used were large compared to the present value
of $\beta = 35$ in order to limit the chordwise extent of the unstable region.
In particular, Masad \& Malik \cite{MaMa:94}, in their investigation of the
stability characteristics of the boundary layer on a swept circular-cylinder
under conditions similar to those studied here ($\Re=1.675\times 10^5$,
$\Lambda=60^\circ$) used $\beta^* r^*_n = 150$ compared to the current value
of $35$.  For large $\beta$, the downstream extent of the region of
instability is reduced and nonparallel effects are weakened since the
chordwise instability wavelength is reduced.  To demonstrate this effect,
Figure \ref{f:Beta100} compares the predicted growth-rate from NPWC and
LNS for $\beta = 100$ computed on a domain from $s=0.32$ to $s=3.18$.  The
agreement of the perturbation theory with LNS is much better for this higher
spanwise wavenumber.  There is no indication of a local maxima and both the
neutral point and the maximum growth-rate are well predicted.  Figure
\ref{f:Beta100NPterms} shows the contributions to the nonparallel growth-rate
and comparing these results to figure \ref{f:NPterms} shows that both the
meanflow and eigenfunction corrections are considerably more benign for the
higher spanwise wavenumber.  For $\beta=100$ strong nonparallel effects occur
considerably upstream of the first neutral point so that the perturbation
approach is adequate in the unstable region.

To determine the stability characteristics over the entire unstable region in
$s$ with both curvature and nonparallel effects accounted for, a series of six
LNS calculations, four runs for $\beta=35$ and two runs for
$\beta=100$, have been performed.  The runs for each wavenumber consist of
partially overlapping domains each with a locally optimized mesh to provide at
least 20 nodes per crossflow wavelength in the chordwise direction.  Using
this ``multidomain'' approach allows for an efficient account of the wide
variance in the chordwise wavenumber over the full unstable region (recall
figure \ref{f:NPwavenumber}) while limiting each individual domain to a
reasonable size, $\le (1024 \times 63)$ node points.  On the inflow to each
domain, the local eigenfunction from LST (with curvature) is forced which
causes a mild inflow transient before the actual instability wave is
established.  On the outflow boundary a sponge term is used which has been
shown to damp stationary crossflow vortices with only minor upstream influence
(see \S\ref{ss:CFspace}).  The use of overlapping domains allows us to
evaluate the agreement in the growth-rate predictions to ensure accuracy.  An
example of the typical level of agreement is shown in figure
\ref{f:multidomain} for two overlapping domains with $\beta=100$.  In this
figure, the complete inflow transient is shown for each domain, however, the
outflow sponge regions have been partially truncated.  The agreement in the
overlap region is excellent downstream of the second inflow transient and a
similar level of agreement is achieved on all the overlap regions.  After
truncating the inflow and outflow regions, the growth-rate and wavenumber
results from each run are combined and shown in figure \ref{f:63} for the
growth-rate, and figure \ref{f:62} for the wavenumber.  These figures include
results for both spanwise wavenumbers along with the corresponding NPWC
predictions.  For $\beta=100$, both the wavenumber and growth-rate from NPWC
are in good agreement with LNS.  The only noticeable discrepancy occurs in the
growth-rate for $s>2$ where the NPWC prediction begins to deviate slightly
from the LNS result.  As noted above, both the NPWC growth-rate and wavenumber
for $\beta=35$ are in poor agreement with LNS near the leading-edge ($s<3$)
due to a breakdown of the perturbation approach used to compute the NPWC
results.  For $s>3$, the wavenumber and growth-rate from NPWC are in good
agreement with LNS; although, similar to the $\beta=100$ case, the NPWC
growth-rate deviates slightly from the LNS for large $s$.  Thus, for both
spanwise wavenumbers considered, the error in the NPWC growth-rate increases
near the second neutral point, and this error takes the form of a slight
under-prediction of the growth-rate.

The $N$-factor is extracted from the LNS solutions and is given in figure
\ref{f:N-factor} which shows results for both spanwise wavenumbers.  Recall
that the $N$-factor is defined as $N = \ln({A(s)/A_0})$ where $A$ is the
amplitude at a given station $s$, and $A_0$ is the amplitude at a reference
location, $s_0$.  As is typically done, the reference location used for figure
\ref{f:N-factor} is the first neutral point so that $N$ is defined to be zero
there.  As expected, the higher growth-rate for the $\beta=100$ case leads to
a quicker initial rise in $N$-factor; however, the very long region of
unstable flow for $\beta=35$ makes this wavenumber considerably more dangerous,
with $N > 10$ beyond $s = 10$.

The $N$-factors computed from the LNS solutions are compared to those computed
from the NPWC predictions in figure \ref{f:N-35-factor} for $\beta=35$ and
figure \ref{f:N-100-factor} for $\beta=100$.  The error in the NPWC
growth-rate near the leading-edge for $\beta=35$ leads to significant errors
($\approx 6\%$) in the $N$-factor predicted downstream.  For $\beta=100$, the
$N$-factor from NPWC is in much better agreement with the LNS value.  However,
the slight errors in the NPWC growth-rate lead to an accumulation of error in
the $N$-factor.  At the second neutral point, the error in the NPWC $N$-factor
is $2.5\%$ for $\beta=100$.  In the following section the $N$-factor is used
to extrapolate the measured amplitude from receptivity calculations back to
the source location.  Given the error in the NPWC stability results, we use
the ``exact'' LNS $N$-factors to perform this extrapolation.

Before concluding this section, we point out that the linear stability results
presented here can be used to develop an approximate criteria for the failure
of the nonparallel linear stability theory.  Toward this goal, we introduce
the quantity
%
\begin{equation}
  \delta_g = \frac{\lambda_s}{\delta_1} \left(\frac{\partial\delta_1} 
                                                   {\partial s}\right)
\end{equation}
%
which can be interpreted as the relative growth of the boundary layer over one
wavelength of the instability wave.  This quantity is plotted as a function of
arclength along the parabolic cylinder in figure \ref{f:NPmeasure}.  From
figure \ref{f:63}a for $\beta=35$, the NPWC growth-rate is seen to depart
significantly from the LNS growth-rate for $s < 3$.  With reference to figure
\ref{f:NPmeasure}, we see that at $s=3$ the relative boundary layer growth is
$\delta_g = 0.22 \%$ for $\beta=35$.  In comparison, $\delta_g$ for
$\beta=100$ reaches a maximum of approximately $0.15 \%$ which is well below
$0.22 \%$.  Based on this result, we suggest the value of $\delta_g > 0.22 \%$
as an approximate criterion for the failure of the nonparallel linear
stability theory for this flow.  For some value of $35 < \beta_{crit} < 100$,
the maximum value of $\delta_g$ will be exactly $0.22 \%$.  For any $\beta <
\beta_{crit}$ the nonparallel linear stability results are likely to be
inaccurate.  Care must, of course, be used in generalizing this result, since
the critical value of $\delta_g$ depends (to some degree) on $\Re$, $\M$, the
geometry, and the particular instability mode under consideration.

In summary, the results presented above indicate that both curvature and
nonparallelism have a significant quantitative effect on the growth of
stationary crossflow vortices in the three-dimensional boundary layer on a
swept parabolic-cylinder.  Comparison of LST growth-rate predictions with LNS
calculations are generally in good agreement when both curvature and
nonparallel effects are accounted for in the theory.  Unlike previous
investigations, we have considered longer wavelength disturbances which, for
this geometry, are unstable over an extended chordwise region.  In this sense,
the current results are similar to the experiments of Saric \etal\
\cite{ReSaCaCh:96} and Bippes \etal\ \cite{DeBi:96} where there is a long
region of crossflow vortex growth.  For this long wavelength mode, nonparallel
effects are enhanced to such a degree that the nonparallel perturbation
approach is inadequate near the leading-edge.

\section{Receptivity to Surface Roughness \label{s:pcyl3d}}

Receptivity to surface roughness is investigated by modeling an array of
roughness elements near the leading edge.  The roughness elements are assumed
to be periodic in the $z$-direction with wavenumber $\beta$.  In the chordwise
direction a Gaussian distribution is used in the arc-length,
%
\begin{equation} \label{e:PHYbump}
  h_w(s) = \exp\left[ \frac{-( s - s_w)^2}{2 \sigma_w^2} \right] \comma
\end{equation}
%
where the roughness is centered about $s_w$ and $\sigma_w$ is the standard
deviation (\ie\ width) of the Gaussian distribution.  The roughness is modeled
as a small perturbation of the wall and, under the linear assumption, the wall
boundary conditions on the perturbed wall are converted to an inhomogeneous
set of boundary conditions enforced at the unperturbed wall location.  The
detailed derivation of such linearized boundary conditions is presented in
Appendix \ref{a:bumpbc} for both isothermal and constant heat-flux
wall-temperature boundary-conditions.  In computing the theoretical
receptivity predictions, the Fourier transform of the bump is required in the
chordwise direction.  For a Gaussian distribution, the Fourier transform is
also Gaussian and is given by
%
\begin{equation} \label{e:FTbump}
  \hat h_w(\alpha) = \sigma_w \exp\left[ \frac{-\alpha^2\sigma_w^2}{2} \right] 
  \period
\end{equation}

The primary objective in this section is to evaluate the accuracy and range of
validity of the recently developed Finite Reynolds Number Theory for the
prediction of receptivity.  This theory is summarized in Appendix \ref{a:FRNT}
in the context of surface roughness on an infinite swept wing.
%
%.... Open issues
%
%	1. What is the effect of mach number
%	2. What is the influence of sweep angle
%	3. How about variable coefficients
%	4. Nonlinearity
%

\subsection{Parallel Boundary-Layer Without Curvature \label{ss:pblfp} }

To evaluate the Finite Reynolds Number Theory (FRNT) for the prediction of
crossflow vortex receptivity, it is instructive to first investigate the
receptivity of stationary crossflow vortices for a parallel meanflow with no
curvature.  In this way, all the assumptions of the theory are satisfied and
the amplitudes from LNS and FRNT should agree to a high degree of precision.
For this purpose, consider the boundary-layer profile from the swept
parabolic-cylinder at the point of maximum crossflow velocity, $s=0.802$, as
given in figure \ref{f:profile}.  In order to make this comparison easily
reproducible, a compressible Falkner--Skan--Cooke (FSC) boundary-layer is fit
to the profile taken from the Navier--Stokes solution.  The methods used to
compute and fit the profile are given in Appendix \ref{a:FSC} and a comparison
of the NS and FSC velocity profiles is shown in figure \ref{f:upro}.  The FSC
profile is placed on a LNS mesh using the parallel flow assumption and, to
simplify the comparison with the parabolic cylinder results, the global
nondimensionalization is retained.  The information required to convert to a
local nondimensionalization is provided in Appendix \ref{a:FSC}.

The domain for the receptivity calculation extends from $x = 0$ to $x = 5$ in
the chordwise direction and from $y=0$ to $y=0.3$ in the wall-normal direction
which places the top boundary $78 \delta_1$ above the wall.  (Since the wall
is flat for this geometry, $s$ and $x$ are equivalent.)  The solutions are
obtained using the techniques introduced in \S\ref{ss:CFspace}.  The grid is
uniform in $x$ with 512 nodes and the standard mapping, equation
(\ref{e:map}), is used in $y$ with 63 nodes.  An outflow sponge is employed
over the last $20\%$ of the domain to damp the crossflow vortices and a zero
disturbance condition is imposed on the top boundary.  A Gaussian bump is
placed at $x_w = 0.7$ with a standard deviation of $\sigma_w=0.05$ and a
spanwise wavenumber of $\beta=35$.  For this spanwise wavenumber, LST predicts
a chordwise wavenumber of $\alpha = -37.39$ and a growth-rate of $\sigma =
0.3043$.  Figure \ref{f:FSC-35} shows the evolution of the square-root of the
integrated disturbance kinetic energy from the LNS solution.  Near the
roughness site, ${E'_k}^{1/2}$ has roughly the shape of the imposed Gaussian
bump.  Downstream, there is a mild transient as the spatially damped modes
excited by the bump quickly decay.  By $x=1.2$ the response is dominated by
the most unstable crossflow vortex and this continues to $x=4$ where the
outflow sponge begins to damp the crossflow mode.  Also included in figure
\ref{f:FSC-35} is the FRNT prediction where the LST growth-rate has been used
to extrapolate the amplitude upstream and downstream of the center of the
bump.  The FRNT prediction is computed using the methods presented in Appendix
\ref{a:FRNT}.  Downstream of the transient, the extrapolated FRNT and LNS
solution are in excellent agreement.

%
%  \beta = 35   \hat h_w = 8.7079E-03,   
%                alpha   = -3.7392922887292E+001  -3.0432293432786E-001
%
%  \beta = 30 	\hat h_w = 1.4698E-02
%		\alpha   = -3.1295756047131E+001   -3.7807393274490E-001
%

To better observe the transient behavior downstream of the bump, results have
also been obtained at $\beta = 30$ which is predicted by LST to have a
chordwise wavenumber of $\alpha = -31.30$ and a stable growth-rate of $\sigma
= 0.3781$.  Figure \ref{f:FSC-30} shows the evolution of ${E'_k}^{1/2}$ for
this case including the extrapolated FRNT prediction.  Again the agreement
between FRNT and LNS is excellent.  In this case, the transient is more
clearly observed since the solution decays for large $x$.  The transient
results in a local maximum in ${E'_k}^{1/2}$ at $x=1.2$ before the asymptotic
decay begins.  The overall transient growth (measured at the local maximum) is
small, about a factor of $1.5$ when measured by ${E'_k}^{1/2}$.  This modest
amount of transient growth is not surprising given the nature of the surface
roughness disturbance.  By examining the linearized roughness boundary
conditions in Appendix \ref{a:bumpbc}, it is seen that (linearized) roughness
leads to a direct perturbation of the wall tangent velocities $u'$ and $w'$
which are proportional to the wall-normal gradients of the mean velocities.
However, for a parallel boundary layer (and, to a good approximation, all
boundary layers) the perturbation of the wall-normal velocity is zero at the
wall.  This is because the wall-normal gradient of the mean normal-velocity is
zero (approximately zero for nonparallel flow).  Thus, the wall-normal
velocity is perturbed only in an indirect manner through the equations of
motion.  Research on temporal instabilities in an incompressible FSC boundary
layer \cite{BrKu:93} has shown that transient growth is most pronounced for
initial disturbances composed primarily of wall-normal velocity.  Compared
with surface roughness, one would therefore expect greater transient growth
associated with wall suction/blowing since this boundary condition directly
perturbs the normal velocity.

Figure \ref{f:59} shows the growth-rate based on $E'_k$ extracted from the LNS
solutions for both $\beta=35$ and $\beta=30$ compared to the LST predictions.
The growth-rate is a very sensitive indicator of both errors in the solution
and transient behavior.  As seen in the figure, the transient behavior
downstream of the bump is complete by $x=2$ for both wavenumbers, with the
growth-rate from the LNS solutions in excellent agreement with LST.  The
slight oscillations in the growth-rate downstream of $x=2$ (more noticeable
for $\beta=30$) are caused by the presence of damped modes which are generated
by the bump.  These modes have sufficiently small amplitudes that they are not
visible on the amplitude plots shown in figures \ref{f:FSC-35} and
\ref{f:FSC-30}.  When extracting receptivity results for the parabolic
cylinder, growth-rate results similar to those presented here are used to
ensure that the crossflow amplitude is measured downstream of the transient
(see \S\ref{ss:LNSpcyl}).

In summary, the amplitude predictions from FRNT are in excellent agreement
with LNS calculations for a parallel boundary layer with no curvature.  Since
the conditions of the LNS calculation have been constructed to satisfy all the
assumptions in the FRNT, the agreement is not surprising.  However, the
consistency of the results does suggest that both the LNS and FRNT
implementations are free of errors.  Furthermore, the transient associated
with local surface roughness is observed to be small even for an
asymptotically stable mode.  This fact greatly simplifies the receptivity
calculations presented in \S\ref{ss:LNSpcyl} for the parabolic cylinder since
a relatively short domain size can be used.

\subsection[Finite Reynolds Number Receptivity Theory for the \protect\\
Parabolic Cylinder]{Finite Reynolds Number Receptivity Theory for the 
Parabolic Cylinder\label{ss:FRNTpcyl} }

Having established agreement between LNS and FRNT for an idealized, parallel
boundary-layer with no curvature, we now consider the FRNT predictions for the
meanflow about the parabolic cylinder.  Since the FRNT is based on
quasi-parallel flow, nonparallel effects are obviously not included.  However,
curvature can be accounted for by forming the FRNT equations in body-fitted
curvilinear coordinates.  Therefore, in presenting the FRNT predictions,
results are given both with and without curvature to determine its influence.
In addition, FRNT results are given for both of the spanwise wavenumbers used
in the stability analysis of \S\ref{s:LSTpcyl}.

Applying FRNT to the mean boundary-layer profiles on the parabolic cylinder
yields the complex valued efficiency factor, $\Lambda$, as discussed in
Appendix \ref{a:FRNT}.  The expression for the initial amplitude and phase of
the crossflow vortex is given by equation (\ref{e:ampcf}) which is repeated
here as
%
\begin{equation} \label{e:ampCF}
  A_{cf} = \ew \: \hat h_w(\acf) \: \Lambda(\beta,\Re,\Pr,\M)
\end{equation}
%
where $\ew = \ew^* / L^*$ is the normalized roughness height and $\acf$ is the
chordwise wavenumber of the local crossflow eigenmode with spanwise
wavenumber, $\beta$.  Notice that the Fourier transform of the roughness
distribution is required at the chordwise wavenumber resonant with the
crossflow mode and this is obtained for the Gaussian bump using equation
(\ref{e:FTbump}).

Figures \ref{f:Lambda35} and \ref{f:Lambda100} show the magnitude of $\Lambda$
for $\beta=35$ and $\beta=100$ as a function of arc-length along the
parabolic-cylinder.  An interesting, and previously undocumented, result that
is immediately obvious is that convex curvature increases the efficiency of
the receptivity process for this flow.  This result holds true over the full
range of unstable spanwise wavenumbers as seen in figure \ref{f:LambdaBeta}
which shows the variation of $|\Lambda|$ with $\beta$ at the point of maximum
crossflow velocity, $s=0.802$, both with and without curvature.  The increase
in receptivity efficiency due to curvature is most pronounced near the first
neutral-point ($\beta \approx 40$) where there is a local maxima in the
receptivity efficiency.  A similar local maxima has been observed by Crouch
\cite{Crouch:93} for an incompressible FSC boundary layer on a flat-plate.
With the exception of the region immediately surrounding the lower neutral
point, curvature causes a relatively uniform increase of $10\%$ in the
receptivity efficiency factor at this station.  In interpreting the results in
figures \ref{f:Lambda35} and \ref{f:Lambda100}, it is important to realize
that $|\Lambda|$ is the response for a Dirac delta function bump in $s$ (\ie\
it is the Green's function response).  This explains why $|\Lambda|$ increases
without bound as $s$ is decreased since the delta function has a unit spectrum
for all $\alpha$.  The actual amplitude, as given by equation (\ref{e:ampCF}),
is modulated by the Fourier spectrum in $s$ of the bump shape, $h_w$, which
limits the amplitude for a fixed bump width as $s$ is decreased.

The increase in receptivity efficiency due to convex curvature is particularly
interesting given the fact that convex curvature has a stabilizing effect on
the growth of crossflow vortices.  The stabilizing effect of convex curvature
follows from Rayleigh's centrifugal stability criterion \cite{DrRe:81} which
states that for axisymmetric, inviscid flows with curved streamlines, the flow
is stabilized if the square of the angular momentum about the center of
curvature increases with the radius.  This is the case for the
three-dimensional boundary layer on a convex surface in which the instability
mechanism is the inviscid crossflow instability.  In fact, the correspondence
between the crossflow and G\"ortler instabilities has been recently examined
by Bossom \& Hall \cite{BaHa:91} who show that G\"ortler vortices smoothly
transition to crossflow vortices as the mean three-dimensionality is
increased.  However, the effect of curvature on crossflow vortex receptivity
has not been previously reported.  The reduced growth-rate with convex
curvature is offset, to some degree, by a greater initial amplitude.  Of
course, the net impact of curvature depends on the particular conditions.  In
the following section, the combined effects of nonparallel flow and curvature
on receptivity are examined using LNS calculations.

\subsection{Linearized Navier--Stokes Receptivity Predictions 
\label{ss:LNSpcyl} }

For comparison with the FRNT predictions presented in the previous section, a
series of LNS calculations have been performed.  The methods used to do so
have already been discussed in \S\ref{ss:CFspace} and \S\ref{ss:pblfp}.  We
first consider the short spanwise wavelength, $\beta=100$, since nonparallel
effects are expected to be less severe for this case.  Similar to the LNS
stability calculations performed above, a multidomain approach is employed
with two domains used to cover the unstable region.  The first domain includes
the region from $s=0.32$ to $1.80$ while the second domain covers $s=1.80$ to
$5.12$.  In each of these domains, a series of calculations are performed
using different bump locations, $s_w$, but all using that same bump width,
$\sigma_w = 0.01$.  The bump locations and associated receptivity results are
presented in table \ref{t:beta100}.  Figure \ref{f:45} shows the evolution of
${E'_k}^{1/2}$ for case 3 of table \ref{t:beta100}.  For this case, the bump
is placed at $s_w=0.5$ and the resulting transient and subsequent exponential
growth are qualitatively similar to that obtained for the parallel FSC profile
in \S\ref{ss:pblfp}.  The growth-rate computed from this solution is shown in
figure \ref{f:46} compared to the LNS stability result from \S\ref{s:LSTpcyl}.
Downstream of $s=1.2$, and until the start of the sponge at $s=1.5$, the two
growth-rates are in excellent agreement.  As a general procedure for
extracting the amplitude of the dominant crossflow mode, we compare the
growth-rate from the receptivity calculation with the LNS stability result.
Since the growth-rate is a very sensitive indication of transient behavior,
this comparison allows us to determine a location, $s_l$, where the transient
is negligible.  For case 3 we selected $s_l = 1.4$ and measured the amplitude
of the response at this location, $A_l$, based on ${E'_k}^{1/2}(s_l)$.  These
results are shown for each bump location in table \ref{t:beta100}.

To compare the results to FRNT, the crossflow amplitudes must be extrapolated
to the location of the bump, $s_w$, since this is the value given by the
theory.  For this purpose, we use the $N$-factors based on the LNS
calculations discussed in \S\ref{s:LSTpcyl}.  The amplitude at $s_w$ is then
given by
%
\begin{equation}
  A_w = A_l e^{(N_w - N_l)}
\end{equation}
%
where $N_w$ is the $N$-factor at the bump location, and $N_l$ is the
$N$-factor at the measurement station.  Another useful measure of receptivity
is the effective amplitude at the upstream neutral point.  This amplitude is
simply given by
%
\begin{equation}
  A_{np} = A_l e^{-N_l}
\end{equation}
%
since the $N$-factor is defined relative to the neutral point.  Both $A_w$ and
$A_{np}$ are also reported in table \ref{t:beta100}.

The amplitude at the bump location is compared to the FRNT predictions, both
with and without curvature, in figure \ref{f:47}.  The FRNT predictions are
computed using equation (\ref{e:ampCF}) with the receptivity efficiency
factors found in figure \ref{f:Lambda100}.  The amplitude, $A_w$, generally
follows the same trend as the efficiency factor for large $s$, with $A_w$
increasing as $s$ is reduced.  At about $s=0.65$, there is a maximum in $A_w$
which is associated with the Fourier spectrum of the roughness distribution
with $\sigma_w=0.01$.  When comparing the LNS results to the FRNT predictions,
the first observation is that FRNT over-predicts the amplitude, by as much as
$29\%$ with curvature and $21\%$ without curvature.  Based on these results,
it is obvious that nonparallel effects tend to reduce the amplitude of
crossflow vortices generated by roughness.  Similar to curvature, this result
is opposite to that observed for stability characteristics, where nonparallel
flow, generally speaking, destabilizes crossflow vortices.  Downstream of the
maximum, the FRNT predictions approach the LNS solution with the no-curvature
prediction in slightly better agreement than the prediction with curvature.
The fact that the FRNT prediction without curvature is in better agreement
with the LNS solution is similar to the stability results where the
growth-rate with curvature is a worse approximation to the actual growth-rate
compared with the local flat-plate prediction.  Since nonparallel and
curvature effects are competing, including only one of the effects results in
a prediction which is in greater error than the original parallel, flat-plate
value.

Similar results have been obtained for $\beta=35$ using a bump with a standard
deviation of $\sigma_w=0.05$.  The results from these calculations are
summarized in table \ref{t:beta35}.  Figure \ref{f:48} shows the effective
amplitude of the crossflow vortex at the bump location for LNS and FRNT.
Similar to the $\beta=100$ case, both FRNT results significantly over-predict
the crossflow amplitude near $s=1$.  The greater nonparallel effects for this
long spanwise wavelength are evident in the maximum error which is $77\%$ with
curvature and $45\%$ without.  Beyond approximately $s=5$, the FRNT amplitudes
are in reasonably good agreement with LNS.  However, upstream of the neutral
point, $s=0.40$, the FRNT amplitudes are slightly lower than the LNS values.
This is more clear in figure \ref{f:49} which also shows $A_w$, but plotted on
a log-log scale.  A similar plot (figure \ref{f:50}) for $\beta=100$ indicates
that LNS and FRNT amplitudes will likely cross over upstream of the neutral
point ($s = 0.465$).

The effect of the width of the Gaussian bump, as predicted by FRNT with
no-curvature, is shown in figures \ref{f:60} and \ref{f:61} for $\beta=100$
and 35 respectively.  With reference to equation (\ref{e:ampCF}) we see that
the influence of the bump width on $|\Lambda|$ is through the amplitude of the
Fourier coefficient of the Gaussian at the chordwise wavenumber which is
locally resonant with the crossflow mode.  For a fixed value of $\sigma_w$
(and $\beta$), there is a variation of $|\hat h_w|$ in the chordwise direction
since the local wavenumber of the dominant crossflow mode changes (see figure
\ref{f:62}).  Thus, as $\sigma_w$ is increased the Fourier magnitude of the
Gaussian bump is reduced for the high wavenumbers near the leading-edge
resulting, eventually, in a drop-off of the amplitude that moves downstream.
From this result, it is clear that the drop-off in the response for both
$\beta=35$ and $\beta=100$ cases above, is due to the choice of $\sigma_w$.
We note in passing that the $\sigma_w$ used in the LNS calculations were
selected to ease the resolution requirements near the bump.  The actual value
of $\sigma_w$ is not expected to alter the qualitative comparison to FRNT.
This is demonstrated in figure \ref{f:58} which shows the effect of $\sigma_w$
on the initial crossflow amplitude with the bump placed at $s=0.6$ and with
$\beta=100$.  As expected, the FRNT prediction overestimates the amplitude for
all $\sigma_w$.  However, the functional dependence of $A_w$ on $\sigma_w$ is
correctly predicted by FRNT.  This is demonstrated by setting $\alpha =
-141.5$, which is the value based on $|\hat\vs|_{max}$ from the LNS stability
calculation (see figure \ref{f:62}), and adjusting $|\Lambda|$ in equation
$\ref{e:ampCF}$ to generate a curve that fits the LNS data.  Excellent
agreement is achieved for $|\Lambda|=193$ and this curve is also shown in
figure \ref{f:58}.  We point out, that estimating $|\Lambda|$ in this manner
is only an approximation, since the value of $\alpha$ used to evaluate $|\hat
h_w|$ is not necessarily the ``correct'' value.  A better fit could be
obtained by optimizing both $\alpha$ and $|\Lambda|$, however the improvement
is slight.

Given the success of the current method, we have used the same procedure to
estimate the value of $|\Lambda|$ from the LNS solutions at other chordwise
stations.  Using $\alpha$, based on $|\hat\vs|_{max}$ from the LNS stability
calculation, equation (\ref{e:FTbump}) is used to compute the local value of
$|\hat h_w|$ and equation (\ref{e:ampCF}) is then solved for $|\Lambda|$.  The
results, for both values of $\beta$, are shown in figure \ref{f:66} along with
the FRNT predictions.  For both spanwise wavenumbers, the receptivity
efficiencies from LNS and FRNT approach one another for large $s$.  This is
consistent with the reduction in nonparallel effects with increasing $s$
observed in the stability analysis.  Near the leading-edge (\ie\ upstream of
$s=1$), nonparallel effects increase causing FRNT to over-predict the
receptivity efficiency.  The influence of nonparallel effects is more
pronounced for $\beta=35$ which is also consistent with the linear stability
results.  At the upstream neutral point, $s=0.395$ for $\beta=35$, the FRNT
result with curvature has an error of 264\% while the prediction without
curvature has a $182\%$ error.  For $\beta=100$ the errors in the FRNT are
reduced considerably to $32.2\%$ for FRNT with curvature, and $24.5\%$ without
curvature.  For both spanwise wavenumbers, nonparallel effects decrease the
receptivity efficiency over the entire unstable region.  Based on these
results, we conclude that the crossover in $A_w$ observed for small $s$ in
figures \ref{f:49} is primarily due to errors in the LST prediction of
$\alpha$ used in the FRNT to compute $|\hat h_w|$.  In general, the
receptivity efficiency is consistently reduced by nonparallel effects for this
flow as shown in figure \ref{f:66}.
%
%.... Receptivity Efficiencies at the neutral point
%
%  Beta = 35
%
%  s_np = 0.395066	Lambda_nc  = 450.32	Error = 182 %	Fact = 2.82
%			Lambda_wc  = 581.48	Error = 264 %	Fact = 3.64
%			Lambda_lns = 159.62
%
%  Beta = 100
%
%  s_np = 0.4657214	Lambda_nc  = 308.96	Error = 24.5 %	Fact = 1.25
%			Lambda_wc  = 328.08 	Error = 32.2 %	Fact = 1.32
%			Lambda_lns = 248.14
%

\section{Discussion of Results \label{s:discussion} }

The results presented in the previous sections have identified curvature and
nonparallel effects as having a significant quantitative impact on the initial
amplitude of stationary crossflow vortices downstream of a localized roughness
element.  Given the previously documented \cite{MaMa:94,MaBa:93} effect of
both curvature and nonparallelism on the growth characteristics of crossflow
vortices, their impact on receptivity is not surprising.  What is interesting,
however, is that curvature, known to stabilize crossflow vortices, actually
enhances receptivity; while nonparallel effects, which typically destabilize
crossflow modes, are found to attenuate the initial crossflow amplitude.  For
the limited range of parameters investigated here, the attenuation due to
nonparallel effects is greater than the enhancement due to convex surface
curvature.  Progressing downstream on the parabolic-cylinder, curvature and
nonparallel effects are naturally reduced so that the FRNT predictions
approach the LNS solution for large $s$.  However, the effect of
nonparallelism is still evident at the furthest downstream stations
investigated and the amplitudes predicted by FRNT are consistently greater
than the NS solution.  For both wavenumbers investigated, the error in the
theory for the farthest downstream location is still approximately $5\%$.  The
fact that nonparallel effects are so significant for this flow stems from the
fact that the first neutral point for crossflow vortices typically lies very
close to the leading-edge.  Choudhari \cite{Choudhari:94}, when presenting
FRNT results for FSC profiles, cautions that since the neutral point for
crossflow vortices occurs near the leading-edge for realistic geometries,
nonparallel effects may be important.  This hypothesis is clearly
substantiated by the current results where the error in the predicted
amplitude (with curvature) is as high as $77\%$ shortly downstream of the
neutral point.

The current results are in contrast to the recent study by Crouch \& Spalart
\cite{CrSp:95} who investigated the influence of nonparallel effects on the
receptivity of a two-dimensional boundary layer with localized suction
subjected to a freestream acoustic wave.  Comparing numerical simulations with
FRNT predictions, they found that the theory is in very good agreement with
simulations downstream of the first neutral point for the prediction of
Tollmien--Schlichting wave amplitudes.  At the neutral point, the theory
under-predicts the amplitude by $4\%$ and this difference increases further
upstream to a value between $6.5\%$ and $11.2\%$.  Based on these results,
they conclude that neglecting the weak boundary-layer growth in the
receptivity theory is an acceptable approximation for this flow.

Based on the current results, the conclusion of Crouch \& Spalart obviously
does not apply for crossflow vortices.  It is particularly interesting to note
that they found nonparallel effects to slightly increase the receptivity
amplitude for TS waves, while we find that initial crossflow vortex amplitudes
are generally attenuated by nonparallel effects.  This is in contrast with the
influence of nonparallel flow on stability characteristics.  Fasel \&
Konzelmann \cite{FaKo:90}, in comparisons of Navier--Stokes solutions and
stability analysis, show that for TS waves in a Blasius boundary layer
nonparallel effects lead to an increase in the growth-rate.  Similarly, we
have shown that nonparallel flow also destabilizes crossflow vortices.  Thus,
there is no apparent trend linking stability and receptivity results
concerning nonparallel effects.  However, we note that the receptivity and
stability processes under consideration are quite different, and this
statement must be considered in this light.

%==============================================================================
%
%  Tables and Figures
%
%==============================================================================

\input chp5fig

