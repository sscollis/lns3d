%%%%%%%%%%%%%%%%%%%%%%%%%%%%%%%%%%%%%%%%%%%%%%%%%%%%%%%%%%%%%%%%%%%%%%%%%%%%%%%
%
%  Chapter 4:  Model Problems and Code Validation
%
%  S. Scott Collis
%
%  Written: 9-5-95
%
%  Revised: 9-6-96
%
%%%%%%%%%%%%%%%%%%%%%%%%%%%%%%%%%%%%%%%%%%%%%%%%%%%%%%%%%%%%%%%%%%%%%%%%%%%%%%%
\chapter[Model Problems and Code Validation]{Model Problems and \protect\\ 
         Code Validation \label{c:valid}}

Although the numerical method is presented in Chapter \ref{c:method} in the
context of the swept, leading-edge receptivity problem, it is general enough
to solve a variety of other problems.  This chapter begins by presenting
results for three model problems that contain many of the fundamental physical
phenomena occurring in leading-edge receptivity problems: unstable
Tollmien--Schlichting waves, growing crossflow vortices, and acoustic
scattering.  By isolating the physical phenomena, the viability of the method
is evaluated and the resolution required to accurately represent each
phenomenon is ascertained.  As a final validation of the method, the last
section of this chapter presents acoustic leading-edge receptivity results
along with comparisons to a previous computation \cite{Lin:92}.  Readers,
anxious for swept leading-edge receptivity results, may wish to bypass this
chapter and go directly to Chapter~\ref{c:pcyl} which discusses receptivity due
to surface roughness on a swept parabolic-cylinder.

\section[Viscous Instability Waves in a Parallel Boundary Layer]{Viscous 
Instability Waves in a \protect\\ Parallel Boundary Layer \label{s:TS}}

A major feature of the transitional flow near a leading-edge is the
development of Tollmien--Schlichting (TS) waves in the boundary layer.  Since
TS waves have large gradients in the wall normal direction, it is
computationally challenging to resolve them accurately while representing the
mean flow variation on the scale of the airfoil.  To determine the ability of
our scheme to accurately compute the development of TS waves, we have
performed a series of calculations using a parallel boundary layer as the base
flow.  In this way, the solutions can be compared directly to parallel, Linear
Stability Theory (LST).

The base flow is computed from the compressible Falkner--Skan--Cooke equations
(see Appendix~\ref{a:FSC}) for an unswept ($\theta=0$), zero-pressure-gradient
($\beta_h=0$) boundary layer.  The flow conditions are $\M=0.3$, $\Re=1000$,
and $\Pr=1.0$ with the reference length scale $L^*=\delta^*_1$ ($\delta^*_1$
is the displacement thickness) and all other reference quantities are based on
edge values.  The wall is assumed to be isothermal with $T_w = T_0$, where
$T_0$ is the freestream stagnation temperature.\footnote{Under these
conditions, $T_0$ is also the adiabatic wall temperature.}  A linear stability
solver (see Appendix~\ref{a:LST}) using Chebyshev collocation with 96 points
in the wall normal direction is used to provide the reference growth-rates for
both temporally and spatially growing waves.  For both the Linearized
Navier--Stokes (LNS) and LST calculations, constant fluid properties are
assumed.

\subsection{Temporally Growing TS Waves \label{ss:TStime}}

First, consider the temporal problem where the waves grow in time. In this
case, the disturbance field is periodic in the streamwise direction.  A
computational domain containing only one wavelength of the TS wave is used
with periodic boundary conditions on the left and right boundaries.  For this
calculation, the top boundary is placed at $y=80$ which is sufficiently far
away from the wall that a zero disturbance boundary condition can be used.  At
the wall, the no-slip and isothermal boundary conditions are enforced as
described in \S\ref{sss:distwallbc}.  Since we are interested in the temporal
evolution of the disturbances, the unsteady solver is used with two-step
implicit time-advancement.

The mesh is uniform in the streamwise direction, $x$, with an algebraic
mapping function in the wall normal direction, $y$.  The mapping function to
computational space, $\eta \in [0,1]$, is given by equation (\ref{e:tsmap})
with $y_{max}=80$ and $y_s=1$.

The LST eigenfunction for $k_x = 0.308620690$ is used as the initial condition
with a theoretical frequency of
%
% FSC: \M=0.3, \theta=0, \beta_h=0, T_w/T_0=1, \Re_\delta_1=1000, \Pr=1
%
% \omega = 1.1467880633548E-001   2.3844530425242E-003  (y_i=0.5, 64)
% \omega = 1.1467880627332E-001   2.3844531249205E-003  (y_i=1.0, 96)
%
% \lambda_{TS} = 20.35892378
%
% \Delta t = 0.04764297421746
%
\begin{equation}
  \omega_{LST} = 0.114678806 +  0.002384453 i
\end{equation}
%
and the growth-rate is given by the imaginary part, $\sigma_{LST} =
0.002384453$.  In passing, we note that the eigenvalues from the LST
calculation are believed to be accurate to the number of digits indicated.
This has been verified by calculations with 128 Chebyshev points which show no
change to 9 decimal places.

Growth-rate results from the Navier--Stokes calculations are based on the
integrated, disturbance kinetic-energy
%
\begin{equation}
  E'_k \equiv \int_0^\infty \frac{1}{2} \left( |\hat u'|^2 + |\hat v'|^2 + 
                                               |\hat w'|^2 \right) dy \period
\end{equation}
%
which is computed from the streamwise Fourier transform of the disturbance
velocities for wavenumber $k_x$.  Given $E'_k$ the growth rate is given by
%
\begin{equation}
  \sigma = \frac{1}{2} \frac{1}{E'_k} \frac{dE'_k}{dt}
\end{equation}
%
where the factor $1/2$ is required since $E'_k$ is a quadratic quantity.

Figure~\ref{f:ttsgrowth} shows the evolution of the growth-rate in the
computed solution as a function of time for two spatial resolutions where each
run was performed at a $\Delta t$ corresponding to $N_t=1150$ time-steps per
period with two iterations of the implicit solver.  For early times, the
growth-rate undergoes a transient as the solution adjusts from the LST initial
condition to the eigenfunction associated with the discrete finite difference
equations. During the transient, additional acoustic modes are also generated
and these modes lead to the high frequency modulation of the growth-rate seen
in figure~\ref{f:ttsgrowth}.  As grid resolution is increased the severity of
the initial transient and the amplitude of the resulting acoustic modes are
reduced since the finite difference eigenfunction converges to the LST
eigenfunction used for the initial condition.

To verify the rate of convergence, the relative error in the average growth
rate is plotted in figure~\ref{f:spconv} over a range of mesh resolutions for
$N_t=1150$ with 2 iterations of the implicit solver.  Fourth-order convergence
is achieved in both spatial dimensions although the relative error begins to
saturate at $10^{-4}$ due to the temporal error. If the $\CFL$ had been held
constant instead of fixing $N_t$, then the saturation due to temporal errors
would not have occurred \cite{KiMo:85}.  However, the conclusion would have
remained the same --- the method is fourth-order accurate in space.

To confirm temporal convergence, figure~\ref{f:tmconv} shows the relative
error in the average growth-rate as a function of $N_t$, for an $(80 \times
127)$ mesh. Results are included for 1, 2, and 3 iterations of the implicit
disturbance solver.  In each case, (at least) second-order temporal
convergence is obtained although the overall accuracy improves with the number
of iterations.  In fact, the rate of convergence for 2 and 3 iterations is
actually greater than second-order over the range of $N_t$ considered. This is
due to the fact that as $N_t$ is increased, the additional iterations are more
successful in removing the factorization error thus driving the solution to
the exact, discrete solution.  Eventually, however, as $N_t$ is further
increased the second-order error of the base time-advancement scheme will
dominate regardless of the number of iterations.

With the convergence results presented above, it is possible to determine a
spatial/temporal resolution required to resolve TS wave development to any
desired accuracy.  For the leading-edge receptivity problem a reasonable level
of accuracy requires less than 1\% error in the amplitude of the TS wave over
the length of the computational domain.  This level of accuracy can be
achieved for a period of 40 oscillations in time with a resolution of
$N_x=20$, $N_y=127$, $N_t=1150$ with 2 iterations of the implicit solver.  For
this resolution, the error in the average growth-rate is $0.22\%$.  In terms
of a spatially growing wave, with 20 nodes per streamwise wavelength, 127
nodes normal to the wall, and 1150 time-steps per period with 2 iterations, a
1\% error in the amplitude will not occur until a domain length of 40
wavelengths which should be adequate for the leading-edge receptivity problems
under consideration.

\subsection{Spatially Growing TS Waves \label{ss:TSspace} }

Now consider the case where the TS waves grow in space on the same parallel,
mean boundary-layer profile.  For a frequency $\omega=0.08$ the spatial
wavenumber predicted by LST is
%
% FSC: M=0.3, lambda=0, beta_h=0, Tw/T0=1, Re_\delta_1 = 1000, Pr=1
%
% k_x = 2.2804739410500E-001  -6.5163146761218E-003  (yi=1.0, 64)
% k_x = 2.2804739411367E-001  -6.5163146912049E-003  (yi=1.0, 96)
%
% \lambda_{TS} = 27.552103
%
\begin{equation}
  {k_x}_{LST} = 0.228047394 - 0.006516315 i
\end{equation}
%
where the growth-rate is given by the negative imaginary part of the
wavenumber, $\sigma_{LST} = 0.006516315$.

For the spatial calculations the same wall-normal grid is used as in
\S\ref{ss:TStime}. However, the streamwise length of the domain is increased
to allow the TS waves to grow in space.  Two domain sizes are considered: $5
\lambda_{TS}$ and $10 \lambda_{TS}$.  Based on the temporal convergence
results, 20 nodes are used per TS wavelength resulting in an overall mesh size
of $(101 \times 127)$ for the $5\lambda_{TS}$ case and $(201 \times 127)$ for
the $10\lambda_{TS}$ case.

Unlike the temporal problem, both inflow and outflow boundary conditions are
required.  On the inflow boundary, the eigenfunction from LST is enforced by
specifying the incoming inviscid char\-acter\-istics \cite{Giles:90}. On the
outflow, a sponge is used over the last wavelength of the domain (see
\S\ref{sss:sponge}).  The sponge function is given by equation
(\ref{e:spgfun}) with $A_s = 2$, $N_s = 3$, and the reference state
corresponds to zero disturbances.  The top and wall boundaries are treated in
the same manner as in \S\ref{ss:TStime}.  Since we are interested in the time
asymptotic solution, the frequency domain approach is used with $\omega =
0.08$.

Growth-rate results for both domains are presented in figure~\ref{f:stsgrowth}
based on the integrated, disturbance kinetic-energy using
%
\begin{equation}
  \sigma = \frac{1}{2} \frac{1}{E'_k} \frac{dE'_k}{dx} \period
\end{equation}
%
In the figure, the sponge regions have been truncated so that the right
boundary is coincident with the start of the sponge.  From the inflow boundary
to the location $1 \lambda_{TS}$ upstream of the sponge the growth-rates from
both domains are in reasonable agreement with LST.  To quantify this, the
average growth-rate in this region gives a relative error of $2.9\times
10^{-3}$ (for both domains) which is consistent with the error in the temporal
problem ($2.2\times 10^{-3}$) for the same resolution.  However, the
introduction of inflow/outflow boundaries causes several important differences
between the spatial and temporal results discussed above.  The most obvious
difference occurs near the outflow sponge where the spatial growth-rate
undergoes a large fluctuation that appears to be limited to about $1
\lambda_{TS}$ upstream of the start of the sponge for both domain
lengths. This upstream influence is likely due to the elliptic nature of the
streamwise viscous terms in the boundary layer.  The nature and extent of the
upstream influence are consistent with the results of previous authors
\cite{GuAd:93,Adams:94} who used a similar sponge term.

Another difference between the spatial and temporal results is that the
magnitude of the oscillations in growth rate about the mean value are much
higher, $\approx 2\%$, for the spatial simulation as compared to $\approx
0.1\%$ for the temporal simulation.  There are several potential sources for
these oscillations.  First, similar to the temporal problem, there is a
mismatch between the LST eigenfunction forced on the inflow and the discrete
eigenfunction of the finite difference scheme.  Second, the one-dimensional,
inviscid boundary condition used to force the eigenfunction at the inflow is
only an approximation.  Third, the outflow sponge may cause reflections (both
numerical waves and physical waves) which could travel upstream.

To isolate the cause of these oscillations, a third simulation was performed
using the $10 \lambda_x$ domain but with the outflow sponge set to use the LST
solution as the reference state.  This ensures that the transient in the
sponge will be less severe since the sponge only has to correct for the errors
in the discrete finite difference solution instead of damping the disturbances
to zero.  Figure~\ref{f:stsgrowth2} shows the growth-rate results from this
simulation.  By comparison to figure \ref{f:stsgrowth}, the extent of the
direct upstream influence of the sponge is similar to the previous results,
although the magnitude of the influence is much less.  The amplitude of the
oscillations in the interior of the domain are also reduced---they are on the
order of $0.1\%$ instead of $2\%$.  From this, we conclude that the majority
of the errors in the previous solutions are due to reflections from the strong
sponge, not errors due to inflow forcing.  In constructing the sponge term,
these results suggest that it is advantageous to devise the reference state so
that the transient in the sponge is minimized.  Unlike the current case,
however, in receptivity calculations the solution in the sponge is not known
{\it a priori} so that different means of improving the solution must be
sought.

Returning to the zero reference state sponge, another way of reducing the
influence of the sponge is to increase the length of the sponge while keeping
the sponge function (\ie\ $A_s$ and $N_s$) constant.  Figure
\ref{f:stsgrowth3} shows growth-rate results for sponge lengths of 1, 2, and 5
$\lambda_{TS}$.  As the sponge length is increased the magnitude of the
upstream influence is reduced for both the fluctuation directly upstream of
the sponge and the oscillations in the interior.  Note: the fluctuation
directly upstream of the sponge can only be seen for the 5 $\lambda_{TS}$
sponge since this portion of the domain is not shown for the shorter sponges.
However, by comparison to the results of figure \ref{f:stsgrowth} the direct
upstream influence of the sponge has been significantly reduced by increasing
the length of the sponge.  A similar, although smaller, reduction is obtained
for the $2.5\lambda_{TS}$ case.  Comparing the solution with a $5\lambda_{TS}$
sponge to the solution with a $\lambda_{TS}$ sponge using the LST reference
state (figure \ref{f:stsgrowth2}) shows that the solutions are of comparable
quality with errors on the order of $0.2\%$.

To summarize the results of this section, the resolution estimates made for
the temporal problem in \S\ref{ss:TStime} are verified for the spatial
problem.  Furthermore, the outflow sponge is found to be successful in damping
TS waves before they interact with the outflow boundary, although not without
some level of upstream influence.  Two types of upstream influence are
observed.  The first is a spatially damped oscillation of approximate length
$\lambda_{TS}$ directly upstream of the start of the sponge.  The second type
of upstream influence is due to reflections (both numerical and physical)
which lead to oscillations of the growth-rate in the interior.  Both forms of
upstream influence can be reduced by either improving the sponge reference
state or by increasing the length of the sponge.  In either case, the effect
is to reduce the strength of the transition in the sponge.  Acceptable
solutions are obtained using a sponge length of $5\lambda_{TS}$ with a zero
disturbance reference state.  For this case, the errors due to upstream
influence are comparable in magnitude to the error in the average growth-rate
and the maximum error in the growth-rate is less than $0.4\%$.
%
%.... TTD:  May want to address the cost of the sponge.
%

\section[Crossflow Vortices in a Parallel Boundary Layer.]{Crossflow 
Vortices in a \protect \\ Parallel Boundary Layer \label{s:CFtest}}

In addition to TS waves, the transitional flow near a swept leading-edge may
also include crossflow (CF) vortices.  To evaluate the ability of our
numerical method to predict the growth of CF vortices we have performed a
series of calculations using a parallel boundary layer as the base flow.
Similar to the TS wave study above, the solutions can be compared directly to
parallel, Linear Stability Theory (LST).

The base flow is computed from the compressible Falkner--Skan--Cooke equations
(see Appendix~\ref{a:FSC}).  However, we now consider profiles with sweep and
favorable pressure gradient ($\theta=45^\circ$, $\beta_h=1$) such that the
crossflow component of velocity is maximized \cite{Mack:84a}.  The flow
conditions are $\M=0.3$, $\Re=400$, and $\Pr=1.0$.  The reference length for
this calculation is the displacement thickness of the chordwise velocity
profile and all other reference quantities are based on edge values.  The wall
is isothermal at the adiabatic wall temperature, $T_w = T_0$, and constant
fluid properties are assumed.  The reference growth rates are obtained from
the linear stability solver (see Appendix~\ref{a:LST}) with 96 Chebyshev
collocation points.

\subsection{Temporally Growing CF Vortices \label{ss:CFtime} }

As in the case of TS waves, we begin by considering the growth of CF vortices
in time.  For this test, a stationary mode with $k_x=-0.287436451$ and
$k_z=0.35$ is simulated and for these wavenumbers LST predicts a purely
imaginary frequency of
%
% FSC: \M=0.3, \theta=45, \beta_h=1, T_w/T_0=1, \Re_\delta_1=400, \Pr=1
%      f'', g' = 1.2385480276669E+00  5.7111548452863E-01
%      delta_1/l = 8.078596E-01, delta_2/l = 3.435010E-01, H = 2.351841
%
% \alpha_c = -0.287436451, \beta_c  =  0.350000000
%
% \omega = 6.3418714665148E-007   6.5335847229711E-003 (y_i=3.0, 64)
% \omega = 6.3418779141636E-007   6.5335847210238E-003 (y_i=3.0, 96)
%
% \lambda_x = 21.85938938962055
% \lambda_z = 17.95195802051310
%
\begin{equation}
  \omega_{LST} = 0.006533585 i
\end{equation}
%
so that the growth rate is given by $\sigma = 0.006533585$.  The LNS
calculations use the same wall normal mapping used for the temporally growing
TS waves in \S\ref{ss:TStime}.  The unsteady disturbance code is used with
$\Delta t = 0.047643$ (the nominal time-step in the TS wave study) and with 2
iterations per time-step.

Figure~\ref{f:tcfgrowth} shows the evolution of the growth-rate of disturbance
kinetic energy from the LNS calculation compared to LST.  The results for two
spatial resolutions, $(20 \times 127)$ and $(80 \times 127)$, are presented
and clearly demonstrate that the LNS solution converges to the LST solution as
resolution is increased. Consistent with the TS wave study, the relative error
in the growth-rate for a resolution of $(20 \times 127)$ is $0.28\%$ which
should be adequate for the present investigation.

\subsection{Spatially Growing CF Vortices \label{ss:CFspace} }

Now consider the spatial growth of a stationary, $\omega=0$, CF vortex mode
with $k_z = 0.35$.  LST predicts the chordwise wavenumber to be
%
% FSC: \M=0.3, \theta=45, \beta_h=1, T_w/T_0=1, \Re_\delta_1=400, \Pr=1
%      f'', g' = 1.2385480276669E+00  5.7111548452863E-01
%      delta_1/l = 8.078596E-01, delta_2/l = 3.435010E-01, H = 2.351841
%
% \omega   =  0.0
% \beta_c  =  0.350000000
%
% \alpha = -2.8831962908108E-001  -1.3854663674703E-002 (y_i=3.0, 64)
% \alpha = -2.8831962908130E-001  -1.3854663671636E-002 (y_i=3.0, 96)
%
% \lambda_x = 21.79242990567478
% \lambda_z = 17.95195802051310
%
% L_x = 10 \lambda_x = 217.9242990567477
%
\begin{equation}
  {k_x}_{LST} = -0.288319629 - 0.013854663 i
\end{equation}
%
so that the spatial growth rate is given by $\sigma_{LST} = 0.013854663$.

Similar to the spatial TS wave investigation, a domain is constructed that is
$10 \lambda_x$ long and uses $20$ nodes per streamwise wavelength.  The wall
normal mesh is identical to that used in the temporal investigation in
\S\ref{ss:CFtime}.  On the inflow boundary the LST eigenfunction is enforced
by setting the incoming characteristics \cite{Giles:90}.  Zero disturbance
conditions are used on the top boundary and a sponge of length $\lambda_x$ is
used on the outflow boundary.  The sponge parameters are $A_s=2$, $N_s=3$, and
the reference state is zero.

The computed growth-rate, based on the disturbance kinetic energy, is shown in
figure~\ref{f:scfgrowth} compared to the value from LST where the sponge
region has again been truncated.  Computing the average growth-rate in the
domain yields a value within $0.3\%$ of the LST result.  Similar to the
spatial TS wave results using a similar sponge, the sponge causes a
fluctuation in the growth-rate that appears limited to $ 1 \lambda_{CF}$
upstream of the start of the sponge.  However, unlike the spatial TS wave
results, the overall level of the oscillations in the interior are much
smaller indicating that the upstream influence of the sponge is more benign.

Given these results, we conclude that the outflow sponge, relative to the TS
wave problem, has relatively little impact on the CF vortex solution.  This
fact allows the use of much shorter sponges, on the order of $\lambda$ for
CF vortices, compared to $5\lambda$ for TS waves.

\section{Acoustic Scattering from a Circular Cylinder\label{s:scat}}

A fundamental aspect of the acoustic receptivity problem is the interaction of
free\-stream acoustic waves with the leading-edge.  This interaction results
in a scattered acoustic wave which propagates away from the body. To ensure
that our simulations accurately represent this phenomenon, we have validated
our code for the simple case of acoustic scattering from a stationary circular
cylinder of radius $a^*$ subject to a plane acoustic-wave traveling
perpendicular to the cylinder axis.  For this calculation, the reference
length is the cylinder radius $L^* = a^*$, the reference velocity is the
far-field acoustic speed $u^*_r = c^*_\infty$, and all other reference
quantities are based on far-field values.

The incident, plane acoustic-wave with nondimensional wavelength $\lambda$ is
given by
%
\begin{equation}
 p_i = P_0 e^{ i k (x - t) }, \quad k = {{2 \pi}\over{\lambda}}
\end{equation}
%
where the direction of propagation is along the positive $x$-axis and $P_0$ is
the incident pressure amplitude, taken to be $0.5$.

Under these conditions, the scattered pressure wave can be expressed by the
following Bessel function expansion \cite{MoIn:68}
%
\begin{equation}
p_s = \sum_{m=0}^{\infty} A_m \cos(m\theta)
      [ J_m(kr) + iN_m(kr)] e^{-i\omega t}
\end{equation}
%
where $(r,\theta)$ are the the usual cylindrical coordinates, $\omega=k$, and
%
\begin{eqnarray}
  A_m = -\epsilon_m P_0 i^{m+1} e^{-i\gamma_m}\sin\gamma_m , \\
         \tan \gamma_0 = -{{J_1(k)}\over{N_1(k)}} , \\ 
         \tan \gamma_m = {{J_{m-1}(k) - J_{m+1}(k)}\over
		          {N_{m+1}(k) - N_{m-1}(k)}}
\end{eqnarray}
%
with $\epsilon_0=1$ and $\epsilon_m = 2$ for all $m$ larger than zero.  In
these expressions, $J_m$ and $N_m$ are Bessel functions of the first- and
second-kind, respectively.  This solution is exact in the inviscid limit and
is used to evaluate the accuracy of our numerical scheme.

An inviscid simulation is conducted for an incident acoustic wave with
$\lambda = 2.5$.  A circular mesh with dimensions $(256 \times 241)$ is used
with the far-field boundary placed at $r_o=13.5$.  The mesh is uniform in
$\theta$ and in $r$ and provides a minimum of 18 nodes per incident wavelength
in $\theta$ and 48 nodes per scattered wavelength in $r$.  Since the flow is
symmetric about the $x$-axis, we solve for the solution only on the upper
half-plane with symmetry conditions used on the $x$-axis.

Before proceeding, we note that the rather high resolution in $r$ is used to
reduce the accumulation of phase errors in the calculation since the acoustic
waves propagate over 10 wavelengths within the domain.  Compared to our
acoustic receptivity calculations (see Chapter \ref{c:2dle}), this scattering
test is a severe test of our method.

%==============================================================================
%.... Inviscid wall boundary condition
%==============================================================================

Since only viscous wall boundary conditions were discussed in
\S\ref{ss:distbc}, the appropriate inviscid wall boundary conditions are
presented here.  For an inviscid wall the only boundary condition required is
the no-penetration condition, $\vn = 0$.  Since the wall boundaries are
curved, in general, the equations and velocity unknowns on the wall boundary
must be locally rotated to a body-normal coordinate system.  Defining the
unit, body normal as $\{n_1,n_2\}^{\rm T}$, the primitive variables in the
body-normal coordinate system are given by
%
\begin{equation}
    \left\{\matrix{\rho\cr \vs\cr \vn\cr w\cr T\cr}\right\} = 
    \left[\matrix{ 1 & & & & \cr & n_2 & -n_1 & & \cr 
                   & n_1 & \ n_2 & & \cr & & & 1 & \cr & & & & 1\cr}\right]
    \left\{\matrix{\rho\cr u\cr v\cr w\cr T\cr}\right\}
\end{equation}
%
which can be written as
%
\begin{equation}
  \bU_s = \bR \bU \period
\end{equation}
%
The matrix, $\bR$ is used to rotate the equations and unknowns at the boundary
to the body-normal coordinate system.  In particular, let $\bR_i$ be the
rotation matrix at node $i$ on the boundary $j=1$.

Before approximate factorization, the linear system that must be solved every
iteration can be written compactly as
%
\begin{equation}
  \bM \delta\bU = \bG \period
\end{equation}
%
A block-diagonal matrix is constructed with the rotation matrix on the
diagonal for nodes on the boundary and the identity matrix elsewhere;
%
\begin{equation}
  \bS = {\rm diag}( \bR_1, \dots, \bR_{N_\xi}, \bI, \dots, \bI)
\end{equation}
%
where the first $N_\xi$ nodes are on the boundary $j=1$.
%
Then the equations and unknowns can be rotated to the new coordinate system
%
\begin{equation}
  [\bS \bM \bS^{\rm T}] (\bS \delta\bU) = (\bS \bG)
\end{equation} 
%
which can be written as
%
\begin{equation}
  \bM_s \bU_s = \bG_s  \period
\end{equation}
%
Thus, on the boundary $j=1$, the velocity unknowns are $(\vs,\vn,w)$ and the
$x$- and $y$-momentum equations have been replaced with the $s$- and
$n$-momentum equations.  In practice, this system of equations is
approximately factored before solving, as discussed in \S\ref{e:base}.

In body normal coordinates at the wall, the impermeable boundary constraint
can be written simply as
%
\begin{equation}
  g_{\vn} = \vn = 0
\end{equation}
%
which is used to replace the normal momentum equation at the boundary (Note
that the tangential momentum-equation is solved as is).  The variation of this
constraint required to form the LHS is simply
%
\begin{equation}
  \delta g_{\vn} = \delta \vn \period
\end{equation}

At the far-field boundary, locally one-dimensional characteristic boundary
conditions \cite{Giles:90} are used to force the plane acoustic wave. In
addition, over the region $r_s=8.5$ to $r_o=13.5$ a sponge term is added to
damp the scattered field while leaving the incident wave unaffected.  This is
accomplished by using the sponge term defined in equation (\ref{e:sponge})
with the reference state, $\bU_{ref}$, equal to the incident, plane acoustic
wave.  The sponge function is defined by equation (\ref{e:spgfun}) with $x$
replaced by $r$, $A_s = 10$, and $N_s = 3$.

The simulation is conducted using the frequency domain approach with $\omega =
k = 2.513$.  Figure~\ref{f:scatter} shows contours of the instantaneous,
scattered pressure field from the simulation demonstrating the damping of the
scattered field in the sponge region.  The acoustic intensity, $\Upsilon = r
|p_s|^2$, for the scattered wave along the ray $\theta=\pi$ is shown in
figure~\ref{f:intense} compared to inviscid theory.  As expected, the acoustic
intensity approaches a constant far from the cylinder which confirms the
expected $1/r$ decay of the scattered wave.  In the sponge, the intensity of
the scattered wave is smoothly damped to zero and there are no noticeable
reflections on this scale.  Often in acoustics the directionality of the
scattered sound is of importance.  Figure~\ref{f:dir} shows a polar plot of
the acoustic intensity at three radial locations.  The top frame, at $r=3.45$,
shows excellent agreement between theory and calculation with the RMS relative
error equal to $0.07\%$ and similar results are obtained for all $r < r_s$.
The middle frame is slightly outside the sponge at $r=8.45$ and we see that
even here the agreement with theory is good with an RMS error of $1\%$.  Note
that most of these errors occur around $\theta=90^\circ$ which is the location
of minimum resolution for the incoming acoustic wave.  The bottom frame, at
$r=11.16$, is well within the sponge and, as expected, the scattered intensity
is significantly reduced at this location.

To assess the level of reflections in the sponge, figure \ref{f:error} shows
the relative error in the computed acoustic intensity along the ray
$\theta=\pi$.  Results are shown for both the original calculation, and a
second calculation in which $r_o=11$ and $N_r = 201$ with all other parameters
the same.  Thus, the second calculation has a sponge that is only one
wavelength long and the number of nodes in $r$ is reduced to keep the number
of points per wavelength constant.  For the original sponge, the RMS error in
the interior region is $8.8 \times 10^{-4}$ while for the shorter sponge, the
RMS error is an order-of-magnitude greater, $8.3 \times 10^{-3}$.  The
oscillations in the error are indicative of both phase errors and reflections,
while the mean error is entirely due to reflections since the spatial scheme
has no inherent amplitude error.  As the sponge is lengthened, both the level
of oscillations, and the mean error are reduced indicating that reflections
are weaker for the longer sponge.

Another means of reducing reflections caused by the sponge is to reduce the
sponge amplitude $A_s$ while keeping the sponge length constant.  Figure
\ref{f:error2} shows the relative error in acoustic intensity along the ray
$\theta=\pi$ for the original calculation, and a third calculation where
$A_s=5$ with all other quantities the same as the original calculation.  For
this case, decreasing $A_s$, reduces the average error with little effect on
the level of oscillations in the error.  The remaining oscillations are due to
the phase error of the finite difference scheme.  We have found for this case,
that further reduction of $A_s$ has negative impact on the solution, because
the sponge is unable to reduce the amplitude of the scattered wave
sufficiently to make it consistent with the far-field boundary condition.

%
%.... Summarize the results
%
In summary, we conclude that our numerical method can accurately predict the
intensity and directionality of acoustic waves scattered from a solid body.
Furthermore, we have found that an inflow sponge, where the reference state
corresponds to an incoming acoustic wave, is a viable technique for enforcing
incoming disturbances while reducing reflections of outgoing disturbances.
For the current problem, a sponge two acoustic wavelengths long is required to
reduce the error due to reflections to the level of the truncation error of
the numerical scheme.

%==============================================================================
%	A C O U S T I C   R E C E P T I V I T Y 
%==============================================================================
\section[Two-Dimensional Leading-Edge Receptivity to Sound]{Two-Dimensional 
Leading-Edge \protect \\ Receptivity to Sound\label{c:2dle} }

In this section, results from two-dimensional calculations are presented for a
flat-plate with a super-ellipse leading-edge.  For this calculation, the
reference length scale is the plate half-thickness, $L^*$, and the reference
velocity is the freestream speed, $u_r^* = u^*_\infty$.  All other reference
quantities are based on their freestream values.  The plate is smooth, unswept
and at zero angle-of-attack.  The super-ellipse geometry is given by
%
\begin{equation}
  \left(1-{{x}\over{\AR}}\right)^m + y^2 = 1
\end{equation}
%
where $\AR=x^*_j / L^*$ is the aspect-ratio of the leading edge (LE) with
$x^*_j$ the location of the junction between the LE and plate; and $m$
determines the level of continuity at the junction.

To assess the ability of our method to predict LE receptivity, we have chosen
the geometry and flow conditions so that our results can be compared to the
previous incompressible results of Lin \cite{Lin:92}.  We therefore use an
$\AR=6$, $m=4$ super-ellipse LE with $\Re=2400$ and $\M=0.1$ where $\M$ is
selected based on mean flow comparisons (shown below) with the incompressible
results.
%
%  The displacement thickness is 0.053 on the outflow plane.  Thus the
%  farfield boundary is 754 \delta^* above the plate!
%

Figure~\ref{SEgrid} shows the computational mesh for this geometry where
symmetry has been used to limit the domain to only the upper surface of the
plate.  A plate length of 40 is used and the farfield boundary is placed 40
units above the plate.  The mesh dimensions are $(384 \times 127)$ and
hyperbolic tangent mapping functions, similar to those used by Lin
\cite{Lin:92}, are used to cluster points near the LE and near the wall.  This
mesh provides at least 30 points in the mean boundary-layer and 18-22 points
per TS wavelength for the receptivity calculation.

\subsection{Mean Flow Computation\label{s:2dmean}}

The first step in performing a receptivity calculation is to determine the
mean flow.  In computing the mean flow, the nonlinear solver discussed in
\S\ref{s:base} is employed.  Adiabatic, no-slip boundary conditions are used
on the plate surface with symmetry enforced directly in the finite-difference
stencil on the boundary upstream of the leading-edge.  On the outflow
boundary, a characteristic based nonreflecting boundary condition with viscous
corrections \cite{PoLe:92} is used while on the inflow boundary first-order
Riemann Extrapolation (see \S\ref{s:inflow}) is performed.  Each of these
boundary conditions is treated fully implicitly to improve convergence to the
steady-state.  Note that after performing the current calculation, an improved
outflow boundary treatment was developed utilizing the parabolized
Navier--Stokes equations (\eg\ \S\ref{sss:ParabolicBC}).  The current
calculations were not rerun, since the outflow boundary treatment is not
expected to have a large impact on the meanflow near the leading-edge.
%
%.... I wish I could rerun these to be sure...
%

The evolution of the displacement thickness, $\delta_1$, along the plate is
shown in figure~\ref{delta}.  The computed result is compared to the Blasius
solution where the virtual origin is set to match the computed solution.  The
computed solution quickly evolves to the Blasius curve downstream of the
junction at $x=6$ and by $x=10$ the development is indistinguishable from the
Blasius curve.  At $x=30$ the solution begins to diverge from the Blasius
curve as the approximate outflow boundary condition begins to influence the
development.  This region of upstream influence does not affect the unsteady
receptivity calculation since this region is inside the outflow sponge (see
\S\ref{s:recep}).

%To verify the farfield inviscid boundary conditions, contours of the pressure
%coefficient, $C_p \equiv 2 ( p - p_\infty )$ are shown in figure~\ref{pfield}
%demonstrating that the inviscid solution is smooth, even near the farfield
%boundaries.  

In comparing our results to those of Lin \cite{Lin:92}, we must verify that
our $\M=0.1$ mean flow is a reasonable approximation to an incompressible
flow.  Figure~\ref{dpds} shows the pressure gradient with respect to
arc-length, $s$, along the plate for a range of Mach numbers.  As $\M$ is
decreased, the curves converge to the incompressible result \cite{Lin:92} and
for $\M=0.1$ the pressure gradient is indistinguishable from the
incompressible curve.  This excellent agreement between our mean flow solution
and Lin's \cite{Lin:92} is also observed in other wall quantities such as the
pressure coefficient (figure \ref{f:cp}) and the wall vorticity (figure
\ref{f:omega}).  Based on these comparisons we believe that our solutions at
$\M=0.1$ are a good approximation to the incompressible solution.
Figure~\ref{profile} shows a streamwise velocity profile, plotted versus the
Blasius variable $\eta_b \equiv y\sqrt{\Re/x}$, in the flat plate region
compared to Lin's results.  Again the comparison is excellent.  However, to
confidently conclude that the stability properties of our mean flow match
those of Lin requires that first- and second-derivatives of the boundary layer
profiles agree.  Unfortunately Lin does not provide these data, but this
comparison can be inferred from comparing the neutral point predictions of
local, parallel LST.  For his mean flow, Lin reports the first neutral point
location to be $x=5.7$ while our analysis yields $x=6.2$ using the parallel
LST solver discussed in Appendix~\ref{a:LST}.  Given the extreme sensitivity
of the neutral point location we judge this roughly 9\% difference to be
acceptable.  Although only parallel LST theory has been applied to this flow,
it is noted that the current results could be refined using the nonparallel
theory developed in Appendix~\ref{s:NPLST}.

\subsection{Receptivity Computation \label{s:recep}}

With the mean flow computed in \S\ref{s:2dmean}, the linear disturbance solver
discussed in \S\ref{s:dist} is used to solve for the response of the boundary
layer to a forced, plane acoustic wave striking the leading edge at normal
incidence.  For comparison to Lin's \cite{Lin:92} results, a nondimensional
forcing frequency $\omega=0.552$ is used, which equates to a frequency
parameter of $\F \equiv \omega / \Re \times 10^6 = 230$.  At $\M=0.1$ this
frequency corresponds to a downstream propagating acoustic wave with
wavelength $\lambda=125$.

When this calculation was performed the more efficient frequency-domain
approach (see \S\ref{ss:frequency}) was not complete.  So, instead, the
unsteady approach was employed as described in \S\ref{ss:unsteady}.  The
time-step is selected to provide $N_t=2400$ time-steps per period of the
acoustic forcing and 2 iterations of the implicit solver are taken at each
step.  Based on the resolution study of \S\ref{ss:TStime}, the spatial and
temporal resolution used should be sufficient to yield TS growth-rates to
within $0.2\%$.

The wall, symmetry, and outflow boundary conditions for the receptivity
calculation are the same as those used to compute the mean flow in
\S\ref{s:2dmean}.  However, the farfield boundary condition is changed to a
locally one-dimensional characteristic boundary condition \cite{Giles:90}
which forces the plane acoustic wave while allowing boundary normal waves to
exit.  This is the same farfield boundary condition used for the cylinder
scattering problem in \S\ref{s:scat}.  See \S\ref{ss:distbc} for a detailed
discussion of boundary condition implementation for the linear disturbance
solver.

To reduce reflections, two sponge terms are added to the NS equations: an
inflow sponge and an outflow sponge.  Figure~\ref{schematic} is a schematic of
the computational domain showing the boundary conditions and sponge regions.
For the inflow sponge, the same sponge used in the cylinder scattering problem
is utilized with $r_s=10$ and $r_0=40$ where $r$ is the normal distance from
the plate.  By construction, this sponge damps all disturbances except the
plane acoustic wave forced on the farfield boundary.  Near the outflow, a
sponge similar to that used in the spatial TS wave problem is utilized with
$x_s=20$ and $x_o=40$.  This sponge damps all disturbances, including the
plane acoustic wave, before they hit the outflow boundary.  Linear stability
theory predicts the wavelength of the TS waves in the boundary layer to be
about $4 L^*$ so that the outflow sponge is approximately $5\lambda_{TS}$
long.  Based on the results of the spatial TS model problem in
\S\ref{ss:TSspace}, this sponge should provide sufficient damping with
negligible reflection of the TS wave.  However, both the inflow and outflow
sponge are only about $16\%$ of an acoustic wavelength in length.  Based the
scattering for a circular cylinder (\eg\ \S\ref{s:scat}), this sponge is very
likely to cause reflections.  However, experience has shown these reflections
to be acceptably small for this problem as demonstrated by the clean
instability wave shown in figure \ref{vclose} below.  Based on this, further
increases in the sponge region were deemed unnecessary for this problem.
%
%.... Need more specifics of the sponges
%

The region in figure~\ref{schematic} where the two sponges overlap requires
special attention.  To prevent incompatibility between the two sponges and the
top boundary condition, the acoustic wave forced at the top boundary and the
wave subtracted off in the inflow sponge must account for the decay of the
plane acoustic wave in the outflow sponge.  This is accomplished by performing
a one-dimensional acoustic wave simulation over the length of the domain
[-40,40] using the same outflow sponge and the same flow conditions.  The
solution to this problem, which now includes the damping in the outflow
sponge, is then forced on the top boundary and subtracted off in the inflow
sponge.

To begin the calculation, the acoustic forcing on the boundary is smoothly
ramped up over one period of the forcing.  The success of the boundary
treatments is evident in instantaneous contours of velocity shown in
figures~\ref{uprime} and \ref{vprime}.  The data in these figures are given
for $t=65.45$ which corresponds to $5.75$ periods of the acoustic forcing.  By
this time in the simulation, TS waves formed at the leading-edge have entered
the outflow sponge so that a full train of TS waves exists on the plate
surface.  The long wavelength acoustic wave is clearly visible in the $u'$
contours of figure~\ref{uprime} with no noticeable distortion in the inflow
sponge.  Included above the contour plot, is the distribution of the Fourier
amplitude of $u'$ specified on the top boundary, showing the decay of the
forcing wave in the outflow sponge.  Returning to the contours of $u'$, the
solution takes the form of a Stokes-wave near the plate which obscures the
development of instability waves.  However, since the Stokes-wave has nearly
zero velocity component in $y$, contours of $v'$ in figure~\ref{vprime} shows
the development and growth of an instability wave on the plate.  Also visible
is the decay of the instability waves in the outflow sponge and the decay of
the scattered acoustic field in the inflow sponge.

Figure~\ref{vclose} again shows contours of $v'$ at $t=65.45$ but with the
sponge regions removed for clarity.  At this particular phase of the forcing,
the amplitude of the scattered acoustic wave and Stokes-wave near the LE
region is negligible so that the development of the instability wave is
clearly observed even close to the LE.  However, taking the Fourier transform
in time (denoted by $\hat v'$) and plotting the amplitude in figure~\ref{vamp}
shows the large scattered acoustic field and the Stokes-wave near the LE.
Also observed in this figure is the initial decay and then growth of the
instability wave in the boundary layer.

To extract the amplitude of the instability wave from the total disturbance
signal requires some form of post-processing.  For incompressible calculations
this is typically accomplished by performing a separate, no-flow, scattering
calculation for the same geometry.  After taking the Fourier transform in
time, the Fourier coefficients of the scattering solution are subtracted from
the Fourier coefficients of the receptivity solution leaving only the
instability wave component \cite{Lin:92}.  However, for compressible flows
there is a change in the acoustic wavelength between no-flow and finite flow
conditions and this change increases with Mach number.  For our $\M=0.1$
calculation, this amounts to a 10\% difference in the wavelengths between flow
and no-flow conditions.  Even for this modest difference, we have found that
subtracting the Fourier coefficients from a no-flow scattering calculation
does not adequately remove the acoustic component.

To analyze the present results, we use the complex plane decomposition
developed by Wlezien \cite{Wlezien:94}.  As an example of this technique,
figure \ref{vmax_phase} shows the trajectory of the Fourier coefficient of the
vertical component of disturbance velocity plotted in the complex plane.  The
large spiral curve shown in the figure is parameterized by $x$ and represents
the Fourier coefficient at each $x$ taken at the $y$ location where the
magnitude is maximum.  Each symbol on the curve represents data from a
different $x$ location in the grid starting at $x=6$ and going to $x=20$.  If
the disturbance consisted solely of a TS wave, the curve would spiral around a
fixed center, first spiraling inward from $x=6$ to $x=9$ (corresponding to the
initial decay of the TS wave) followed an outward spiral due to the growth of
the instability wave.  However, as seen in the figure, the center of the
spiral is not fixed, but appears to move as one follows the curve.  The
changes in the center of the spiral are due to the presence of the scattered
acoustic field resulting from the interaction of the incoming acoustic wave
and the leading edge.  To separate the TS and acoustic components, Wlezien
\cite{Wlezien:94} suggests that three consecutive samples along the spiral can
be used to, locally, define a circle as shown schematically in figure
\ref{cpd_schematic}.  The center of this circle approximates the acoustic
component, while the radius approximates the TS component of the total
disturbance.  Clearly, the success of this techniques depends on a large
wavelength disparity between the TS wave and the acoustic wave which for our
calculation is roughly 1:30.  The amplitude of the TS wave must also be on the
order of, or greater than that of the acoustic wave for this technique to be
useful.  Using the complex plane decomposition, the TS and acoustic components
of the disturbance have been extracted and their amplitudes are shown in
figure \ref{vmax_corr} compared to the total disturbance amplitude.  Clearly,
the modulation in the total amplitude was due to interference between the two
wave types.  The extracted TS wave amplitude has significantly less
oscillations while the acoustic amplitude is of small amplitude and roughly
constant in the flat plate region, as expected.

Figure \ref{tscomp} shows the extracted TS wave Fourier amplitude compared to
the results of Lin \cite{Lin:92}.  Recall that these results are taken at the
local maximum in $y$ of $|\hat v'|$ for each $x$ location.  From the figure we
see that our results are in excellent qualitative agreement with the
incompressible results.  The Branch I locations (different from those of LST
due to nonparallel effects) are in excellent agreement, although our
calculation predicts the TS wave amplitude at Branch I to be 10\% higher than
that predicted by Lin.  The source for this difference is not apparent,
however the difference is of the same order as the difference in the LST
predicted Branch I locations discussed in \S\ref{s:2dmean}.  Although, not
performed here, it would be interesting to compare the current results to
predicts based on the nonparallel linear stability theory of
Appendix~\ref{s:NPLST}.

To verify that the signal extracted by the complex plane decomposition is in
fact a TS wave, figure~\ref{efun} shows the extracted $\hat v'$ variation in
the $y$-direction compared to the $v'$-eigenfunction from parallel LST, both
at $x=12$.  The agreement is excellent, confirming that the decomposition has
extracted a TS wave from the full disturbance solution.  Ideally, one would
like to perform a similar decomposition for other variables such as $\rho'$,
$u'$, and $T'$.  However, since the TS wave amplitude is small compared to the
forced/scattered acoustic wave, the complex plane decomposition is unable to
extract the TS component for these variables.

The final results in this section are for the same super-ellipse geometry but
with the Mach number increased to 0.8.  Although this problem is somewhat
academic since the pressure distribution is unrealistic for a wing in flight,
it does serve as an introduction to high-speed acoustic receptivity
calculations.  The mean pressure gradient along the plate for $\M=0.8$ is
shown in figure~\ref{dpds}.  The most important observation from this plot is
that both the magnitude and the streamwise extent of the adverse pressure
gradient are significantly larger compared to the $\M=0.1$ case.  This
suggests that the flow will be more unstable as is verified by LST which
predicts the Branch~I location to be $x=3.0$ and the maximum growth rate to be
an order of magnitude greater than the $\M=0.1$ case for the same frequency,
$\F=230$.

The receptivity calculation is conducted in the same manner as the $\M=0.1$
case with $\F=230$ and figure~\ref{mhigh} shows the evolution of the local
maximum of the instability wave amplitude along the plate.  These results are
again computed using the complex plane decomposition \cite{Wlezien:94} where
the instability to acoustic wavelength ratio is now only 1:6.5.  However, due
to the large amplitude of the instability wave relative to the acoustic wave,
nearly identical results are obtained by plotting the total disturbance
amplitude without performing the complex plane decomposition.  Because of
this, we can obtain disturbance mode-shapes for all the primitive variables
for this case.  These curves are compared in figure~\ref{mheig} to the LST
eigenfunction demonstrating conclusively that the primary response of the
boundary layer is a TS wave.

\section{Summary}

For all three model problems---unstable TS wave, growing CF vortices, and
acoustic scattering---the numerical method presented in Chapter \ref{c:method}
produces accurate solutions when compared to exact theoretical results.  From
these studies we have determined that for 20 points per wavelength and 128
points in the wall normal direction (with suitable non-uniform mesh
distribution away from the wall), both TS and CF average growth-rates can be
predicted to within $0.2\%$.  Given the domain lengths anticipated for
receptivity studies this level of accuracy ensures that the amplitude errors
are less than $1\%$ over the entire domain.  To achieve this level of accuracy
for spatially growing TS waves requires a sponge outflow region of
approximately five instability wavelengths.  In contrast, this same level of
accuracy can be achieved for CF vortices with a sponge length of only one
wavelength.  We believe that the inviscid character of the CF instability
makes it less susceptible to upstream influence caused by the sponge.  For the
case of acoustic scattering from a circular cylinder, we have shown that an
inflow sponge term can also be used to enforce an incoming acoustic
disturbance while damping outgoing scattered waves.  This capability is vital
to the study of acoustic receptivity which is the subject of the final
validation study in which an incoming plane acoustic wave strikes a flat-plate
with a super-ellipse leading-edge.  The geometry and flow conditions are
modeled after an incompressible calculation performed by Lin \cite{Lin:92} and
the mean flow at Mach number 0.1 is shown to be in excellent agreement with
Lin's incompressible mean solution.  For the receptivity calculation, acoustic
waves with $\F=230$ are forced on the far-field boundary using the ``inflow''
sponge treatment developed for the cylinder scattering problem.  Performing a
complex plane decomposition\cite{Wlezien:94} on the solution yields a TS wave
amplitude that qualitatively agrees with Lin's results and compares to within
10\% at the first neutral point.  The $y$-variation of the $v'$-component of
the decomposed solution is shown to match the eigenfunction shape predicted
from local, parallel, linear stability theory---verifying that the boundary
layer response is indeed a TS wave.  Increasing the Mach number to $\M=0.8$
for this geometry results in a mean boundary-layer with a severe adverse
pressure gradient leading to large TS growth rates which are verified in the
receptivity results.

%==============================================================================
%
%  Tables and Figures
%
%==============================================================================
%
%.... Tollmien--Schlichting waves
%
\begin{figure}[p]
\centering
\sethlabel{ $t$ }
\setvlabel{ $\sigma$ }
\epsfxsize=5.0in \epsfboxo{figures/ch4/ttsgrowth.eps}
\caption[Evolution of the temporal growth-rate for a TS wave] {Evolution of the
temporal growth-rate for a TS wave with $N_t=1150$: \solid is LST, \dashed is
a computed solution with $N_x=20$ and $N_y=127$, and \dotted is a computed
solution with $N_x=80$ and $N_y=127$. \label{f:ttsgrowth}}
\end{figure}
%
\begin{figure}[p]
\centering
\sethlabel{ $N_x, N_y$ }
\setvlabel{ $(\bar\sigma - \sigma_{LST}) / \sigma_{LST}$ }
\epsfxsize=4.5in \epsfbox{figures/ch4/spconv.ai}
\caption[Spatial convergence of the temporal growth-rate for a TS wave]
{Spatial convergence of the temporal growth-rate for a TS wave with
$N_t=1150$.  The \solid line denotes grid refinement in $x$ with $N_y=127$,
while the \dashed line denotes grid refinement in $y$ with
$N_x=80$. \label{f:spconv} }
\end{figure}
%
\begin{figure}[p]
\centering
\sethlabel{ $N_t$ }
\setvlabel{ $(\bar\sigma - \sigma_{LST}) / \sigma_{LST}$ }
\epsfxsize=4.5in \epsfbox{figures/ch4/tmconv.ai}
\caption[Temporal convergence of the temporal growth-rate for a TS wave]
{Temporal convergence of the temporal growth-rate for a TS wave with $N_x=80$
and $N_y=127$.  The \solid line is for 1 iteration, the \dashed line for 2
iterations, and the \dotted line for 3 iterations. \label{f:tmconv} }
\end{figure}
%
%.... Spatial TS
%
\begin{figure}[p]
\centering
\sethlabel{$x$}
\setvlabel{$\sigma$}
\figlab 1.0in 2.4in {\a} 
\epsfxsize=5.0in \epsfboxo{figures/ch4/sts1.eps}
\bigskip
\sethlabel{$x$}
\setvlabel{$\sigma$}
\figlab 1.0in 2.4in {\b} 
\epsfxsize=5.0in \epsfboxo{figures/ch4/sts2.eps}
\caption[Evolution of the spatial growth-rate for a TS wave with sponge to
zero] {Evolution of the spatial growth-rate for a TS wave with sponge to zero:
\solid is LST and \dashed are the computed results.  Frame \a shows results
from the $5 \lambda_{TS}$ domain while the $10 \lambda_{TS}$ results are shown
in frame \b. \label{f:stsgrowth}}
\end{figure}
%
\begin{figure}[p]
\centering
\sethlabel{$x$}
\setvlabel{$\sigma$}
\epsfxsize=5.0in \epsfboxo{figures/ch4/sts3.eps}
\caption[Evolution of the spatial growth-rate for a TS wave using a sponge to
the LST solution] {Evolution of the spatial growth-rate for a TS wave using a
sponge to the LST solution: \solid is LST and \dashed is the computed result.
\label{f:stsgrowth2}}
\end{figure}
%
\begin{figure}[p]
\centering
\sethlabel{$x$}
\setvlabel{$\sigma$}
\epsfxsize=5.0in \epsfboxo{figures/ch4/sts4a.ai}
\caption[Effect of sponge length on the spatial growth-rate of a TS
wave] {Effect of sponge length on the spatial growth-rate of a TS wave: \solid
is LST, \dashed uses a $\lambda_{TS}$ sponge, \dotted a $2.5\lambda_{TS}$
sponge, and \chndash\ a $5\lambda_{TS}$ sponge.
\label{f:stsgrowth3}}
\end{figure}
%
%.... Crossflow vorticies
%
\begin{figure}[p]
\centering
\sethlabel{$t$}
\setvlabel{$\sigma$}
\epsfxsize=5.0in \epsfboxo{figures/ch4/tcfgrowth.eps}
\caption[Evolution of the temporal growth-rate for a CF vortex] {Evolution of
the temporal growth-rate for a CF vortex: \solid is LST, \dashed is a computed
solution with $N_x=20$ and $N_y=127$, and \dotted is a computed solution with
$N_x=80$ and $N_y=127$. \label{f:tcfgrowth}}
\end{figure}
%
\begin{figure}[p]
\centering
\sethlabel{$t$}
\setvlabel{$\sigma$}
\epsfxsize=5.0in \epsfboxo{figures/ch4/scfgrowth.eps}
\caption[Evolution of the spatial growth-rate for a CF vortex] {Evolution of
the spatial growth-rate for a CF vortex: \solid is LST, \dashed is a computed
solution with $N_x=201$ and $N_y=127$. \label{f:scfgrowth}}
\end{figure}
%
%.... Acoustic scattering
%
\begin{figure}[p]
\centering
\figlab -0.156in 2.46in {$y$} 
\figlab 5.0in 0.0in {$x$} 
\epsfxsize=5.0in \epsfbox{figures/ch4/cyl/scat.ai}
\caption[Real component of the scattered pressure disturbance field] {Real
component of the scattered pressure field. Contour are from -0.5 to 0.5 with
increment $0.05$.  Solid contours denote positive pressure perturbations,
dotted contours denote negative pressure perturbations and the shaded region
is the sponge; tics every cylinder radii. \label{f:scatter}}
\end{figure}
%
\begin{figure}[p]
\centering
\sethlabel{$x$}
\setvlabel{$\Upsilon_s$}
\epsfxsize=5.0in \epsfboxo{figures/ch4/cyl/intense.eps}
\caption[Comparison of acoustic intensity] {Comparison of acoustic intensity
along the ray $\theta=\pi$ for a circular cylinder subject to an incoming
planar acoustic wave with $\lambda=2.5$. The \solid line is inviscid theory
while the \dashed line is from the simulation. \label{f:intense}}
\end{figure}
%
%.... Directionality results
%
\begin{figure}[p]
\centering
\figlab 1.25in 1.5in {$\a \quad r = 3.45$} 
\epsfxsize=6.0in \epsfbox{figures/ch4/cyl/d5e.48.eps}
\vskip 0.25in
\figlab 1.25in 1.5in {$\b \quad r = 8.45$} 
\epsfxsize=6.0in \epsfbox{figures/ch4/cyl/d5e.144.eps}
\vskip 0.25in
\figlab 1.25in 1.5in {$\c \quad r = 11.16$} 
\epsfxsize=6.0in \epsfbox{figures/ch4/cyl/d5e.192.eps}
\caption[Polar plot of scattered acoustic intensity] {Polar plot of scattered
acoustic intensity, $\Upsilon$, for an incoming planar acoustic wave with
$\lambda=2.5$. The \solid line is inviscid theory while the \dashed line is
from the simulation. \label{f:dir} }
\end{figure}
%
%.... Error plots
%
\begin{figure}[p]
\centering
\sethlabel{$x$}
\setvlabel{Percent error $(\Upsilon_s)$}
\epsfxsize=5.0in \epsfbox{figures/ch4/cyl/error.eps}
\caption[Effect of sponge-length on the relative-error in scattered acoustic
intensity] {Effect of sponge-length on the relative-error in scattered acoustic
intensity along the ray $\theta=\pi$ for a circular cylinder subject to an
incoming planar acoustic wave with $\lambda=2.5$. \label{f:error}}
\end{figure}
%
\begin{figure}[p]
\centering
\sethlabel{$x$}
\setvlabel{Percent error $(\Upsilon_s)$}
\epsfxsize=5.0in \epsfbox{figures/ch4/cyl/error2.eps}
\caption[Effect of sponge-amplitude on the relative-error in scattered
acoustic intensity] {Effect of sponge-amplitude on the relative-error in
scattered acoustic intensity along the ray $\theta=\pi$ for a circular
cylinder subject to an incoming planar acoustic wave with
$\lambda=2.5$. \label{f:error2}}
\end{figure}
\clearpage
%
%.... Super Ellipse (mean flow)
%
\begin{figure}[p]
\centering
\epsfxsize=4.25in \epsfbox{figures/ch4/grid.eps}
\caption [Computational mesh for the flat-plate with $\AR=6$, $m=4$
super-ellipse leading-edge]{Computational mesh for the flat-plate with
$\AR=6$, $m=4$ super-ellipse leading-edge.  Total resolution is [384,127]
however, only every third line is shown for clarity. \label{SEgrid} }
\end{figure}
%
\begin{figure}[p]
\centering
\sethlabel{$x$}
\setvlabel{$(\delta^*)^2$}
\epsfxsize=5.0in 
\epsfbox{figures/ch4/delta_wtheory.eps}
\caption [Evolution of displacement thickness squared over the
plate]{Evolution of displacement thickness squared over the plate.  The \solid
line denotes the Blasius solution and the \dotted line is the computed
solution. \label{delta} }
\end{figure}
%
% \begin{figure}[p]
% \centering
% \epsfxsize=4.0in \epsfbox{figures/ch4/pres_025.eps}
% \caption [Pressure coefficient contours for $\M=0.1$ mean flow]{Pressure
% coefficient contours for $\M=0.1$ mean flow: Contours every $\pm 0.025$.
% Solid lines denote positive $C_p$, dotted lines denote negative
% $C_p$. \label{pfield} }
% \end{figure}
%
\begin{figure}[p]
\centering
\sethlabel{$x$}
\setvlabel{$\partial p / \partial s$}
\epsfxsize=5.0in \epsfbox{figures/ch4/dpds_new.eps}
\caption {Pressure gradient along the plate for several Mach numbers.  The
incompressible result is taken from \protect\cite{Lin:92}.
\label{dpds} }
\end{figure}
%
\begin{figure}[p]
\centering
\sethlabel{$x$}
\setvlabel{$C_p$}
\epsfxsize=5.0in \epsfbox{figures/ch4/cp_comp.eps}
\caption [Comparison of the pressure coefficient]{Comparison of the pressure
coefficient, $C_p$: \solid is the $\M=0.1$ result and \dashed is the
incompressible result \protect\cite{Lin:92}.
\label{f:cp} }
\end{figure}
%
\begin{figure}[p]
\centering
\sethlabel{$x$}
\setvlabel{$\omega_z$}
\epsfxsize=5.0in \epsfbox{figures/ch4/vorticity.eps}
\caption [Comparison of the wall vorticity] {Comparison of the wall vorticity,
$\omega_z$: \solid is the $\M=0.1$ result and \dashed is the incompressible
result \protect\cite{Lin:92}.
\label{f:omega} }
\end{figure}
%
\begin{figure}[p]
\centering
\sethlabel{$u$} 
\setvlabel{$\eta_b$} 
\epsfxsize=3.0in \epsfbox{figures/ch4/u_profile.ai}
\caption [Streamwise velocity profile at $x=22.7$ for $\M=0.1$]{Streamwise
velocity profile at $x=22.7$ for $\M=0.1$. The \solid line is the
incompressible result \protect\cite{Lin:92}, the \ldashed line is the $\M=0.1$
result. \label{profile} }
\end{figure}
%
%.... acoustic receptivity results
%
\begin{figure}[p]
\centering
\epsfxsize=5.0in \epsfbox{figures/ch4/schematic.eps}
\caption [Schematic of the computational domain]{Schematic of the
computational domain showing boundary conditions for unsteady receptivity
calculations. \label{schematic} }
\end{figure}
%
\begin{figure}[p]
\centering
\epsfxsize=5.0in \epsfbox{figures/ch4/uprime.eps}
\caption [Contours of instantaneous streamwise velocity at
$t=65.45$.]{Contours of instantaneous streamwise velocity at $t=65.45$.
Contours from $-0.05$ to $0.05$ with increment $0.002$.  Solid contours
indicate positive velocity while dotted contours indicate negative velocity.
Included above the contour plot is the Fourier amplitude of $u'$ specified on
the top boundary. \label{uprime} }
\end{figure}
%
\begin{figure}[p]
\centering
\epsfxsize=6.0in \epsfbox{figures/ch4/vprime.eps}
\caption [Contours of vertical velocity at $t=65.45$.]{Contours of vertical
velocity at $t=65.45$. Contours from $-0.002$ to $0.002$ with increment
$0.0002$.  Solid lines indicate positive velocity while dotted lines indicate
negative velocity. \label{vprime} }
\end{figure}
%
\begin{figure}[p]
\centering
\epsfxsize=5.5in \epsfbox{figures/ch4/vclose.ai}
\caption [Closeup of the vertical disturbance velocity contours at
$t=65.45$.]{Closeup of the vertical disturbance velocity contours at
$t=65.45$.  Contours from $-0.002$ to $0.002$ with increment $0.0002$.  Solid
lines indicate positive velocity while dotted lines indicate negative
velocity.  \label{vclose}}
\end{figure}
%
\begin{figure}[p]
\centering
\epsfxsize=5.5in \epsfbox{figures/ch4/vamp.ai}
\caption [Contours of the Fourier amplitude of the vertical disturbance
velocity.]{Contours of the Fourier amplitude of the vertical disturbance
velocity.  Transform taken over the sixth period of the forcing.  Contours
from $0$ to $0.02$ with increment $0.0002$. \label{vamp}}
\end{figure}
%
% The complex plane decomposition
%
\begin{figure}[p]
\centering
\sethlabel{$\mbox{Re}(\hat v')$}
\setvlabel{$\mbox{Im}(\hat v')$}
\epsfxsize=4.0in \epsfbox{figures/ch4/vmax_phase.eps}
\caption [Trajectory of the Fourier coefficient, $\hat v'$]{Trajectory of the
Fourier coefficient, $\hat v'$, in the complex plane.  \hbox{\solid\kern
-.25in\solidcircle} ~denotes the trajectory in $x$ of the maximum in $y$ of
$|\hat v'|$.  The \solid line in the center of the figure denotes the acoustic
component obtained after using the complex-plane
decomposition. \label{vmax_phase}}
\end{figure}
%
\begin{figure}[p]
\centering
\epsfxsize=3.0in \epsfbox{figures/ch4/cpd_schematic.ai}
\caption [Schematic of the complex-plane decomposition]{Schematic of the
complex plane decomposition. \label{cpd_schematic}}
\end{figure}
%
\begin{figure}[p]
\centering
\sethlabel{$x$}
\setvlabel{$|\hat v'|_{max}$}
\epsfxsize=5.25in \epsfbox{figures/ch4/vmax_corr.eps}
\caption [Decomposition of the local maximum of the disturbance
amplitude]{Decomposition of the local maximum of the amplitude of vertical
component of disturbance velocity, \solid is the TS wave amplitude,
\dashed is the total amplitude, and \dotted is the amplitude of the
acoustic component. \label{vmax_corr}}
\end{figure}
%
% Continue with the receptivity results
%
\begin{figure}[p]
\centering
\sethlabel{$x$}
\setvlabel{$|\hat{v}'|_{max}$}
\epsfxsize=4.9in \epsfbox{figures/ch4/vmax_comp.eps}
\caption [Comparison of the local maximum of the TS wave amplitude along the
plate.]{Comparison of the local maximum of the TS wave amplitude based on the
vertical component of disturbance velocity along the plate.  The \solid line
denotes the $\M=0.1$ result and the \dashed line denotes the incompressible
result \protect\cite{Lin:92}. \label{tscomp} }
\end{figure}
%
\begin{figure}[p]
\centering
\sethlabel{$y$}
\setvlabel{$\mbox{Re}(\hat v')$, $\mbox{Im}(\hat v')$}
\epsfxsize=4.6in \epsfbox{figures/ch4/efun.eps}
\caption [Extracted wave amplitude compared to LST at $x=12$ for
$\M=0.1$.]{Extracted wave amplitude for the vertical component of disturbance
velocity compared to LST at $x=12$ for $\M=0.1$. \solid is the real component
of the computed solution, \ldashed is the imaginary component of the computed
solution, \dashed is the real component of the LST solution, and \dotted is
the imaginary component of the LST solution. \label{efun} }
\end{figure}
%
% M=0.8 TS amplitude
%
\begin{figure}[p]
\centering
\sethlabel{$x$}
\setvlabel{$|\hat{v}'|_{max}$, $|\hat{u}'|_{max}$}
\epsfxsize=5.2in \epsfbox{figures/ch4/norm_M=0.8b.eps}
\caption [Comparison of the local maximum of the TS wave amplitude along the
plate for $\M=0.8$.]{Comparison of the local maximum of the TS wave amplitude
along the plate for $\M=0.8$.  The \solid line denotes the $|\hat{v}'|_{max}$
result and the \dashed line denotes the $|\hat{u}'|_{max}$
result. \label{mhigh} }
\end{figure}
%
% M=0.8 eigenfunctions
%
\clearpage
\begin{figure}[p] \label{mheig:a}
\centering
\figlab 4.0in 2.70in {\a} 
\sethlabel{$y$}
\setvlabel{$\mbox{Re}(\hat \rho')$, $\mbox{Im}(\hat \rho')$}
\epsfxsize=4.5in
\epsfbox{figures/ch4/rho_efun.eps}
\end{figure}

\begin{figure}[p] \label{mheig:b}
\centering
\figlab 4.0in 2.70in {\b} 
\sethlabel{$y$}
\setvlabel{$\mbox{\Re}(\hat u')$, $\mbox{Im}(\hat u')$}
\epsfxsize=4.5in
\epsfbox{figures/ch4/u_efun.eps}
\figtit{\protect\ref{mheig}}{See following page for caption.}
\end{figure}

\begin{figure}[p] \label{mheig:c}
\centering
\figlab 4.0in 2.70in {\c} 
\sethlabel{$y$}
\setvlabel{$\mbox{Re}(\hat v')$, $\mbox{Im}(\hat v')$}
\epsfxsize=4.5in
\epsfbox{figures/ch4/v_efun.eps}
\end{figure}

\begin{figure}[p]
\centering
\figlab 4.0in 2.70in {\d} 
\sethlabel{$y$}
\setvlabel{$\mbox{Re}(\hat T')$, $\mbox{Im}(\hat T')$}
\epsfxsize=4.5in
\epsfbox{figures/ch4/t_efun.eps}
\caption [Extracted wave amplitudes compared to LST at $x=12$ for
$\M=0.8$.]{Extracted wave amplitudes compared to LST at $x=12$ for $\M=0.8$:
\solid is the real component of the computed solution, \ldashed is the
imaginary component of the computed solution, \dashed is the real component of
the LST solution, and \dotted is the imaginary component of the LST
solution. Frame \a shows density, \b streamwise velocity, \c vertical
velocity, and \d temperature.
\label{mheig} }
\end{figure}
