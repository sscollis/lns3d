%%%%%%%%%%%%%%%%%%%%%%%%%%%%%%%%%%%%%%%%%%%%%%%%%%%%%%%%%%%%%%%%%%%%%%%%%%%%%%%
%
%  Appendix 5
%
%  S. Scott Collis
%
%  Written: 9-5-95
%
%  Revised: 9-18-96
%
%%%%%%%%%%%%%%%%%%%%%%%%%%%%%%%%%%%%%%%%%%%%%%%%%%%%%%%%%%%%%%%%%%%%%%%%%%%%%%%
\chapter[Finite Reynolds-Number Receptivity Theory]
{Compressible, Finite Reynolds-Number Receptivity Theory \label{a:FRNT}}

Finite Reynolds-Number Theory (FRNT) is based on the work of Zavol'skii \etal\
\cite{ZaReRy:83} who approximated the linearized Navier-Stokes equations
locally using the quasi-parallel disturbance equations.  This was done within
the same receptivity framework used by Goldstein \cite{Goldstein:85} and Ruban
\cite{Ruban:84} (who themselves used high Reynolds number asymptotics).
Recently FRNT has been applied by Crouch \cite{Crouch:92,Crouch:93} and
Choudhari \& Streett \cite{ChSt:92,ChSt:94} to investigate receptivity
phenomena under a variety of situations.  However, most applications of FRNT
have been for incompressible flow with the notable exception of the work of
Zhigulev \& Fedorov \cite{ZhFe:87} for two-dimensional roughness.  In the
following, we outline the straightforward extension of FRNT to compressible
flow for three-dimensional boundary layers. The development follows the
exposition given by Choudhari \cite{Choudhari:94} with appropriate extensions.

In FRNT, the linearized Navier-Stokes equations are approximated by the
quasi-parallel disturbance equations.  Towards that end, two length scales are
defined: $l^*$ corresponding to the streamwise location of the wall
disturbance downstream of the leading-edge and $L^*$ which is proportional to
the undisturbed boundary layer thickness at $s^* = l^*$.  $L^*$ is defined
relative to $l^*$ through the relation
%
\begin{equation}
  L^* = \frac{l^*}{\sqrt{\Re_l}} ,
\end{equation}
%
where $\Re_l \equiv U^*_e(s^* = l^*) l^* / \nu^*$ is the Reynolds number based
of the local, slip-velocity tangent to the surface at the wall disturbance
location.  The length-scale $L^*$ is used to scale both the height of the bump
and its variations in the streamwise and spanwise directions.  With this
scaling, the streamwise direction is given by $S = (s^* - l^*) / L^*$, the
wall normal direction by $N = n^* / L^*$, and the scaled spanwise direction by
$Z = z^*/L^*$.  The global coordinates $(s^*,n^*,z^*)$ when scaled by $l^*$
are denoted as $(s,n,z)$.

Our primary focus is on the generation of stationary cross-flow vortices so
that we consider small-amplitude, spanwise periodic wall disturbances
(relative to the undisturbed wall, $N=0$) of the form
%
\begin{equation} \label{e:bump}
  N_w(S,Z) = \ew h_w(S) e^{i \bw Z}
\end{equation}
%
where $\ew = \ew^* / L^* \ll 1$ is the scaled roughness height, $h_w(S)$ is
the streamwise shape of the roughness, and $\bw = \bw^* L^*$ is the spanwise
wavenumber of the bump.  Since the analysis is linear, the current results can
be thought of as the response of the boundary layer to a single Fourier mode
of a more complicated spanwise distribution.

The flow is denoted by the vector field $\bU = \{\rho,u,v,w,T\}^{\rm T}$ and,
if the amplitude of the wall disturbance is sufficiently small, the dependent
variables can be expanded as a regular perturbation series in $\ew$
%
\begin{equation}
  \bU(S,N,Z) = \bUm(N) + \ew \bU_w(S,N) e^{i \bw Z} + \O(\ews) .
\end{equation}
%
Consistent with FRNT, in writing this expression we utilized the fact that
$L^*/l^* \ll 1$ so that the undisturbed mean boundary layer flow $\bUm(s,N)$
can be replaced with the local, quasi-parallel approximation $\bUm(N) \equiv
\bUm(s=1,N)$.

The mean-flow disturbance, $\bU_w$, is obtained by solving the Navier-Stokes
equations linearized about the parallel mean flow, $\bUm(N)$.  The boundary
conditions for this problem are the usual no-slip, no-penetration conditions
applied at the perturbed wall location, $N(S,Z) = \ew h_w(S) e^{i \bw Z}$.  A
detailed derivation of the linearized boundary conditions (transfered to
$N=0$) for compressible flow over curved surfaces is given in
Appendix~\ref{a:bumpbc}.  In addition, the solutions also satisfy
zero-disturbance conditions as $N \rightarrow \infty$.

After taking the Fourier transform $(S \rightarrow \alpha)$ the linearized
Navier-Stokes equations with the quasi-parallel approximation take the form of
the compressible stability equations which are the compressible analogue to
the Orr-Sommerfeld equation.  These equations can be formally written as
%
\begin{equation}
  {\cal L}^{(\alpha,\bw,\Re,\Pr,\M)} \hat \bU_w = 0
\end{equation}
%
where the hat denotes the Fourier transform
%
\begin{equation}
  \hat \bU_w(\alpha,N) = \frac{1}{\sqrt{2\pi}} \int_{-\infty}^{\infty} 
                         \bU_w(S,N) e^{-i \alpha S} dS .
\end{equation}
%
Appendix~\ref{a:LST} presents the compressible stability equations in detail
and describes the methods used to obtain eigensolutions for the homogeneous
problem.  In the present case, we require solutions to the same equations, but
with inhomogeneous boundary conditions at the wall.  These solutions are
obtained with the same Chebyschev spatial discretization used in
Appendix~\ref{a:LST} but by including the boundary conditions in the discrete
approximation and using Gaussian elimination to solve the resulting linear
system.

As discussed in detail by Choudhari \cite{Choudhari:94}, the solution of the
compressible stability equations in Fourier space can be obtained without
regard for causality.  However, when the Fourier inversion,
%
\begin{equation} \label{e:finv}
  \bU_w(x,N) = \frac{1}{\sqrt{2\pi}} \int_{\Gamma} 
                        \hat \bU_w(\alpha,N) e^{i \alpha S} d\alpha ,
\end{equation}
%
is performed to return the solution to physical space, causality must be
accounted for in choosing the inversion contour, $\Gamma$.  Herein, we assume
that the crossflow instability is convective in nature without applying a
rigorous causality condition (\ie\ the Briggs-Bers criterion \cite{Bers:83}).
This same assumption has been used by several researchers in the context of
FRNT \cite{Choudhari:94,Crouch:94}.

The contribution of the integral in (\ref{e:finv}) to a particular
CF-instability mode is given by the residue of the integrand at the complex
streamwise wavenumber of the CF-mode, $\acf$.  The residue theorem states that
%
\begin{equation}
  \int_C f(z) dz = 2\pi i {\rm Res}_{z=z_0}[f(z)] = 2\pi i b_1
\end{equation}
%
where the function $f(z)$ is analytic in $C$ except at $z_0$ which is an
isolated singular point.  If $f(z)$ is written as
%
\begin{equation}
  f(z) = \frac{p(z)}{q(z)}
\end{equation}
%
where both $p$ and $q$ are analytic at $z_0$ and
%
\begin{equation}
  p(z_0) \ne 0, \qquad q(z_0) = 0, \qquad {\rm and} \qquad q'(z_0) \ne 0
\end{equation}
%
then the residue at $z_0$ is
%
\begin{equation}
  b_1 = \frac{p(z_0)}{q'(z_0)} .
\end{equation}
%
Returning to equation (\ref{e:finv}), we have
%
\begin{eqnarray}
  p(\alpha) = e^{i \alpha S} \comma \\
  q(\alpha) = \hat U_n^{-1}(\alpha,N)
\end{eqnarray}
%
Where $\hat U_n$ is a particular component of $\hat \bU_w$.  With these
expressions, the residue at $\acf$ is given by
%
\begin{equation} \label{e:residue}
  U_n(S,N) = \frac{ i \sqrt{2\pi} }
                  { \frac{\partial \hat U_n^{-1}}
                         {\partial \alpha}(\acf,N) } e^{i \acf S} \period
\end{equation}
%
The value of $\frac{\partial \hat U_n^{-1}}{\partial \alpha}(\acf,N)$ is
determined by computing $\hat U_n^{-1} (\alpha,N)$ at $\alpha = \acf \pm
\Delta\alpha$ and using a central difference to approximate the
derivative.

The form of the perturbation corresponding to the residue at $(\acf,\bw)$ is
then given by \cite{Goldstein:85}
%
\begin{equation}
  \bU_{cf} = \hat h_w(\acf) \: \Lambda(\bw,\Re,\Pr,\M) \: 
             \bE(N,\bw,\Re,\Pr,\M) \: e^{i(\acf S + \bw Z)}
\end{equation}
%
where $\hat h_w(\acf)$ is the amplitude of the Fourier component of the
roughness shape-function which is resonant with the crossflow vortex mode. The
factor $\Lambda(\bw,\Re,\Pr,\M)$ is called the ``efficiency'' function which
characterizes the local receptivity process independent of the geometry of the
roughness, and $\bE(N,\bw,\Re,\Pr,\M)$ is the vector of eigenfunctions for the
stationary crossflow vortex under consideration.  Similar to
\cite{Choudhari:94}, the eigenfunctions are normalized such that maximum of
the eigenfunction for the velocity perturbation in the direction of the local
inviscid streamline is unity.  Thus, the initial, complex amplitude of the
crossflow vortex based on this quantity is
%
\begin{equation} \label{e:ampcf}
  A_{cf} = \ew \: \hat h_w(\acf) \: \Lambda(\bw,\Re,\Pr,\M) .
\end{equation}
%
For the purpose of comparing the effectiveness of various parameters on the
receptivity of CF vortices, it is sufficient to examine the magnitude of
$\Lambda(\bw,\Re,\Pr,\M)$.  However, when directly comparing to the results of
experiments or computations, equation (\ref{e:ampcf}) must be used.

To test our implementation of the FRNT, we have examined the receptivity of
the compressible Falkner-Skan-Cooke profile (see Appendix~\ref{a:FSC}) to
surface roughness.  The parameters were chosen to match the incompressible
results of Choudhari \cite{Choudhari:94}.  We use $\M=0.1$, $\Re_l=80,000$,
$\theta=45^\circ$, and $\Pr=1$ with the isothermal boundary condition, $T_w =
T_0$.  Under these conditions, the Branch I neutral point for stationary
crossflow-vortices is found to be at $\bw=0.1589$, $\acf = -0.1169$.  Solving
for the residue yields $\Lambda = -0.01955 + 0.03889 i$ which has a magnitude
of $|\Lambda| = 0.04352$.  This value agrees with Choudhari's
\cite{Choudhari:94} incompressible result to at least two significant figures.

