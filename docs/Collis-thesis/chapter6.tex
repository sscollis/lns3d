%%%%%%%%%%%%%%%%%%%%%%%%%%%%%%%%%%%%%%%%%%%%%%%%%%%%%%%%%%%%%%%%%%%%%%%%%%%%%%
%
%  Chapter 6:  Conclusions
%
%  S. Scott Collis
%
%  Written: 9-5-95
%
%  Revised: 11-8-96
%
%%%%%%%%%%%%%%%%%%%%%%%%%%%%%%%%%%%%%%%%%%%%%%%%%%%%%%%%%%%%%%%%%%%%%%%%%%%%%%%
\chapter{Conclusions and Recommendations\label{c:conclude}}

\section{Conclusions}

The receptivity of the three-dimensional boundary-layer on a high-speed,
swept, parabolic-cylinder with surface roughness has been investigated using
numerical simulation.  Taking advantage of the linearity of the receptivity
process, a numerical method is developed to accurately and efficiently solve
receptivity problems near a swept leading-edge.  In this method, the
Navier--Stokes equations are written in terms of a nonlinear meanflow
solution and a linear disturbance about the meanflow.  Equations are
developed for both the meanflow and disturbances and these equations are
solved in terms of primitive variables using a generalized coordinate system
to represent an arbitrary, infinite-span, leading-edge geometry.
%
% By using primitive variables, the mesh metrics, associated with the
% generalized coordinate system, can be implemented exactly, thereby improving
% accuracy on highly stretched meshes.
%
Spatial derivatives in the plane of the leading-edge are approximated with a
fourth-order central-difference scheme.  For the linearized disturbance
equations, a single Fourier mode is used in the spanwise direction.  Due to
the high spatial resolution required near the leading edge and in the boundary
layer, a fully implicit time-advancement scheme is implemented.  To improve
efficiency for unsteady disturbances, the linearized equations are solved in
the frequency domain.  The implicit equations are solved at every iteration
using approximate factorization with a specially optimized
block-penta-diagonal matrix solver.  All boundary conditions are implemented
fully implicitly to preserve convergence of the iterative solver.

The numerical method has been extensively validated for model problems that
represent the physical phenomena expected in receptivity calculations.  These
include Tollmien-Schlichting waves in a parallel boundary-layer, crossflow
vortices in a parallel boundary-layer, and acoustic scattering from a circular
cylinder.  Exact solutions are available for each of these problems and the
current numerical solutions are shown to converge to the exact solutions.  In
performing these model calculations, the resolution required to adequately
resolve each physical phenomena is established.  As a further test of the
method, the problem of leading-edge receptivity to sound is investigated for a
flat-plate with a super-ellipse leading-edge.  The geometry and flow
conditions are modeled after an incompressible calculation performed by Lin
\cite{Lin:92}.  Our meanflow solution at $\M=0.1$ is shown to be in very good
agreement with the incompressible solution.  Furthermore, our acoustic
receptivity results qualitatively agree with Lin's solution, and the amplitude
of the Tollmien-Schlichting wave at the first neutral-point is within 10\% of
Lin's value.

With the computational method validated, the receptivity of a
three-dimensional boundary-layer on a swept, infinite-span parabolic-cylinder
subject to surface roughness is investigated.  This geometry serves as a
first-order approximation to the leading-edge of a swept-wing.  Both two- and
three-dimensional mean solutions are obtained.  The two-dimensional solutions
at low Mach numbers compare well with the reference incompressible solutions
of Davis \cite{Davis:72}.  The three-dimensional meanflow over a swept
parabolic cylinder is obtained for conditions which are modeled after a
transonic commercial airliner in cruise condition.  In particular, the
Reynolds number, when converted to an equivalent wing-chord takes the value of
$10^7$, which is in the realm of flight Reynolds numbers.  The characteristics
of this three-dimensional boundary-layer are documented including linear
stability predictions incorporating curvature and nonparallel effects.  The
linear stability code solves the full, compressible, stability-equations in
curvilinear coordinates.  Unlike recent stability analyses for swept circular
cylinders, we allow for chordwise variations of the surface curvature in our
formulation.  Nonparallel effects are accounted for using a perturbation
approach by allowing the solution to vary slowly in the chordwise direction.
By including compressibility, curvature, and nonparallel effects our
linear-stability-code represents the current ``state-of-the-art'' in local
stability analysis.

The stability analysis for the swept parabolic-cylinder shows that convex
surface curvature reduces the growth-rate, while nonparallel effects increase
the growth-rate of stationary crossflow vortices.  These results are in
qualitative agreement with recent analysis for a swept circular-cylinder
\cite{MaMa:94}.  Stability calculations performed using the full Linearized
Navier--Stokes (LNS) equations are used to provide reference solutions which
have an exact account of nonparallel effects.  Comparing the nonparallel
stability results to the LNS solutions shows that for short wavelength modes,
$\beta=100$, the nonparallel theory agrees well with the LNS solutions.
However, for longer wavelengths, $\beta=35$, the nonparallel theory breaks
down near the leading-edge causing qualitative, as well as quantitative,
errors in the stability characteristics.

Receptivity to surface roughness is investigated by modeling a roughness
element as a spanwise periodic, linear perturbation of the wall boundary with
a Gaussian distribution in the chordwise direction.  Receptivity results are
obtained both from LNS solutions with surface-roughness boundary-conditions
and from Finite Reynolds Number Theory (FRNT) predictions.  The theoretical
predictions are determined both with and without surface curvature to identify
the effect of curvature on receptivity.  Based on the theoretical predictions,
curvature is shown to enhance the receptivity efficiency over the entire
unstable region and the curvature effect is greatest near the lower branch of
the neutral curve.  The impact of nonparallel flow on the receptivity of
crossflow vortices is established by a series of LNS calculations with surface
roughness placed at various locations on the wall.  Comparison to the FRNT
shows that nonparallel flow significantly attenuates the initial amplitude of
stationary crossflow vortices downstream of a roughness element near the
leading-edge.  The effect is most severe for the long wavelength $\beta=35$
case, where FRNT over-predicts the amplitude of the crossflow mode by as much
as $77\%$.  For the shorter wavelength case, $\beta=100$, the maximum error in
the FRNT prediction is $29\%$.  Sufficiently far from the leading-edge, the
theoretical predictions approach the LNS solution and are within $5\%$ of the
LNS solution at the furthest downstream stations considered here.

By considering a range of bump widths, we have verified the FRNT result that
the receptivity amplitude can be written as the product of the Fourier
coefficient of the chordwise roughness distribution at the local crossflow
wavenumber and a function representing the efficiency of the receptivity
process.  Based on this form of the solution, the receptivity efficiency
function is extracted from the linearized Navier--Stokes solutions.  The
results indicate conclusively that nonparallel effects dramatically reduce
receptivity near the leading-edge.  For the cases considered here, the FRNT
over-predicts the efficiency function by as much as a factor of 3 near the
first neutral point.  These results clearly indicate that the prediction of
crossflow receptivity near a realistic leading-edge must account for the
strongly nonparallel flow near the upstream neutral point.

\section{Recommendations for Future Research}

The most pressing need for future research is in the addition of nonparallel
corrections to the FRNT receptivity predictions.  Based on the current
results, the inclusion of nonparallel effects is vital for the accurate
prediction of crossflow receptivity for realistic geometries.  The current
solutions provide a basis upon which enhanced theories can be evaluated.

Another area for future research is the investigation of nonlinear receptivity
mechanisms for surface roughness in a three-dimensional boundary layer.  A
recent theoretical investigation by Choudhari \& Duck \cite{ChDu:96} suggests
that receptivity is likely to become nonlinear even before the stability
characteristics of the boundary layer are modified by the surface roughness.
This early nonlinear effect may be partially responsible for the extreme
sensitivity to surface roughness observed by Radeztsky \etal\ \cite{RaReSa:93}
in their experiments.  Numerical simulations of nonlinear receptivity due to
surface roughness would provide valuable information on this relatively
unexplored phenomenon including: the critical bump height for the onset of
nonlinearity receptivity; the possibly nonlinear wavelength selection
processes; and the early onset of nonlinear, modal interactions.

The application of Parabolized Stability Equations (PSE) to the study of
receptivity is also an avenue for future research.  Recent work with the PSE
method for the prediction of receptivity \cite{HeLi:93} has involved the
imposition of inhomogeneous boundary conditions on the initial condition for
the PSE marching procedure.  Unfortunately, this method appears to rely on the
approximate transient behavior during the startup of PSE marching.  Therefore,
the relationship of the boundary condition to the final instability waves
downstream is unclear.  Comparisons to linearized Navier--Stokes solutions,
such as those presented here, are required to determine the capabilities and
limitations of PSE for receptivity investigations.  A related area for
research involves the coupling of PSE to the LNS calculations performed here.
This offers the advantage of computing LNS solutions on short domains to
capture the receptivity process, and then feeding the established boundary
layer disturbances into a PSE code for subsequent linear and nonlinear
evolution.

Finally, recent research by Lingwood \cite{Lingwood:95,Lingwood:96} on the
nature of the crossflow instability on the rotating disk has shown that
absolute instability is possible for this flow.  However, the
absolute/convective nature of the crossflow instability for the infinite-span
swept-wing is currently unknown, although all current evidence indicates that
it is purely convective.  As suggested by Tobak \cite{Tobak:96}, the
absolute/convective nature of this flow can be investigated using the
techniques developed in this \thesis\ by performing a transient calculation
where the surface roughness boundary condition is applied impulsively at the
start of the calculation.  The development of the resulting wave packet can
then be tracked similar to that done in the experiments of Lingwood
\cite{Lingwood:96}.
