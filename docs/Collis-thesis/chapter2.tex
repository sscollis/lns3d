%%%%%%%%%%%%%%%%%%%%%%%%%%%%%%%%%%%%%%%%%%%%%%%%%%%%%%%%%%%%%%%%%%%%%%%%%%%%%%%
%
%  Chapter 2:  Governing Equations
%
%  S. Scott Collis
%
%  Written: 9-5-95
%
%  Revised: 9-6-96
%
%%%%%%%%%%%%%%%%%%%%%%%%%%%%%%%%%%%%%%%%%%%%%%%%%%%%%%%%%%%%%%%%%%%%%%%%%%%%%%%
\chapter{Governing Equations \label{c:eqn}}

This chapter presents the compressible Navier--Stokes equations in a form
amiable to the numerical simulation of receptivity phenomena.  The equations
are first discussed in dimensional form followed by a conversion to
nondimensional variables and an introduction of the relevant nondimensional
parameters.  In the final section, the equations are linearized about a steady
base flow, and these linearized Navier--Stokes equations form the basis for our
receptivity study.

\section{Compressible Navier--Stokes Equations \label{s:deqn}}

The compressible Navier--Stokes (NS) equations in primitive form using
Cartesian tensor notation and dimensional variables can be written as
%
  \begin{equation} \label{e:dcon}
    \rho^*_{,t^*} + \left(\rho^* u^*_i \right)_{,i} = 0 ,
  \end{equation}
  \begin{equation} \label{e:dmom}
    \rho^* u^*_{i,t^*} + \rho^* u^*_{j} u^*_{i,j} =  -p^*_{,i} + \tau^*_{ij} ,
  \end{equation}
  \begin{equation} \label{e:deng}
    \rho^* h^*_{,t^*} + \rho^* u^*_{j} h^*_{,j} =
    \left( p^*_{,t^*} + u^*_{j} p^*_{,j} \right) + 
    q^*_{j,j} + \tau^*_{ij} u^*_{i,j} ,
  \end{equation}
%
where $\rho^*$ is the density, $u^*_i$ is the velocity vector, $p^*$ is the
thermodynamic pressure, $\tau^*_{ij}$ is the viscous stress tensor, $q^*_{j}$
is the heat flux vector, and $h^*$ is the fluid enthalpy defined by
%
  \begin{equation} \label{e:denthalpy}
    h^* = e^* + p^* / \rho^* , 
  \end{equation} 
%
where $e^*$ is the internal energy per unit mass.  In the above equations,
subscripts following a comma denote partial differentiation with respect to
the subscript, and the Einstein summation convention is followed.

We assume a Newtonian fluid such that the viscous stress tensor and the heat
flux vector are given by
%
  \begin{equation} \label{e:dtau}
    \tau^*_{ij} = \lambda^* u^*_{k,k} \delta_{ij} + 2 \mu^* S^*_{ij},
  \end{equation}
%
  \begin{equation} \label{e:dheat}
    q^*_{j} = - \kappa^* T^*_{,j},
  \end{equation}
%
where $\mu^*$ is the first coefficient of viscosity, $\lambda^*$ is the second
coefficient of viscosity, $\kappa^*$ is the thermal conductivity,
$S^*_{ij}=\frac{1}{2}(u^*_{i,j}+u^*_{j,i})$ is the rate-of-strain tensor,
$T^*$ is the absolute temperature, and $\delta_{ij}$ is the Kronecker delta.
  
The system of equations (\ref{e:dcon})--(\ref{e:deng}) is closed with an
equation-of-state, relating the thermodynamic variables, and additional
equations which relate the fluid properties to the thermodynamic variables.
We use an ideal, calorically perfect gas with constant specific heats such
that
%
  \begin{equation} \label{e:dideal}
    p^* = \rho^* \Rgas^* T^* ,
  \end{equation}
%
where $\Rgas^* = \cp^* - \cv^*$ is the gas constant, $\cp^*$ is the specific
heat at constant pressure, and $\cv^*$ is the specific heat at constant
volume.  Both $\cp^*$ and $\cv^*$ are constant as is the ratio, $\gamma =
{\cp^*} / {\cv^*}$.  Under these conditions, the internal energy and enthalpy
are related to the absolute temperature by
%
  \begin{equation} \label{e:eoft}
    e^* = \cv^* T^* ,
  \end{equation}
%
  \begin{equation}  \label{e:hoft}
    h^* = \cp^* T^* .
  \end{equation}
%
For a calorically perfect gas, the fluid properties (viscosity coefficients
and conductivity) are functions of temperature only.  For low-speed flows with
small temperature variations, the fluid properties can be assumed constant
with little loss of accuracy and this assumption is used for many of the
low-speed results presented here.  However, for high-speed flows the
temperature dependence of the fluid properties becomes important.  Before
describing the relations used to determine the fluid properties, we will first
convert the governing equations to nondimensional variables.

\section{Nondimensional Navier--Stokes Equations \label{s:eqn}}

In equations (\ref{e:dcon})--(\ref{e:hoft}), the superscript ``$*$''
indicates that the quantities are dimensional. In nondimensionalizing these
equations the following relations have been used,
%
\begin{equation} \label{e:nondimen}
  \begin{array}{lll}
    \rho    = \frac{\rho^*}{\rho^*_r} , & 
    u_i     = \frac{{u^*_i}}{u^*_r} , &
    T       = \frac{{T^*}}{T^*_r} , \\[0.1in]
    p       = \frac{{p^*}}{\rho^*_r {u^*_r}^2} , &
    \mu     = \frac{\mu^*}{\mu^*_r} , &
    \lambda = \frac{\lambda^*}{\mu^*_r} , \\[0.1in]
    \kappa  = \frac{\kappa^*}{\kappa^*_r} , &
    x_i     = \frac{{x^*_i}}{L^*} \ , &
    t       = \frac{{t^*}}{L^* / u^*_r} , \nonumber
  \end{array}
\end{equation}
%
where $L^*$ is some convenient reference length and quantities denoted with a
subscript ``$r$'' are reference values typically taken to be far-field
ambient conditions.  The particular values of $L^*$, $\rho^*_r$,
$u^*_r$, \etc\ are defined for each problem considered.

Using equations (\ref{e:nondimen}) in equations (\ref{e:dcon}) through
(\ref{e:dideal}) yields the following non-dimensional equations
%
  \begin{equation} \label{e:con}
    \rho_{,t} + \left(\rho u_i \right)_{,i} = 0 ,
  \end{equation}
  \begin{equation} \label{e:mom}
    \rho u_{i,t} + \rho u_{j} u_{i,j} =  -p_{,i} + 
    {1\over{\Re}}\Big[(\lambda u_{j,j})_{,i} + (2 \mu S_{ij})_{,j}\Big] ,
  \end{equation}
  \begin{eqnarray} \label{e:eng}
    \rho T_{,t} + \rho u_{j} T_{,j} + (\gamma - 1) \rho T u_{j,j} =  
    \frac{\gamma}{\Pr\Re} \Big[ \kappa T_{,i} \Big]_{,i} + \nonumber \\
    {{\gamma (\gamma-1) \M^2}\over{\Re}} 
    \Big[ \lambda S_{ii} S_{jj} + 2 \mu S_{ij} S_{ij} \Big] ,
  \end{eqnarray}
  \begin{equation} \label{e:ideal}
    p = \frac{\rho T}{\gamma \M^2} ,
  \end{equation}
%
where 
%
\begin{equation}
  \M =  \frac{u^*_r}{c^*_r} , \qquad 
  \Re = \frac{\rho^*_r u^*_r L^*}{\mu^*_r} , \qquad
  \Pr = \frac{\mu^*_r \cp^*}{\kappa^*_r} , 
\end{equation}
%
are the Mach number, Reynolds number, and Prandtl number respectively.  In
deriving the non-dimensional energy equation (\ref{e:eng}) from
(\ref{e:deng}), use is made of equation (\ref{e:hoft}) to convert enthalpy to
temperature and the continuity equation (\ref{e:dcon}) along with the
equation-of-state (\ref{e:dideal}) are used to remove the explicit pressure
dependence.

For low-speed conditions where the dependence of fluid properties on
temperature is negligible, $\mu=1$.  For cases where the temperature
dependence of the fluid properties is important, $\mu$ is related to the
temperature through Sutherland's law
%
  \begin{equation} \label{e:mu}
    \mu = T^{\frac{3}{2}} \frac{1 + S}{T + S} ,
  \end{equation}
%
where $S = S_0^* / {T^*_r}$ and $S_0^* = 111 K$ for air
\cite{White:91}. The second coefficient of viscosity is computed using Stokes
hypothesis, $\lambda= -\frac{2}{3} \mu$ (giving zero bulk viscosity), and the
thermal conductivity is determined assuming constant Prandtl number,
$\kappa=\mu$.  Unless otherwise specified, the fluid is assumed to be air with
a Prandtl number of 1.0 and a ratio of specific heats, $\gamma$ = 1.4.

\section{Linearized Navier--Stokes Equations \label{s:linNS}}

In determining receptivity or stability characteristics it is often convenient
(where appropriate) to treat disturbances as linear perturbations about a
steady base-flow.  Thus, flow variables are written as
%
\begin{equation}
  \begin{array}{ll}
    \rho    = \bar \rho(x_i)    + \rho'(x_i,t) ,    &
    u_j     = \bar u_j(x_i)     + u_j'(x_i,t) ,     \\
    T       = \bar T(x_i)       + T'(x_i,t) ,       &
    \mu     = \bar \mu(x_i)     + \mu'(x_i,t) ,     \\
    \lambda = \bar \lambda(x_i) + \lambda'(x_i,t) , &
    \kappa  = \bar \kappa(x_i)  + \kappa'(x_i,t) ,  \\
    p       = \bar p(x_i)       + p'(x_i,t) ,       &
    ~
  \end{array}
\end{equation}
%
where a bar denotes base-flow variables and primes denote (linear)
perturbations about that base flow.

Substituting these expressions into equations (\ref{e:con})--(\ref{e:ideal})
and dropping terms that are nonlinear in the perturbations leads to two sets
of equations: one for the base flow and one for the perturbation flow.  The
base-flow equations are just the steady versions of equations
(\ref{e:con})--(\ref{e:ideal}).
%
  \begin{equation} \label{e:bcon}
    \left(\bar\rho \bar u_i \right)_{,i} = 0 ,
  \end{equation}
  \begin{equation} \label{e:bmom}
    \bar\rho \bar u_{j} \bar u_{i,j} =  -\bar p_{,i} + 
    {1\over{\Re}}\Big[(\bar\lambda \bar u_{j,j})_{,i} + 
    (2 \bar\mu \bar S_{ij})_{,j}\Big] ,
  \end{equation}
  \begin{eqnarray} \label{e:beng}
    \bar\rho \bar u_{j} \bar T_{,j} + 
    (\gamma - 1) \bar\rho \bar T \bar u_{j,j} =  
    \frac{\gamma}{\Pr\Re} \Big[ \bar\kappa \bar T_{,i} \Big]_{,i} + 
    \nonumber \\
    {{\gamma (\gamma-1) \M^2}\over{\Re}} 
    \Big[ \bar \lambda \bar S_{ii} \bar S_{jj} + 
    2 \bar \mu \bar S_{ij} \bar S_{ij} \Big] ,
  \end{eqnarray}
  \begin{equation} \label{e:bideal}
    \bar p = \frac{\bar\rho \bar T}{\gamma \M^2} ,
  \end{equation}
%
while the perturbation equations are given by
%
  \begin{equation} \label{e:pcon}
    \rho'_{,t} + \left(\bar\rho u'_i \right)_{,i} + 
    \left(\rho' \bar u_i \right)_{,i}= 0	,
  \end{equation}
  \begin{eqnarray} \label{e:pmom}
    \bar\rho u'_{i,t} + \bar\rho \bar u_{j} u'_{i,j} + 
    \bar\rho u'_{j} \bar u_{i,j} + \rho' \bar u_{j} \bar u_{i,j} =
    -p'_{,i} + \nonumber \\
    \frac{1}{\Re}\Big[(\bar\lambda u'_{j,j})_{,i} +
                      (\lambda' \bar u_{j,j})_{,i} + 
                      (2 \bar\mu S'_{ij})_{,j} + 
		      (2 \mu' \bar S_{ij})_{,j} \Big]	,
  \end{eqnarray}
%
% Steady continuity is used to derive the energy equation 
% ( the term -\bar\rho_{,j} \bar u_j T' = \bar\rho \bar u_{j,j} T' )
%
% Corrected the minus sign on 5-16-97
%
  \begin{eqnarray} \label{e:peng}
    \bar\rho T'_{,t} + \bar\rho \bar u_{j} T'_{,j} + 
              \bar\rho u'_{j} \bar T_{,j} + 
              \rho' \bar u_{j} \bar T_{,j} =
    \nonumber \\
    (1 - \gamma) \Big[ \rho' \bar T \bar u_{j,j} - 
                     \bar \rho_{,j} \bar u_j T' +
                     \bar \rho \bar T u'_{j,j} \Big] +
    \frac{\gamma}{\Pr\Re} \Big[ \bar\kappa T'_{,i} + 
                          \kappa' \bar T_{,i} \Big]_{,i} +
    \nonumber \\
    {{\gamma(\gamma-1) \M^2}\over{\Re}}
    \Big[ 2 \bar \lambda \bar S_{ii} S'_{jj} + 
          \lambda' \bar S_{ii} \bar S_{jj} +
          4 \bar \mu \bar S_{ij} S'_{ij} +
          2 \mu' \bar S_{ij} \bar S_{ij} \Big] ,
  \end{eqnarray}
  \begin{equation} \label{e:pideal}
    p' = (\rhom T' + \Tm \rho') / (\gamma \M^2) .
  \end{equation}
%
Since the coefficients $\mu$, $\lambda$, and $\kappa$ are functions only of
$T$, the linear perturbations of these coefficients and their derivatives are
defined solely in terms of $\bar T$, $T'$, and their derivatives.  For
example,
%
\begin{equation}
  \mu' = \left(\frac{d\mu}{dT}\right)_{\bar T} T' ,
\end{equation}
%
and
%
\begin{equation}
  \mu'_{,j} = \left(\frac{d^2\mu}{{dT}^2}\right)_{\bar T} \bar T_{,j} T' + 
              \left(\frac{d\mu}{dT}\right)_{\bar T} T'_{,j} ,
\end{equation}
%
where the derivatives of $\mu$ are zero for constant properties flow or are
determined from Sutherland's law, equation (\ref{e:mu}).  Similar expressions
for $\lambda'$ and $\kappa'$ follow directly.

With these expressions for the fluid properties and by dividing through by
$\bar\rho$, equations (\ref{e:pcon})--(\ref{e:peng}) can be rewritten using
the more compact notation
%
  \begin{eqnarray} \label{e:pNS}
    \bU'_{,t} + \bA \bU'_{,1} + \bB \bU'_{,2} + \bC \bU'_{,3} + \bD \bU' =
    \bVxx \bU'_{,11} + \bVxy \bU'_{,12} + \nonumber \\
    \bVxz \bU'_{,13} + \bVyy \bU'_{,22} + \bVyz \bU'_{,23} + \bVzz \bU'_{,33}
  \end{eqnarray}
%
where
%
  \begin{equation}
    \bU' = \left\{\matrix{\rho'\cr u'_1\cr u'_2\cr u'_3\cr T'\cr}\right\}
  \end{equation}
%
and the matrices $(\bA,\bB,\bC,\bD,{\bf V}_{ij})$, defined in
Appendix~\ref{s:linmat}, depend only on the mean flow variables and their
gradients.

The form of the linearized, compressible Navier--Stokes equations given in
equation (\ref{e:pNS}) is a suitable starting point for numerical simulation
of receptivity (\S\ref{s:dist}), linear-stability analysis (Appendix
\ref{a:LST}), and parabolized stability analysis \cite{ChMaErHu:93}.
